%\conclusion
\chapter{Conclusion}\label{ch:conclusion}


The work presented here aimed to probe the Fermi surface of the heavy fermion material YbPdBi using angle-resolved magnetoresistance oscillations (AMRO). To contextualize these AMRO measurements, we characterized a range of properties of YbPdBi. These included measurements of its magnetotransport properties, its specific heat, sub-Kelvin neutron diffraction, and its magnetization. In the specific heat measurements, we found that the Schottky-like anomaly occurring below $T=10$~K exhibits a strong dependence on the magnetic field and cannot be well-described using CEF states. In the neutron diffraction data, we found no evidence of a magnetic phase transition down to $T=0.1$~K. In the magnetization data, we found evidence for enhanced Pauli paramagnetism at low temperatures. \\


\noindent From the magnetotransport measurements, we were able to extract evidence of both an anomalous Hall effect (AHE) and a topological Hall effect (THE). An alternative analysis of magnetotransport found linear regions separated by a crossover field strength at low temperatures. The temperature dependence of this crossover field suggests it is related to the $T=1$~K peak in specific heat. Evidence for a chiral anomaly was observed in the longitudinal magnetoresistivity, with chiral coefficients that exhibit the expected temperature dependence. Unexpectedly, this chiral anomaly was observed for all values of $\theta$ between parallel and orthogonal electric and magnetic fields. ARPES measurements of YbPdBi could show direct evidence of Weyl nodes in the bulk and Fermi arcs on the surface.\\

% \noindent Possible evidence for mixed valency of the Yb ions : Brillouin + Pauli fits of magnetization, 3-level fits of Schottky anomaly in zero and non-zero field, saturation entropy of $R \ln 6$ (how to distinguish from just a high-level CEF? Does it really matter if the low-T fits fail, since the upper CEF state should be above the range of T?). LeBras and them: the degeneracies of the excited states were estimated according to M\"ossbauer and iNS measurements (can't trust the iNS measurements). Two-fluid model of heavy fermions has some moments which are itinerant and some are localized. In what ways could this affect magnetization and schottky stuff?\\
 
\noindent During the AMRO measurements, we discovered that the magnetic field induced a change from a two-fold oscillation symmetry to an unexpected four-fold symmetry. This change in symmetry implies that the Fermi surface of YbPdBi has a dependence on magnetic field strength. Let us consider this change in symmetry in the context of the characterization measurements shown in Ch.~\ref{ch:characterization}. \\


%%%%% Transverse Linear Regimes%%%%%
\noindent To be associated with the observed change in AMRO symmetry, related phenomena should exhibit a dependence on $H$ comparable to the AMRO data. Stated more explicitly, related phenomena should demonstrate two distinct regions of behaviour, separated along $H$ by a crossover region, as with the AMRO symmetry. Recall that we have seen this two region behaviour in $\rho_{xy}(H)$ in Sec.~\ref{sec:transverse_res} with the measurements of $\rho_{xy}(H)$ and $H_{Cross}$. However, at $T=10$~K and $T=15$~K, we see no change in AMRO symmetry for $\mu_0 H <9$~T, which would need to occur since $\mu_0H_{cross}<9$~T at these temperatures.\\

% At first, $H_{Cross}$ does not seem to correspond to values of $H$ and $T$ at which the AMRO would change symmetry. However, we calculated $H_{Cross}$ to measure the temperature dependence of a gradual change in $\rho_{xy}(H)$ from one linear regime to another. 

% \noindent Let us associate the two-fold symmetry with the regime of $\rho_{xy}(H)$ with a larger slope below $H_{cross}$ and the four-fold symmetry with the smaller slope above $H_{cross}$. Comparing the crossover against the corresponding AMRO data shows that the gradual change of AMRO symmetry overlaps at least partially with the crossover between linear regimes of $\rho_{xy}(H)$. This is most evident comparing $H_{Cross}$ for $T=2$~K and $T=5$~K against the relevant AMRO data in Fig.~\ref{fig:fitted_actrot11_4x4grid} and Fig.~\ref{fig:fitted_actrot11_4x4grid}. It is harder to make this association for the data at $T=10~$K and $T=15$~K.\\

%%%%% magnetotransport %%%%%
\noindent %If the linear regimes of $\rho_{xy}(H)$ are indeed associated with the two-fold and four-fold AMRO symmetry, the fit of Eqn.~\ref{eq:h_cross_function} to the $T$-dependence of $H_{cross}$ implies that at $\mu_0H=0$~T the change in AMRO symmetry should occur at $T=(0.9\pm0.2)$~K. This would then associate the change of AMRO symmetry with the $T=1$~K peak seen by Lebras \emph{et al}.~\cite{lebras_local_1995}. 
Note that though the longitudinal magnetoresistance $\rho_{xx}(H)$ in Sec.~\ref{sec:magnetoresistivity} shows some crossover behaviour between WAL and $H^2$ behaviour, this crossover doesn't shift significantly in $H$ as $T$ is decreased. \\


%%%%% Schottky-like anomaly %%%%%
\noindent We saw in Sec.~\ref{sec:cv_section} that, between $T=3$~K and $T=50$~K, the Schottky-like anomaly doubled in strength as $\mu_0H\xrightarrow[]{}9$~T. Similar behaviour was observed for isostructural YbPtBi~\cite{mun_yb-based_nodate}, where a sub-Kelvin peak in specific heat was pushed to higher temperatures by the application of $H$. We require additional specific heat measurements to determine whether this is happening for the $T=1$~K peak of YbPdBi. \\

\noindent Moreover, a comprehensive understanding of the observed chiral anomaly and its connection to our observed THE demands further investigation. Measurements of the planar angle-resolved magnetoresistivity $\rho_{xy}^{PAMR}(H)$ would provide additional evidence for or against the presence of the chiral anomaly. The aforementioned sub-Kelvin specific heat measurements in non-zero field could also show evidence for a Weyl node contribution to specific heat~\cite{guo_evidence_2018, lai_weylkondo_2018}. \\



%%%%% Magnetization %%%%%
\noindent The magnetization measurements in Sec.~\ref{sec:mag_sus} exhibit changes in behaviour within the relevant range of $T$. The fits of the Brillouin function and Pauli magnetization to $M(H)$ in Sec.~\ref{sec:mag_pauli_brill_fits} suggest a reduction in the strength of the total angular momentum $J$ for the Yb$^{3+}$ ions, perhaps attributable to a growth in the number of heavy fermion quasiparticles below $T=15$~K. The inverse magnetic susceptibility $1/\chi$ at $\mu_0H=0.1$~T shows deviations from Curie-Weiss behaviour beginning below $T=20$~K. \\


%%%%% Topology %%%%%
\noindent Regarding our observations suggesting non-trivial topology, it is difficult to associate the AMRO symmetry with the chiral coefficient $c_a$ in Sec.~\ref{sec:chiral_mc}. Nonetheless, $c_a$ changes significantly in the relevant region of $T$. Though the $H$ behaviour of the oscillatory THE term calculated in Sec.~\ref{sec:hall_AHE_THE} lacks two regions separated by a crossover, the amplitude of THE oscillations does grow as $T$ is lowered below $T=15$~K, correlating with the $T$-dependencies of the symmetry amplitudes. \\


%%%%% higher symmetries %%%%%
\noindent Regarding the AMRO data itself, the presence of higher symmetries is difficult to address. At a surface-level consideration, it seems noteworthy that these higher symmetries are both related to the two-fold and four-fold oscillations by simple arithmetic operations (i.e. $2+4=6$ and $2\cross 4=8$). Though we did restrict the AMRO fits to consider only these symmetries, we did so because they are among the strongest symmetries in the majority of the Fourier transforms of the AMRO data. The presence of these higher symmetries could imply that these AMRO techniques are sensitive to smaller FS details than expected.\\


%%%%% Outlook %%%%%

\noindent It is evident that YbPdBi possesses intriguing physics stemming from its magnetism. Future studies should focus on elucidating the precise mechanisms governing the observed change in AMRO symmetry and its connection, if any, with the Schottky-like anomaly in the specific heat and with the crossover behaviour in the transverse resistivity. Additional specific heat measurements in $\mu_0H\neq0$~T from above the Schottky-like anomaly to sub-Kelvin temperatures would be an excellent next step. If we observe that the $T=1$~K peak is shifted into the Schottky-like anomaly by increasing $H$, this would associate the Schottky-like anomaly with the $T=1$~K peak and the fit of $H_{Cross}$, and therefore to magnetotransport properties of YbPdBi. \\

\noindent Further investigations of the magnetic ground states are also recommended. Local structure measurements such as X-ray absorption fine structure (XAFS) spectroscopy and/or the neutron  pair distribution function (PDF) could be used to  probe if the local structure is really being distorted~\cite{booth_local_2001,booth_lattice_1998}. This would help to explain the M\"ossbauer spectra seen previously in~\cite{lebras_local_1995}.\\
