\chapter{Synthesis and Characterization}\label{ch:characterization}
% TODO
% \begin{enumerate}
%     \item Fix graphs independent parameters as $\mu_0H$
%     \item Ensure the appendices are adequately filled out
%     \item Review each section
% \end{enumerate}


This chapter will detail the synthesis and characterization of single crystal samples of the half-Heusler compound YbPdBi. Synthesis was done by flux-growth methods and characterization was performed using a range of methods. From the measurement of longitudinal and transverse resistivities, we were able to extract both an anomalous Hall resistivity and a topological Hall resistivity. Through specific heat measurements, we find an unexpected $H$-dependence of a Schottky-like anomaly and examine the low-lying excited states of the magnetic moment of the ytterbium atom. We find evidence for Pauli paramagnetism at very low temperatures via magnetization measurements. Using elastic neutron diffraction, we find no evidence of magnetic ordering at $T=1$~K, contrary to previous measurements of the specific heat and of the magnetic susceptibility. \\

% Transport measurements and specific heat measurements were performed in a Quantum Design (QD) Physical Property Measurement System (PPMS). DC magnetization measurements were performed in a QD MPMS3 SQUID. Powder X-ray diffraction (PXRD) measurements were performed on an Empyrean diffractometer by Malvern Panalytical.

%%%%%%%%%%%%%%%%%%%%%%%
%%%%%%%%%%%%%%%%%%%%%%%1
%%%%%%%%%%%%%%%%%%%%%%%
\section{Measurement Attributions}


Angle-resolved magnetoresistance $\rho_{xx}(H, \theta)$ measurements and transverse resistivity $\rho_{xy}(H,T)$ were taken by Dr. Michael Nicklas at the Max Planck Institute for Chemical Physics of Solids.\\

\noindent Magnetization measurements $M(T, H)$ were taken by Raphael François and Prof. Cris Adriano at UNICAMP Universidade Estadual de Campinas.\\

\noindent Single crystal and powder X-ray diffraction measurements were taken by Dr. Daniel Chartrand and Dr. Thierry Maris at l'Universit\'e de Montr\'eal.\\

\noindent Neutron diffraction data were obtained with the help of Dr. Denis Sheptyakov on the High-Resolution Powder Diffractometer for Thermal Neutrons in an experiment at the Swiss Spallation Neutron Source (SINQ) at the Paul Scherrer Institut in Villigen, Switzerland.\\

% Specific heat All other measurements and all data analysis, sample growth, and sample puck preparation were performed by James Fraser. (except the one taken by Seb, which showed the low T anomaly for Bi and the LeBras data). I feel like Mathieu did something that I'm forgetting.\\

\noindent Invaluable support through the maintenance of the Physical Property Measurement System and the helium recovery system from Mathieu Desmarais, William Dupuis, Sebastien Laughrea, Avner Fitterman, Nils Lamouche, Dr. Gabrielle Beaudin, lab technician Robert Rinfret, and Prof. Andrea Bianchi.\\


%%%%%%%%%%%%%%%%%%%%%%%%%%
%%%%%%%%%%%%%%%%%%%%%%%%%%
%%%%%%%%%%%%%%%%%%%%%%%%%%
\section{Synthesis by Flux Growth}\label{sec:growth}

The growth of high-purity, single crystals is an important first step in the search for novel materials. Though a myriad of methods for sample growth exist~\cite{rosa_flux_2018, tachibana_beginners_nodate}, here we focus on the flux growth of single crystals from metallic solutions. In flux growth, constituent elements of a desired material are combined in stoichiometric ratios. An element with a low melting point (such as bismuth) is then added in greater quantity, which ensures a suitable chemical environment for the precipitation of the desired crystals. This so-called “flux” is comparable to a solvent in which a solute is dissolved. If one uses a constituent element of the desired compound, it is called a self-flux, but an alternative element may be used as the flux when required. These materials are melted and allowed to homogenize. The resulting liquid solution is then cooled slowly, following a metastable curve of solute concentration against temperature. The desired compound precipitates out of the solution as crystalline samples which are separated from the flux using a centrifuge.\\



%%%%%%%%%%%%%%%%%%%%%%%
%%%%%%%%%%%%%%%%%%%%%%%
\subsection{YbPdBi Synthesis}

All samples of YbPdBi were grown using one of two flux growth recipes. Elemental ytterbium (Yb), palladium (Pd), and bismuth (Bi) were added to an aluminum oxide crucible in either a 1~:~1~:~7 ratio with a Bi self-flux or a 1~:~1~:~1~:~11 ratio with a lead (Pb) flux. The crucible was then sealed in a quartz ampoule under argon (Ar) pressure. Quartz wool was placed above and below the crucible, which helped to secure it in place in the ampoule. Later on, during the centrifuging stage, the quartz wool doubles as a filter that separates the crystal samples from the remaining liquid. \\

\noindent For the Bi-flux recipe, the now-sealed ampoules were heated over the course of two days from room temperature to 1150$^\circ$C, held there for two hours for homogenization, and then cooled to 500$^\circ$C at a rate of 4$^\circ$C per hour. After reaching this final temperature, the crucibles were centrifuged to separate the single crystals from the remaining liquid flux.  For the Pb-flux recipe, the ampoules were heated to 1150$^\circ$C and held there for 48 hours for homogenization. They were then cooled to 850$^\circ$C over the span of 168 hours. \\


\begin{figure}[h]
    \centering
    \includegraphics[width=0.8\linewidth]{Figures_Appendices/Crystal Photos app/JF036 Example Crystal With Measurment.pdf}
    \caption[Microscope Photo of YbPdBi Crystals Grown in Bi-Flux]{A selection of YbPdBi single crystals grown in bismuth self-flux. Though twinning is prevalent in three of the four crystals shown, it is much less prevalent in the results of this recipe than the image might imply. Millimetre paper is shown at the top of the image for scale.}
    \label{fig:Jf036_Bi_crystal}
\end{figure} 


\noindent It should be noted that the Bi self-flux method was initially preferred out of concerns regarding Pb impurities. Contrary to this, it was found late in our measurements that the Bi-flux method had an unidentified impurity phase which shows a strong peak in the specific heat at $T\sim13$~K. Per App.~\ref{app:Pb_vs_Bi_flux}, we can show it has minimal influence on transport measurements, but we recommend the Pb-flux method for future sample growth. where appropriate, it is noted in the body text Whether measurements were performed on the Pb-flux samples or on the Bi-flux samples.\\


\begin{figure}[h]
    \centering
    \includegraphics[width=0.9\linewidth]{Figures_Ch4_Synth_Charac/JF039_YbPdBi_PXRD_Results.pdf}
    \caption[PXRD Refinement for YbPdBi Grown in Bi-Flux]{PXRD refinement results for YbPdBi samples grown in a bismuth (Bi) self-flux. There is good agreement between the observed peaks (blue) and the fitted peaks (green). The presence of an impurity phase is visible between $2\theta=30^\circ$ and $2\theta=40^\circ$, where the difference (cyan) shows peaks and the weighted difference (black) shows peaks and a broad rise. A Chebyshev polynomial (red) was fitted to the background.  Refinement was performed using GSAS-II~\cite{toby_gsas-ii_2013}.}
    \label{fig:sample_PXRD_bi_flux}
\end{figure}

\noindent A selection of crystals grown in Bi-flux is shown in Fig.~\ref{fig:Jf036_Bi_crystal} with millimetre paper for scale; although twinning is present in three of the four crystals shown, such twinning was less prevalent overall, and the crystal sizes were more varied than the image may suggest. The lattice structures of the samples were typically characterized by powder X-ray diffraction (PXRD) due to the high X-ray absorption rate of the Yb atoms. However, the structure of the Pb-flux samples was confirmed using single crystal X-ray diffraction with a lattice constant of $a=6.5784(2)$~\AA. The fit report for the single-crystal XRD can be found in App.~\ref{app:singl_xtal_XRD}.\\



\noindent Shown in Fig.~\ref{fig:sample_PXRD_bi_flux} is a Rietveld refinement on PXRD data of Bi-flux samples. Using GSAS-II, we confirmed the identity of the samples and found a lattice constant of $a=6.5905(1)$~\AA. Shown in blue are the measured X-ray intensities as a function of $2\theta$, with the fitted reflections shown in green. The red line shows the background calculated using a Chebyshev polynomial; cyan is the difference between the observed and the calculated peaks. The reflections of an impurity phase are visible between $2\theta=30^\circ$ and $2\theta=40^\circ$, clearly visible in the difference (cyan) and weighted difference (black) lines. We were unable to identify this impurity phase using various compounds consisting of combinations of Yb, Pd, and/or Bi. \\



%%%%%%%%%%%%%%%%%%%%%%%
%%%%%%%%%%%%%%%%%%%%%%%
\subsection{YPdBi Synthesis}
We required a non-magnetic analogue of YbPdBi to approximate the phonon contribution to the specific heat. To address this, samples of YPdBi were grown using the Bi-flux method. We found that a simple substitution of the non-magnetic yttrium (Y) in place of Yb was sufficient, with the addition of arc melting the constituent elements in a 1:1:1 ratio beforehand. Initially, without arc melting, we observed an exothermic reaction which split the crucible in half, leaving behind only a red metallic residue. Arc melting alleviated the intensity of this reaction, which allowed single crystals to be produced. Although some metallic residue was still present on the crucible, the process did not crack the crucible. \\

\noindent Shown in Fig.~\ref{fig:YPdBi_PXRD} is the Rietveld refinement of PXRD data for YPdBi. We find good agreement between the observed reflections (blue) and the calculated reflections (green) with a lattice constant of $a=6.6194(3)$~\AA.  The peaks at higher $2\theta$ are underestimated, and the peaks are over-estimated at lower $2\theta$, suggesting the powdered sample has a preferred orientation.\\


\begin{figure}[h]
    \centering
    \includegraphics[width=\linewidth]{Figures_Appendices/JF045_YPdBi_PXRD_Results.pdf}
    \caption[PXRD Refinement for YPdBi Grown in Bi-Flux]{PXRD refinement results for YPdBi samples grown in a bismuth (Bi) self-flux. There is good agreement between the observed peaks (blue) and the fitted peaks (green). Their difference (cyan) and their weighted difference (black) are also shown. A Chebyshev polynomial (red) was fitted to the background.  Refinement was performed using GSAS-II~\cite{toby_gsas-ii_2013}.}
    \label{fig:YPdBi_PXRD}
\end{figure}

\noindent It should be noted that the growth of YPdBi was suboptimal. Hopper morphology~\cite{tachibana_beginners_nodate} was a dominant feature of the single crystals. Such morphology would make connecting contacts for electron transport experiments difficult. Further, it seems there is again an impurity phase causing peaks within the same range of $2\theta$ as for YbPdBi, suggesting that the impurity phase may consist of only Pd and/or Bi. It could also be that we have simply substituted Y for Yb in the impurity phase. Again, we were unable to fit these peaks using various combinations and structures of Yb, Pd, and/or Bi.\\


%%%%%%%%%%%%%%
%%%%%%%%%%%%%%
%%%%%%%%%%%%%%
% \newpage
\section{Resistivity Measurements}\label{sec:resistivity}


%%%%%%%%%%%%%%
%%%%%%%%%%%%%%
\subsection{Resistivity $\rho_{xx}(T)$}\label{sec:thermoresistivity}

As was presented in Sec.~\ref{sec:HF_Kondo_Lattice}, one of the characteristic signatures of heavy fermion materials (HFM's) lies in the temperature dependence of its resistivity $\rho_{xx}(T)$~\cite{coleman_heavy_2007, kondo_resistance_2006}. In a normal metal, $\rho_{xx}(T)$ decreases monotonically as the temperature $T$ is lowered, eventually reaching some minimum value at zero temperature. For a heavy fermion material, however, as the temperature is lowered from room temperature, Kondo scattering causes $\rho_{xx}(T)$ to gradually increase. Eventually, a maximum is reached at the HFM's coherence temperature $T^*$, below which there is a sharp decrease in the resistivity. This sharp decrease is due to the formation of heavy fermion quasiparticles, which scatter coherently. $T^*$ is associated with a range of quantities, including a proposed order parameter of the two-fluid model for HFM's~\cite{yang_scaling_2008, yang_two-fluid_2016, yang_emerging_2022}.\\




\noindent Shown in Fig.~\ref{fig:thermo_res_0_field} is the longitudinal resistivity $\rho_{xx}(T)$ of a sample of YbPdBi grown using the Pb-flux method. Gold wires were attached to the surface using conductive silver epoxy. The current $I$ ran parallel to the [100] crystal axis, across which we measured $\rho_{xx}(T)$. The gradual increase in $\rho_{xx}(T)$ as the temperature is lowered is indicative of Kondo scattering~\cite{kondo_resistance_1964, coleman_heavy_2015}. A maximum then occurs at a coherence temperature $T^*=34\pm4$~K, below which there occurs a dramatic drop in $\rho_{xx}(T)$. This drop is attributable to the onset of coherent scattering caused by the formation of heavy fermion quasiparticles (HFQPs)~\cite{coleman_heavy_2007,  coleman_heavy_2015}.\\




\noindent Previous measurements of $\rho_{xx}(T)$ showed that it has a quadratic dependence below $T=1$~K, with a residual resistivity of $\rho_{0xx}\approx50$~$\mu$ohm-cm and peak resistivity of $\approx 460$~$\mu$ohm-cm~\cite{kaczorowski_magnetic_1999}. Our resistivity data behaves comparably to what was presented there. However, our $\rho_{xx}(T)$ is larger by an order of magnitude. The difference is possibly related to the fact that our samples were single crystals, whereas their samples were polycrystals.\\
%We attribute this to the silver epoxy used to attach the gold wires to the sample, which was also found to interfere with preliminary AMRO experiments. This was addressed by switching to spot-welding the wire contacts. I do not believe that.
\begin{figure}[h]
    \includegraphics[width=\linewidth]{Figures_Ch4_Synth_Charac/Resistivity/YbPdBi, JF011, Thermoresistivity at Zero field.pdf}
    \caption[$\rho_{xx}(T)$ for $2~\text{K}\leq T\leq300~\text{K}$]{Longitudinal resistivity $\rho_{xx}(T)$ of YbPdBi grown in Pb-flux. 
    % The increase in resistivity as $T$ is lowered is due to the Kondo effect, and the sharp drop below this is due to the onset of the heavy fermion phase.
    The dashed vertical black line represents the heavy fermion coherence temperature $T^*=34\pm4$~K.}
    \label{fig:thermo_res_0_field}
\end{figure}

\noindent Shown in Fig.~\ref{fig:thermo_res_in_field_zoomed}a  is $\rho_{xx}(T, H)$ where a magnetic field $H$ is applied perpendicularly to $I$. As $H$ is increased, the maximum value of the resistivity is reduced and $T^*$ is shifted upward to $T\approx 50$~K. 
% This reduction reflects the suppression of Kondo scattering due to $H$~\cite{pietri_magnetoresistance_2000, coleman_heavy_2015}.
The $H$-dependence of $T^*$ is shown in Fig.~\ref{fig:thermo_res_in_field_zoomed}b, increasing with $H$. These values of $T^*$ were obtained by fitting second-degree polynomials to a range of resistivities near each maximum. The error bars shown are the standard deviations of the fitted values. The continued existence of the resistivity maximum and the sharp decrease as the $T$ is lowered suggests that the HF phase is robust up to $\mu_0H=9$~T.\\


\begin{figure}[H]
    \centering
    \begin{subfigure}[t]{0.14\textwidth}
        \textbf{a)}
    \end{subfigure}    
    \centering
    \begin{subfigure}[t]{0.75\textwidth}  
        \includegraphics[width=\linewidth, valign=t]{Figures_Ch4_Synth_Charac/Resistivity/ZOOMED YbPdBi, JF011,Thermoresistivity at Multiple fields.pdf}
    \end{subfigure}
    \centering
    \begin{subfigure}[t]{0.14\textwidth}
        \textbf{b)}
    \end{subfigure}
    \centering
    \begin{subfigure}[t]{0.75\textwidth}  
        \includegraphics[width=\linewidth, valign=t]{Figures_Ch4_Synth_Charac/Resistivity/YbPdBi, H dependence of T star.pdf}
    \end{subfigure}
    \caption[$\rho_{xx}(T)$ and $T^*(H)$ for $0~\text{T}\leq\mu_0H\leq9\text{~T}$]{(a) Thermoresistivity of YbPdBi a range of applied magnetic fields $H$. Both the resistivity curve and the resistivity maximum near $T^*=34\pm4$~K are suppressed with increasing $H$~\cite{coleman_heavy_2015}. (b) Dependence of the coherence temperatures $T^*$ on an applied magnetic field $H$. 
    % Values can be found in Table~\ref{tab:T_star_values}.
    }
    \label{fig:thermo_res_in_field_zoomed}
\end{figure}

% \begin{table}[h]
% \centering
% \begin{tabular}{c|c|c|c|c}
% $H$ (T)     & 0        & 3            & 6        & 9        \\\hline
% $T^*$ (K) & 34\pm4$ & $37.5\pm2.7$ & $46\pm4$ & $58\pm6$
% \end{tabular}
% \caption[Coherence temperature $T^*$ for $H\geq0$~T]{Coherence Temperature $T^*$ for applied magnetic fields $H\geq0$~T. Found by fitting a quadratic polynomial to a subset of the data shown in Fig.~\ref{fig:thermo_res_in_field_zoomed}a.}\label{tab:T_star_values}
% \end{table}


%%%%%%%%%%%%%%%%%%%%%%%%%%%%%%
%%%%%%%%%%%%%%%%%%%%%%%%%%%%%%
%%%%%%%%%%%%%%%%%%%%%%%%%%%%%%
\subsection{WAL Effects with Large Negative Magnetoresistivity $\rho_{xx}(H)$}\label{sec:magnetoresistivity}

Magnetoresistance ($MR$) is defined as the change in a sample's resistivity caused by a magnetic field ($H$). It is typical to measure it relative to its value at zero-field ($\rho_0$), as a fraction or percentage of that zero-field value~:
\begin{equation}\label{eq:MR_definition}
   MR = \Delta\rho/\rho_0 = \frac{\rho_{xx}(H)-\rho_{xx}(0)}{\rho_{xx}(0)} 
\end{equation}

\noindent For normal metals, the $MR$ is predicted to be small and positive due to the Lorentz force acting on the conduction electrons~\cite{huang_observation_2015, pippard_magnetoresistance_1989, kittel_quantum_1963}. This is often not the case in reality, where deviations from this are indicative of interesting physics. In a range of materials, large $ MR$ has been observed, including the appropriately named giant magnetoresistance (GMR) associated with iron/chromium/iron (Fe/Cr/Fe) multilayers and colossal magnetoresistance (CMR) associated with manganese (Mn) oxides~\cite{blundell_magnetism_2001}.\\

\noindent Additionally, weak-localization (WL) and weak-antilocalization (WAL) can alter the behaviour of the $MR$~\cite{bergmann_weak_nodate}. These effects arise due to quantum interference of the paths conduction electrons take when scattered. For zero-field measurements, WL will suppress resistivity, and WAL will increase resistivity. A distinct signature for both occurs when an increasing $H$ is applied. There forms a cusp-like change in resistivity, which increases for WAL and decreases for WL before disappearing at sufficiently strong $H$.\\

\noindent Shown in Fig.~\ref{fig:MR_actrot13} is the $MR$ for a sample of YbPdBi grown using the Bi-flux method. Data was taken at $|\mu_0H| \leq 9$~T for a range of temperatures. Gold wires were spot welded to the sample such that $I$ ran parallel to the [100] crystal axis. The full range of data is shown in Fig.~\ref{fig:MR_actrot13}a where there is a large negative $MR$ for $T\leq15$~K. At $T=2$~K, $MR$ seems to exhibit saturating behaviour at large $H$.  We highlight the presence of a WAL effect in Fig.~\ref{fig:MR_actrot13}b, where we show only positive $MR$. The positive cusp-like increase for $\mu_0|H|<3$~T is due to the effects of WAL, which is more effectively suppressed by $H$ at lower $T$.\\


\begin{figure}[H]
    \centering
    \begin{subfigure}[t]{0.14\textwidth}
        \textbf{a)}
    \end{subfigure}
    \centering
    \begin{subfigure}[t]{0.85\textwidth}  
        \includegraphics[width=\linewidth, valign=t]{Figures_Ch4_Synth_Charac/Resistivity/YbPdBi, pct diff MR actrot13.pdf}
    \end{subfigure}
    \centering
    \begin{subfigure}[t]{0.14\textwidth}
        \textbf{b)}
    \end{subfigure}
     \begin{subfigure}[t]{0.85\textwidth}
        \includegraphics[width=\linewidth, valign=t]{Figures_Ch4_Synth_Charac/Resistivity/YbPdBi, pct diff MR ZOOM actrot13.pdf} 
    \end{subfigure}
    \caption[$\rho_{xx}(H)$ for $2~\text{K}\leq T \leq 150~\text{K}$]{(a) Magnetoresistance $\Delta\rho/\rho_0$ calculated according to Eqn.~\ref{eq:MR_definition} for a sample of YbPdBi grown using Bi-flux. The magnetic field $H$ is applied perpendicularly to the current $I$. (b) The same data is re-plotted for $\Delta\rho/\rho_0>0$. The dashed lines are guides for the eyes and do not signify data points.}\label{fig:MR_actrot13}
\end{figure}


%%%%%%%%%%%%%%%

\section{Hall Effect and Transverse Resistivity}\label{sec:Hall_measurementst}
% Upon review:
% \begin{enumerate}
%     \item Make sure your wording is not such that the two sections are in competition re:AHE
%     \item Consider whether the Results Comparison section is even necessary anymore. Wording fixe should make it obsolete, I imagine.
% \end{enumerate}

Shown in Fig.~\ref{fig:HallEffectAshcroft} is a diagram of the Hall Effect. An electric field $E_x$ induces a current density $j_x$ through a conducting rectangular prism, and a magnetic field $H_z$ is applied parallel to the $z$-axis. The drift velocity of the electrons runs opposite to $j_x$, causing them to experience a Lorentz force of $-e v_x H_z$ in the negative $y$-direction. As electrons build up on one side of the prism, an electric field $E_y$ forms. This is the foundation of the normal Hall effect (NHE), which is characterized by the Hall coefficient~:
\begin{equation}
    R_\mathrm{H} = \frac{E_y}{j_x H}=\frac{\rho_{xy}}{H}
\end{equation}
\noindent where $\rho_{xy}=\frac{E_y}{j_x}$ is the transverse resistivity. \\

\noindent $R_\mathrm{H}$ is measured by either a four-point contact arrangement or a five-point contact arrangement. A four-point measurement makes use of four contacts, two to induce $j_x$ and two to measure the voltage drop of $E_y$. A misalignment of the two voltage contacts results in a measured value of $R_H$ that is dominated by the much larger $\rho_{xx}(H)$, as they add in quadrature. The misalignment can be negated by the five-point setup, wherein one side gets a second voltage contact. The voltage drop between these two voltage contacts is zeroed with respect to the third voltage contact on the opposite side of the sample, offsetting the influence of $\rho_{xx}(H)$. However, the four-point measurement is still useful as it is simple to wire up. We can account for any wire misalignment by taking measurements under both positive and negative values of $H$ and anti-symmetrizing the measured values. \\


\begin{figure}
    \centering
    \includegraphics[width=\linewidth]{Figures_Ch4_Synth_Charac/Hall Data/Hall effect Diagram.pdf}
    \caption[Hall Effect Diagram]{Diagram of the Hall Effect for a rectangular prism of conducting material. A Lorentz force $-ev_xH_z$ on the conduction electrons causes a build-up of electrons along one side, thus creating an electric field $E_y$. %Figure adapted from~\cite{ashcroft_solid_1976}.
    }
    \label{fig:HallEffectAshcroft}
\end{figure}


\noindent Four-point Hall measurements were taken under an applied magnetic field $|\mu_0H|\leq 9$~T on a sample of YbPdBi grown using the Bi-flux. The sample was made rectangular via sanding with fine grit, and gold wires were spot welded onto the surface such that $I$ ran parallel to the [100] crystal axis. Anti-symmetrization of the data was necessary to eliminate the longitudinal contributions. This was performed using~:
\begin{equation}\label{eq:hall_AS}
    \rho_{xy}(H,T) = \frac{\rho^{US}_{xy}(+H,T)-\rho^{US}_{xy}(-H,T)}{2}
\end{equation}
\noindent where $\rho_{xy}^{US}$ is the raw, un-symmetrized transverse resistivity. See App.~\ref{app:hall_measurements} for $\rho^{US}_{xy}(H)$ data.\\


%%%%%%%%%%%%%%%%%%%%
%%%%%%%%%%%%%%%%%%%%
\subsection{Crossover Behaviour in the Transverse Resistivity $\rho_{xy}(H)$}\label{sec:transverse_res}

Shown in Fig.~\ref{fig:asym_rhoxy} is $\rho_{xy}(H)$ obtained using Eq~\ref{eq:hall_AS}. On the left, Fig.~\ref{fig:asym_rhoxy}a shows data taken at higher temperatures, both above and below $T^*$. The slope of each curve increases as the temperature is lowered. On the right, Fig.~\ref{fig:asym_rhoxy}b shows data taken at lower temperatures, with data for $T=120$~K included to aid in visual comparison. These lower temperatures exhibit a marked deviation from the linear behaviour present in the high-$T$ data. These low-$T$ curves of $\rho_{xy}(H)$ overlap for small $H$ until reaching a $T$-dependent crossover field ($H_{cross}$). At the crossover point, the curves bend downwards, then continue increasing linearly with a reduced slope. Within the range of $H$ available, the change in behaviour of $\rho_{xy}$ begins below $T=40$~K.\\



\begin{figure}[h]
    \centering
    \includegraphics[width=\linewidth]{Figures_Ch4_Synth_Charac/Hall Data/YbPdBi, JF042, ASym Resxy ch2.pdf}
    \caption[ Transverse Resistivity $\rho_{xy}(H)$ for $2~\text{K}\leq T \leq 120~\text{K}$]{Transverse resistivity $\rho_{xy}$ for YbPdBi for a range of temperatures, plotted separately for ease of viewing. (a) High-temperature data demonstrates only linear behaviour. (b) Low-temperature data sets that show a departure from linear behaviour. Dashed lines are shown simply to guide the eye. Data for $T=120$~K is presented in both plots to act as a reference for comparison. }
    \label{fig:asym_rhoxy}
\end{figure}


\noindent Normally, such a change of slope is observed in the presence of an anomalous Hall effect (AHE)~\cite{blundell_magnetism_2001}. An anomalous Hall resistivity $\rho^A_{xy}(H)$ contributes to the total transverse resistivity and is linearly proportional to the sample's magnetization $M$. Thus, the slope of $\rho_{xy}(H)$ changes when $M$ saturates. However, we do not observe a saturation in $M$ for $\mu_0H<9$~T and $T\geq 2$~K (see Sec.~\ref{sec:mag_sus}). Therefore, we cannot easily attribute this change in slope to an AHE. This does not, however, rule out the existence of an AHE, which we will elaborate on in Sec.~\ref{sec:hall_AHE_THE}.  \\

\noindent We also did not find that the data was well-fitted by a multi-band Drude model. Instead, we investigated the linear regions of $\rho_{xy}$ and the crossover by fitting straight lines of the form~:
\begin{equation}\label{eq:single_band_Hall}
    \rho_{xy} = m H + \rho_0
\end{equation}
\noindent where $m$ is the slope, and $\rho_0$ is the y-intercept. For the single-band Drude model, $\rho_0=0$ and $m=R_\mathrm{H}$~\cite{ashcroft_solid_1976}. For data taken at $T\leq30$~K, two ranges were fitted on either side of $H_{cross}$.\\


\begin{figure}[h]
    \centering
    \includegraphics[width=0.75\linewidth]{Figures_Ch4_Synth_Charac/Hall Data/YbPdBi, JF042, rhoxy linear fits, Hi_T.pdf}
    \caption[Linear Fits of $\rho_{xy}(H)$ for $40~\text{K}\leq T\leq 120~\text{K}$.]{High-temperature transverse resistivity ($\rho_{xy}$) of YbPdBi. Each temperature is fitted with a single-band model, $\rho_{xy}=R_\mathrm{H} H$ where $R_\mathrm{H}$ is the Hall coefficient. The data is well-fitted up to $\mu_0H=9$~T, with $R_\mathrm{H}$ increasing as the temperature is decreased.}
    \label{fig:hi_T_Hall_Fits_nice}
\end{figure}


\noindent Shown in Fig.~\ref{fig:hi_T_Hall_Fits_nice} are the fits of Eqn.~\ref{eq:single_band_Hall} for $T\in\{120~\text{K}, 60~\text{K}, 40~\text{K}\}$ which are well described up to $\mu_0H=9$~T. The slope $m$ increases with decreasing temperature. Since the intercepts are approximately zero, we can say that these fits correspond to a single-band Drude model. Shown in Fig.~\ref{fig:low_T_rhoxy_lin_fits} are the fits of Eqn.~\ref{eq:single_band_Hall} for $T\leq30$~K. As mentioned before, this low-$T$ data appears to exhibit two linear regimes separated by a crossover field $H_{cross}$. We approximated $H_{cross}$ as the value at which the linear fits intersect. The value of $H_{cross}$ grows with increasing $T$.\\

\noindent Considering that heavy fermion behaviour is believed to begin at $T^*=34\pm4$~K, one could associate this crossover behaviour for $T\leq30$~K with the heavy fermion phase. However, it could also be that $H_{cross}$ occurs beyond $\mu_0H=9$~T above $T=30$~K. Regardless of its relation to $T^*$, the change in slope appears to occur gradually, indicating a potential second-order phase transition.\\

\begin{figure}[H]
    \centering
    \begin{subfigure}{0.49\linewidth}  
    \includegraphics[width=\linewidth]{Figures_Ch4_Synth_Charac/Hall Data/YbPdBi, JF042, rhoxy linear fits lo_T 30K.pdf}
    \end{subfigure}
    \centering
     \begin{subfigure}{0.49\linewidth} 
     \includegraphics[width=\linewidth]{Figures_Ch4_Synth_Charac/Hall Data/YbPdBi, JF042, rhoxy linear fits lo_T 15K.pdf} 
    \end{subfigure}
    \centering
     \begin{subfigure}{0.49\linewidth} 
     \includegraphics[width=\linewidth]{Figures_Ch4_Synth_Charac/Hall Data/YbPdBi, JF042, rhoxy linear fits lo_T 10K.pdf}
    \end{subfigure}
    \centering
     \begin{subfigure}{0.49\linewidth} 
     \includegraphics[width=\linewidth]{Figures_Ch4_Synth_Charac/Hall Data/YbPdBi, JF042, rhoxy linear fits lo_T 5K.pdf} 
    \end{subfigure}
    \centering
     \begin{subfigure}{0.49\linewidth} 
     \includegraphics[width=\linewidth]{Figures_Ch4_Synth_Charac/Hall Data/YbPdBi, JF042, rhoxy linear fits lo_T 2K.pdf}
    \end{subfigure}
    \caption[Double Linear Fits of $\rho_{xy}(H)$ for $2~\text{K}\leq T\leq30~\text{K}$.]{Transverse resistivity ($\rho_{xy}$) of YbPdBi, exhibiting two regions of linear behaviour. These regions were each fitted with linear functions. A crossover field strength ($H_{cross}$) between the two linear regions was approximated as the intersection of each pair of linear fits and is shown as a vertical dashed purple line.}\label{fig:low_T_rhoxy_lin_fits}
\end{figure}



\noindent Shown in Fig.~\ref{fig:rhoxy_lin_fit_params}a and Fig.~\ref{fig:rhoxy_lin_fit_params}b are the slopes and intercepts of the linear fits, respectively, as a function of temperature. The error bars shown correspond to the standard deviations calculated when fitting a straight line. The zero-field coherence temperature $T^*$ is plotted as a vertical dashed black line. For $T>T^*$, $\rho_{xy}$ is well described by the single-band model up to $\mu_0H=9$~T since $\rho_0$ is approximately zero. Thus, the slope corresponds to $R_\mathrm{H}$ and increases non-linearly with decreasing temperature. This also applies below $T^*$ for the fit parameters of $H<H_{cross}$, so we may consider them governed by similar physics. Below $T=10$~K, the values of $R_\mathrm{H}$ begin to drop as the temperature is lowered. We see different behaviour for the fit parameters of the $H>H_{cross}$ data. The temperature dependence of $m$ appears to almost be an inversion that for $H<H_{cross}$, and the intercept $\rho_0$ varies strongly with temperature below $T^*$. \\


\begin{figure}[h]
    \centering
    \begin{subfigure}[t]{0.01\textwidth}
        \textbf{a)}
    \end{subfigure}
    \centering
    \begin{subfigure}[t]{0.7\linewidth}  
        \includegraphics[width=\linewidth, valign=t]{Figures_Ch4_Synth_Charac/Hall Data/YbPdBi, ACT5, JF042, Rh Fit param Plot.pdf}
    \end{subfigure}\hfill\\
    \centering
    \begin{subfigure}[t]{0.01\textwidth}
        \textbf{b)}
    \end{subfigure}
    \centering
    \begin{subfigure}[t]{0.7\linewidth} 
        \includegraphics[width=\linewidth, valign=t]{Figures_Ch4_Synth_Charac/Hall Data/YbPdBi, ACT5, JF042, b int Fit param Plot.pdf} 
    \end{subfigure}
    \caption[Parameters for Linear Fits of $\rho_{xy}(H)$]{(a) Slopes $m$ and (b) intercepts $\rho_0$ of  Eqn.~\ref{eq:single_band_Hall} fitted to the transverse resistivity $\rho_{xy}$ of YbPdBi. The coherence temperature $T^*$ is shown as a dashed black line.}\label{fig:rhoxy_lin_fit_params}
\end{figure}


\noindent Shown in Fig.~\ref{fig:rhoxy_n_and_Hcross}a are the carrier densities $n$ for the fits whose intercepts $\rho_0$ were approximately zero. These were calculated using~\cite{kittel_introduction_2005}~:
\begin{equation}\label{eq:carrier_densities}
    n = \frac{1}{q R_\mathrm{H}}
\end{equation}
\noindent where $q$ is the elementary charge for hole carriers and $R_\mathrm{H}$ is the Hall coefficient. We have assumed $R_\mathrm{H}$ to equal the slope since the corresponding intercepts are very nearly zero. The carrier density, therefore, decreases non-linearly until approximately $T=10$~K. Below this temperature, the carrier density begins to increase again. The error bars shown correspond to the standard deviations calculated by the fitting function. \\


\begin{figure}%[h]
    \centering
    \begin{subfigure}[r]{0.05\textwidth}
        \textbf{a)}
    \end{subfigure}
    \centering
     \begin{subfigure}[t]{0.7\linewidth} 
     \includegraphics[width=\linewidth, valign=t]{Figures_Ch4_Synth_Charac/Hall Data/YbPdBi, ACT5, JF042, carrier density n Fit param Plot.pdf}
    \end{subfigure}\hfill\\
    \centering
    \begin{subfigure}[t]{0.05\textwidth}
        \textbf{b)}
    \end{subfigure}
    \centering
     \begin{subfigure}[t]{0.65\linewidth} 
     \includegraphics[width=\linewidth, valign=t]{Figures_Ch4_Synth_Charac/Hall Data/YbPdBi, JF042, rhoxy Bcross.pdf}
    \end{subfigure}
    \caption[Carrier Densities and $H_{cross}$ Values from Fits of $\rho_{xy}(H)$]{Carrier densities $n$ and crossover field strengths $H_{cross}$, with the coherence temperature $T^*$ shown as a dashed black line. (a) Carrier densities calculated using Eqn.~\ref{eq:carrier_densities}. (b) Values of $H_{cross}$ fitted with Eqn.~\ref{eq:h_cross_function} which suggests the zero-field crossover occurs at $T_0=(0.9\pm0.2)$~K.}\label{fig:rhoxy_n_and_Hcross}
\end{figure}

\noindent Finally, shown in Fig.~\ref{fig:rhoxy_n_and_Hcross}b are the values of $H_{cross}$ with error bars calculated via the standard deviations of the linear fit parameters. The growth in fit uncertainties arises because, as $T$ increases, $H_{cross}$ approaches the maximum field strength available to us, and so fewer data points are available to fit. Within this range of temperatures, $H_{cross}$ exhibits strong, non-linear $T$-dependence. We can infer that there exists a temperature $T_0$ for which $\mu_0H_{cross}=0$~T. Thus, we fitted the data with the empirical function~:
\begin{equation}\label{eq:h_cross_function}
    H_{cross} = \alpha(T-T_0)^n
\end{equation}
\noindent where $T_0=(0.9\pm0.2)$~K, $n=(0.66\pm0.03)\approx 2/3$ and $\alpha=(0.78\pm0.07)$~T/K$^n$ are the fit parameters and their fitted values. \\

\noindent It appears we can associate the existence of a crossover in $\rho_{xy}(H)$ to two important temperatures. The fitted value $T_0=(0.9\pm0.2)$~K suggests that the zero-field crossover coincides with a peak in the specific heat at $T=1$~K (see Sec.~\ref{sec:cv_section}). This peak has been attributed to a magnetic phase transition~\cite{lebras_local_1995}, but we see no evidence of magnetic ordering in powder neutron diffraction experiments (see Sec.~\ref{sec:neutrons}). Further, this crossover behaviour seems to be strongly related to $T^*$ and the presence of HFQPs. Recall the fit slopes $m$ and intercepts $\rho_0$ shown in Fig.~\ref{fig:rhoxy_lin_fit_params}. The size of the differences between their values above and below $H_{cross}$ possess a temperature dependence. These differences decrease as the temperature approaches $T^*$ and $T=1$~K, being maximal at their approximate mid-point. This implies their differences are approximately zero above or below these temperatures, which would mean that this crossover behaviour only exists between these two temperatures and with the presence of HFQPs.\\



\subsubsection{Relaxation Times at $\mu_0H=0.2$~T}

Using the carrier densities $n$ found in low-$H$ fits of Eqn.~\ref{eq:single_band_Hall} and Eqn.~\ref{eq:carrier_densities}
from the previous section (shown in Fig.~\ref{fig:rhoxy_n_and_Hcross}a), carrier relaxation times were extracted using~\cite{ashcroft_solid_1976}~:
\begin{equation}\label{eq:relax_times}
    \tau = \frac{m_e}{\rho_{xx}(T)ne}
\end{equation}
where $m_e$ is the mass of an electron and $e$ is the elementary charge. $\rho_{xx}(T)$ and $n$ are the longitudinal resistivities and the carrier densities, respectively, found at $\mu_0 H=0.2$~T for each value of $T$.\\ 


\begin{figure}[h]
    \centering
    \includegraphics{Figures_Ch4_Synth_Charac/Hall Data/ACT5_relaxation_time_0.2T.pdf}
    \caption[Relaxation Times Below $H_{cross}$]{Relaxation times below $H_{cross}$. The vertical dashed black line is the coherence temperature $T^*$. The solid blue line is a simple linear fit used to approximate a $T=0$~K relaxation time, with the fit parameters $b=\tau_{0}=(2.78\pm0.02)$~x$10^{-15}$~s and $m=(-7.8\pm0.2)$~x$10^{-17}$~s/K.}
    \label{fig:relaxation_times}
\end{figure}

\noindent The values of $\tau$ for $\mu_0 H=0.2$~T are displayed in Fig.~\ref{fig:relaxation_times}. The data below $T=20$~K was fitted with a linear equation to extract an approximate relaxation time for $T=0$~K. From this we extracted a zero-temperature scattering rate of $\frac{1}{\tau_0} = (3.60\pm0.03)$~x$10^{14}$~s$^{-1}$.\\



%%%%%%%%%%%%%%%%%%%%
%%%%%%%%%%%%%%%%%%%%
\subsection{Anomalous and Topological Hall Resistivities, $\rho^{A}_{xy}$ and $\rho^{T}_{xy}$}\label{sec:hall_AHE_THE}
% \begin{enumerate}
%     \item wording: just because we couldn't attribute the change in slope doesn't mean we can't parse out an AHE.
%     \item write a caption for Fig.~\ref{fig:AHE_curve_fit_sanity_check}
% \end{enumerate}


Here we analyze $\rho_{xy}$ from an alternative perspective. While we were unable to readily ascribe the change in slope of $\rho_{xy}(H)$ to an AHE, here we attempt to parse the contributions to $\rho_{xy}$ arising from the normal Hall effect (NHE) and the Anomalous Hall effect (AHE). Following previous analyses on the isostructural YbPtBi~\cite{guo_evidence_2018} and GdPtBi~\cite{suzuki_large_2016}%li_robust_2013}
, we suggest any unaccounted-for contributions to $\rho_{xy}(H)$ are attributable to a topological Hall effect (THE) caused by non-trivial properties of YbPdBi.\\

\noindent While the NHE occurs due to the conduction electrons experiencing a Lorentz force, the AHE is associated with the magnetization $M$ induced by a magnetic field $H$. When localized moments are present (as is the case for the \textit{4f}-electrons of Yb$^{3+}$), there is a spin-orbit interaction (SOI) between the conduction electrons and these localized moments~\cite{blundell_magnetism_2001}. As the conduction electrons pass a localized moment with spin $S$, they possess an orbital angular momentum $L$ relative to the localized moment. Whether they pass on one side or the other of the moment determines the sign of $L$. The magnitude of the SOI is proportional to $L\cdot S$, so there is asymmetric scattering due to paths along one side of the moment being less energetically favourable than those on the other side. \\

\noindent This ultimately manifests in $\rho_{xy}(H)$ as an AHE term linearly proportional to $M$~\cite{blundell_magnetism_2001, nagaosa_anomalous_2010}~:
\begin{equation}\label{eq:AHE_simple_rhoxy}
    \rho_{xy} = R_\mathrm{H} H + R_A M
\end{equation}
\noindent where $R_\mathrm{H}$ is the normal Hall coefficient, and $R_A$ is the anomalous Hall coefficient. A clear indicator of the AHE in  $\rho_{xy}(H)$ is when the saturization magnetization $M_s$ is reached, causing the contribution from the AHE term to also saturate. This will result in a change in the slope of $\rho_{xy}(H)$ comparable to what was found in the previous section. However, we do not find that $M$ saturates for $\mu_0H\leq9$~T at these temperatures (see Sec.~\ref{sec:mag_sus}), which motivates the analysis presented here.\\

\noindent When the NHE and the AHE contributions to $\rho_{xy}$ can be separately accounted for in a possibly topologically non-trivial material, any unaccounted-for contribution may be due to the presence of a topological Hall Effect (THE). in this case, $\rho_{xy}$ may be separated into three contributions for magnetic systems~\cite{guo_evidence_2018,suzuki_large_2016,li_robust_2013}~:
\begin{equation}\label{eq:rhoxy_terms}
    \rho_{xy} = \rho^N_{xy} +\rho^A_{xy} +\rho^T_{xy}
\end{equation}
\noindent where $\rho^N_{xy}=R_\mathrm{H} H$ is the NHE contribution, $\rho^A_{xy}=R_A M$ is the AHE contribution, and $\rho^T_{xy}$ is the THE contribution. Typically, $R_\mathrm{H}$ lacks a strong $T$-dependence while $R_A$ has a strong $T$-dependence~\cite{blundell_magnetism_2001}.\\

\noindent These terms must be parsed from $\rho_{xy}$ to extract a THE term, and we must make a series of approximations to do so. Firstly, we know that $\rho_{xy}^N$ is linear in $H$. Based on the $\rho_{xy}$ curves shown in Fig.~\ref{fig:asym_rhoxy} and observation of comparatively small $M$ at higher $T$ (see Sec.~\ref{sec:mag_sus}), we will approximate the NHE contribution by~\cite{suzuki_large_2016}~:
\begin{equation}\label{eq:normal_rhoxy}
    \rho_{xy}^N \approx \rho_{xy}(T=120~\text{K})
\end{equation}
\noindent It should be noted that this assumption is contrary to certain assumptions made in Sec.~\ref{sec:transverse_res}. We emphasize that this is intended to follow an alternative method of analysis that has been used previously to find a THE contribution. \\

\noindent At $T=120$~K, the system is far from any magnetic ordering and $M$ being comparatively small means any AHE contribution will be small compared to that of the NHE contribution. Shown in Fig.~\ref{fig:del_rhoxy} are the results of taking the difference~:
\begin{equation}
    \rho^A_{xy}+\rho^T_{xy} = \rho_{xy}(H,T) - \rho_{xy}(H,T=120~\text{K})
\end{equation}
\noindent which we assume is the sum of the AHE contribution and the THE contribution.\\
 
\begin{figure}[h]
    \centering
    \includegraphics[width=0.8\linewidth]{Figures_Ch4_Synth_Charac/Hall Data/YbPdBi, hall delta rhoxy ch2.pdf} % =\rho_{xy}-\rho_{xy}^N
    \caption[Anomalous and Topological Hall Resistivities, $\rho^A_{xy} + \rho^T_{xy}$]{Transverse resistivities $\rho_{xy}$ at low-temperature, which is assumed to be the sum of an anomalous Hall effect contribution $\rho^A_{xy}$ and a possible topological Hall effect contribution $\rho^T_{xy}$. This was obtained by assuming the contribution due to the normal Hall effect is well approximated by $\rho^N_{xy}=\rho_{xy}(T=120~\text{K})$, such that $\rho^A_{xy} + \rho^T_{xy} =\rho_{xy}-\rho_{xy}^N$.}
    \label{fig:del_rhoxy}
\end{figure}



\noindent Next, we estimate $\rho^A_{xy}$ to check for any leftover contribution. As stated previously, $\rho^A_{xy}$ typically takes a form that is linear in $M$: $\rho_H^{AHE} = R_A M$. However, we possess the ability to impose an additional restriction by making use of our longitudinal magnetoresistivity $\rho_{xx}(H)$ data and our $M(H)$ data extrapolated to $\mu_0H=9$~T (see App.~\ref{app:demag} for extrapolation procedure). This allows us to introduce the following scaling relation for the AHE~\cite{guo_evidence_2018,li_robust_2013,tian_proper_2009}~:
\begin{align}
    \rho_{xy}^A &= \alpha M \rho_{xx0} + \beta M \rho_{xx0}^2+bM\rho_{xx}^2 \\
                &= (\alpha  \rho_{xx0} + \beta \rho_{xx0}^2 +b\rho_{xx}^2)M \\
\xrightarrow[]{}\rho_{xy}^A(H) &= \left(\gamma  + b\rho_{xx}(H)^2\right)M(H) \label{eq:AHE_scaling_relation}
\end{align}
\noindent where $M(H)$ are given in Sec.~\ref{sec:mag_sus}, $\rho_{xx0}$ is the residual longitudinal resistivity in zero-field, and $\rho_{xx}(H)$ are the longitudinal magnetoresistivities measured using a different sample, per Sec.~\ref{sec:resistivity}. The terms with coefficients $\alpha$, $\beta$, and $b$ correspond to skew-scattering, the side jump, and the intrinsic contribution due to the presence of non-zero Berry curvature, respectively. The first two terms are each linear in $M(H)$, so we assume the coefficients $\alpha$ and $\beta$ are $H$-independent and roll them into $\gamma$ to reduce the degrees of freedom of the fit.\\

\noindent Since each $\rho_{xy}^A(H)$, $M(H)$, and $\rho_{xx}(H)$ are all functions of $H$, we perform a best fit of Eqn.~\ref{eq:AHE_scaling_relation} on the low-$T$ results of $\rho^A_{xy} + \rho^T_{xy}$. The results of this best fit are shown in Fig.~\ref{fig:AHE_curve_fits}. See App.~\ref{app:AHE_fit_params} for plots of the fit parameters. \\

\noindent For $T=15$~K and $T=10$~K, the data is well described by Eqn.~\ref{eq:AHE_scaling_relation} suggesting that the AHE is the dominant term. There is a small oscillation in both sets of fit residuals, which are also present in the fit residuals for the curves at $T=5$~K and $T=2$~K. The fits of data at the two lower temperatures are visibly worse, showing poor overlap with the data and significant oscillations in their residuals.\\

\begin{figure}[H]
    \centering
    \begin{subfigure}{0.47\linewidth} 
    \includegraphics[width=\linewidth]{Figures_Ch4_Synth_Charac/Hall Data/YbPdBi, act5 AHE curve fit 15K.pdf}
    \end{subfigure}
    \centering
     \begin{subfigure}{0.47\linewidth}
     \includegraphics[width=\linewidth]{Figures_Ch4_Synth_Charac/Hall Data/YbPdBi, act5 AHE curve fit 10K.pdf}
    \end{subfigure}
    \centering
     \begin{subfigure}{0.47\linewidth}
     \includegraphics[width=\linewidth]{Figures_Ch4_Synth_Charac/Hall Data/YbPdBi, act5 AHE curve fit 5K.pdf}
    \end{subfigure}
    \centering
     \begin{subfigure}{0.47\linewidth}
     \includegraphics[width=\linewidth]{Figures_Ch4_Synth_Charac/Hall Data/YbPdBi, act5 AHE curve fit 2K.pdf}
    \end{subfigure}
    \caption[Extracting $\rho^A_{xy}$ Using by Scaling Magnetization]{Transverse resistivity data $\rho^A_{xy}+\rho^T_{xy}$ with normal Hall effect contribution removed. The data were fitted using the magnetization scaling relation given by Eqn.~\ref{eq:AHE_scaling_relation}. The curves for $T=15$~K and $T=10$~K are well described by the curve fit, suggesting $\rho^A_{xy}>>\rho^T_{xy}$ at these temperatures, but possess an oscillation in the residuals. The curves for $T=5$~K and $T=2$~K are not as well described. The amplitude of the oscillation in the residuals has grown.}\label{fig:AHE_curve_fits}
\end{figure}

% \noindent To check the fit accuracy, we may check the linearity of $\rho^A_{xy}(H)$ with respect to $M$ at high-$H$~\cite{guo_evidence_2018}. the magnetization $M$ should be a dominant feature at high-$H$, and so too $\rho^A_{xy}$, meaning the fitted. Based on the simpler definition of the AHE in Eqn.~\ref{eq:AHE_simple_rhoxy}, we expect $\rho^A_{xy}$ to be linear in $M$. Shown in Fig.~\ref{fig:AHE_curve_fit_sanity_check} are plots of the values of $\rho^A_{xy}$ obtained by best-fit at $\mu_0H=6$~T, $\mu_0H=7$~T, $\mu_0H=8$~T, and $\mu_0H=8.8$~T, with simple linear fits. Each of the fits describes these $\rho^A_{xy}$ to mixed results, and the negative slope suggests that the anomalous Hall coefficient $R_A$ is negative at these field strengths.
% \begin{figure}[h]
%     \centering
%     \begin{subfigure}{0.47\linewidth} \includegraphics[width=\linewidth]{Figures_Ch4_Synth_Charac/Hall Data/YbPdBi, act5, AHE 6T linearity.pdf}
%     \end{subfigure}
%     \centering
%      \begin{subfigure}{0.47\linewidth}
%      \includegraphics[width=\linewidth]{Figures_Ch4_Synth_Charac/Hall Data/YbPdBi, act5, AHE 7T linearity.pdf}
%     \end{subfigure}
%     \centering
%      \begin{subfigure}{0.47\linewidth}\includegraphics[width=\linewidth]{Figures_Ch4_Synth_Charac/Hall Data/YbPdBi, act5, AHE 8T linearity.pdf}
%     \end{subfigure}
%     \centering
%      \begin{subfigure}{0.47\linewidth}\includegraphics[width=\linewidth]{Figures_Ch4_Synth_Charac/Hall Data/YbPdBi, act5, AHE 8p8T linearity.pdf}
%     \end{subfigure}
%     \caption[Magnetization Scaling Linearity at High-$H$]{\textbf{write this caption}}\label{fig:AHE_curve_fit_sanity_check}
% \end{figure}



\noindent Finally, we move on to the THE contribution $\rho_{xy}^T$, which we determine by taking the negative of the best-fit residuals of $\rho^A_{xy}$. Shown in Fig.~\ref{fig:rhoxy_THE_graphs} are these values for $\rho_{xy}^T$. We chose to fit these data to confirm they exhibit a shared $H$ and $T$ dependence. They are well-described by an equation of the form~:
\begin{equation}\label{eq:the_osc_fun}
    \rho^T_{xy} = a\sin\left(b H^n\right)
\end{equation}
\noindent where $0<a$, $0<b$, and $0<n<1$ are fit parameters.  Residuals are plotted above each fit. It should be noted that this equation is not motivated by pre-existing theory and was chosen simply because it appeared to describe the data well. One data point was omitted as an outlier, having a value of $\rho_{xy}(H=8.4\text{~T})=-1.32$~$\mu$ohm-cm at for $T=2$~K. Shown in Table~\ref{tab:THE_fit_params} are the parameters of these fits. The data appears reasonably well described by Eqn.~\ref{eq:the_osc_fun}. Though the residuals are large relative to the amplitudes, they appear to be fairly randomly distributed about zero. For $T=5$~K, the residuals tend to be randomly distributed about a negative offset. Residuals at low-$H$ tend to be negative for all $T$. \\


\begin{table}[h]
    \centering
    \begin{tabular}{c|c|c|c}
    T (K) & n             & a ($\mu$ohm-cm) & b (T$^{-1}$)          \\ \hline
    15    & 0.9$\pm$0.1   & 0.24$\pm$0.04   & 0.9$\pm$0.2           \\ \hline
    10    & 0.93$\pm$0.07 & 0.45$\pm$0.05   & 1.0$\pm$0.1           \\ \hline
    5     & 0.73$\pm$0.05 & 0.66$\pm$0.06   & 1.5$\pm$0.1           \\ \hline
    2     & 0.70$\pm$0.03 & 0.55$\pm$0.03   & 1.8$\pm$0.1           \\ \hline
    % 15    & 0.9           & 0.24$\pm$0.04   & 0.98$\pm$0.04         \\ \hline
    % 10    & 0.9           & 0.44$\pm$0.04   & 1.03$\pm$0.02         \\ \hline
    % 5     & 0.9           & 0.60$\pm$0.07   & 1.09$\pm$0.03         \\ \hline
    % 2     & 0.9           & 0.53$\pm$0.04   & 1.29$\pm$0.02             
    \end{tabular}
    \caption[$\rho^{T}_{xy}(H)$ Oscillation Fit Parameters]{Fit parameters of Eqn.~\ref{eq:the_osc_fun} to $\rho^T_{xy}$ data. }
    \label{tab:THE_fit_params}
\end{table}

\noindent Finally, recall the fits for the anomalous Hall resistivity $\rho^A_{xy}$, where the fits at $T=15$~K and $T=10$~K were much more representative of the data than the fits at $T=5$~K and $T=2$~K. Despite stemming from fits that qualitatively differed greatly, our $\rho^T_{xy}$ can be described by the same function at each $T$, and are therefore not at random. This suggests that our $\rho^T_{xy}$ represents a genuine contribution to $\rho_{xy}$. Further discussion and analyses relating to the topology of YbPdBi may be found in Ch.~\ref{ch:topology}.\\


\begin{figure}[H]
    \centering
    \begin{subfigure}{0.47\linewidth} \includegraphics[width=\linewidth]{Figures_Ch4_Synth_Charac/Hall Data/YbPdBi, act5 THE osc curve fit 15K n0p93.pdf}
    \end{subfigure}
    \centering
     \begin{subfigure}{0.47\linewidth}
     \includegraphics[width=\linewidth]{Figures_Ch4_Synth_Charac/Hall Data/YbPdBi, act5 THE osc curve fit 10K n0p93.pdf}
    \end{subfigure}
    \centering
     \begin{subfigure}{0.47\linewidth}\includegraphics[width=\linewidth]{Figures_Ch4_Synth_Charac/Hall Data/YbPdBi, act5 THE osc curve fit 5K n0p73.pdf}
    \end{subfigure}
    \centering
     \begin{subfigure}{0.47\linewidth}\includegraphics[width=\linewidth]{Figures_Ch4_Synth_Charac/Hall Data/YbPdBi, act5 THE osc curve fit 2K n0p7.pdf}
    \end{subfigure}
    \caption[Topological Hall Resistivity, $\rho_{xy}^T$]{Possible topological Hall term values calculated as $\rho_{xy}^{T}=\rho_{xy}-\rho_{xy}^N-\rho_{xy}^{A}$, where $\rho_{xy}^N$ was approximated as $\rho_{xy}(T=120\text{~K})$ and $\rho_{xy}^A$ was estimated by via best fit of Eqn.~\ref{eq:AHE_scaling_relation}. For all temperatures, $\rho_{xy}^T$ oscillates about zero with an amplitude that increases with lower $T$ and a wavelength that might decrease with lower $T$. However, the curves are quite jittery, so these features are somewhat obscured.}
    \label{fig:rhoxy_THE_graphs}
\end{figure}

% You have to say somewhere which M(H) you used.


% \noindent One thing of note is that the $\rho^T_{xy}$ could also be fitted with $n=0.9$. Of course, the fits with fixed $n$ were not as accurate, but they were qualitatively similar to the fits where $n$ was allowed to change. If fixing $n=0.9$  accurately models the data, this would suggest that the wavelength of these oscillations would be proportional to $H^{-0.1}$ for each $T$. A shared exponential $H$-dependence across these $T$ would support the assertion that $\rho^T_{xy}$ represents a real phenomenon, regardless of the value that $n$ is fixed to. However, it could also be an artifact of the noisyness of the data permitting a wide range of acceptable $n$. Higher-quality data would eliminate this possibility in the future. 

% \begin{figure}[h]
%     \centering
%     \includegraphics[width=\linewidth]{Figures_Ch4_Synth_Charac/Hall Data/YbPdBi, act5 topological hall term.pdf}
%     \caption[Possible Topological Hall Effect Term $\rho_{xy}^T$]{Possible Topological Hall term values calculated as $\rho_{xy}^{T}=\rho_{xy}-\rho_{xy}^N-\rho_{xy}^{A}$, where $\rho_{xy}^N$ was approximated as $\rho_{xy}(T=120\text{~K})$ and $\rho_{xy}^A$ was estimated by via best fit of Eqn.~\ref{eq:AHE_scaling_relation}. For all temperatures, $\rho_{xy}^T$ oscillates about zero with an amplitude that increases with lower $T$ and a wavelength that might decrease with lower $T$. However, the curves are quite jittery, so these features are somewhat obscured.}
%     \label{fig:rhoxy_THE_graphs}
% \end{figure}

%%%%%%%%%%%%%%%%%%%%
%%%%%%%%%%%%%%%%%%%%
% \subsection{Results Comparison}
% \begin{enumerate}
%     \item lack of sign of $H_{cross}$ in specific heat or elsewhere is reasonable, since dominated by different phenomena (mass vs lowest resistivity)
% \end{enumerate}

% At the heart of it, these two analyses of $\rho_{xy}$ are contradictory in their assumptions: the first analysis notes that $\rho_{xy}$ exhibits two linear regimes and assumes they are each fully described by a temperature-dependent normal Hall coefficient ($R_\mathrm{H}$); the second analysis assumes that the normal Hall contribution is essentially both temperature- and field-independent and that it is well approximated by its value at $T=120$~K. If these are to be reconciled, let us first note the limitations of the first analysis. \\

% \noindent In essence, the first analysis is a description of two ends of the data, one at low-$H$ and one at high-$H$. It does not pay much attention to what happens between those regimes, aside from asserting the existence of a crossover field strength $H_{cross}$. Further, it is a rather ad-hoc and empirical analysis, primarily motivated by simple observations of patterns in the $\rho_{xy}$ data. The coincidence of $\mu_0H_{cross}0$~T at the purported magnetic phase transition at $T=1$~K provides some reassurance that the analysis is descriptive of something real. We re-iterate that the magnetization does not saturate below $\mu_0H=9$~T for temperatures $T\geq2$~K, so we cannot attribute the crossover just to the saturation of an anomalous Hall contribution. It may be that though the linear fits are accurate descriptions of $\rho_{xy}$, we misattribute their slopes as being the normal Hall coefficient.

% The bandstructure of this material is rather complicated. A single band picture is most certainly wrong.

% In this case, the analysis is simply describing the macroscopic changes in the data, and the conclusions regarding carrier densities and such are invalid. It would be worthwhile to perform heat capacity measurements as a function of $H$ while the temperature is held constant. In doing so, we might be able to observe features in the specific heat data that correlate approximately to $H_{cross}$.\\

% \noindent The second analysis mimics previously published works on the half-Heuslers YbPtBi~\cite{guo_evidence_2018} and GdPtBi~\cite{suzuki_large_2016}, and on MnSi thin films~\cite{li_robust_2013}. These had the stated purpose of extracting a topological Hall contribution to $\rho_{xy}$ on what compound. Within the second analysis, we take into consideration both magnetization data and longitudinal resistivity ($\rho_{xx}$) data, so it is more inclusive of the underlying physics. We could say that the second analysis is more interested in how $\rho_{xy}$ evolves with $H$ with a consideration towards the microscopic.


%%%%%%%%%%%%%%%%%%
%%%%%%%%%%%%%%%%%%
%%%%%%%%%%%%%%%%%%
% \newpage
\section{Specific Heat }\label{sec:cv_section}

Specific heat measurements at constant pressure ($C_p$) were taken, where a known quantity of thermal energy was added to the sample. The change in temperature of the sample due to this thermal energy was measured both during and after the heating period. This provides $C_p$ via~:
\begin{equation}
    C_p = \left(\frac{dQ}{dT}\right)_p
\end{equation}
\noindent where $Q$ is the thermal energy and $T$ is temperature. We approximate $C_p$ to be equal to the specific heat at constant volume $C_v$ since the volume of the sample does not change significantly. Using this, we can garner information about the lattice of the material, as well as information about its electronic and magnetic properties. At phase transitions, there is typically an increase in the specific heat due to a change of entropy and/or to the latent heat of transition, which allows one to identify temperatures where interesting physics is occurring. \\

\begin{figure}
    \centering
    \includegraphics[width=\linewidth]{Figures_Ch4_Synth_Charac/Specific Heat/SpecificHeatApparatus.pdf}
    \caption[Specific Heat Measurement Diagram]{Experimental setup used to measure the specific heat of a sample. The puck frame acts as a thermal bath. %Figure adapted from~\cite{noauthor_physical_2017}.
    }
    \label{fig:qdUSA_heat_cap}
\end{figure}

\noindent Shown in Fig.~\ref{fig:qdUSA_heat_cap} is the experimental setup used by the QDUSA Physical Property Measurement System (PPMS) for specific heat measurements. A sample is placed under a vacuum after it is affixed to a platform by a small quantity of Apiezon grease. The heater adds $Q$ for a period of time during which the thermometer measures the change in temperature of the platform, the grease, and the sample. Once the heater is stopped, the temperature change is again measured as everything cools. To obtain the heat capacity of just the sample, one must have previously measured the heat capacities of the grease and the platform without the sample present, which is subtracted from the final measurements.\\


%%%%%%%%%%%%%%%%%%
%%%%%%%%%%%%%%%%%%
\subsection{Magnetic Specific Heat $C_{\mathrm{mag}}(T)$}\label{sec:YbPdBi_spec_heat}

Shown in Fig.~\ref{fig:sample_specific_heat} is the molar specific heat from $T=4$~K to $T=300$~K of a sample of YbPdBi grown using Pb-flux. The inset shows $\frac{C_p}{T}$ as a function of $T^2$ for $T<40$~K. A sharp peak is implied to occur in $\frac{C_p}{T}$ below our minimum temperature. This aligns with specific heat data taken below $T=1$~K in previous publications~\cite{lebras_local_1995, pietri_magnetoresistance_2000, kaczorowski_magnetic_1999, dhar_heavy-fermion_1988}, where it has been attributed to an anti-ferromagnetic (AFM) phase transition. However, during powder neutron diffraction experiments down to $T=0.1$~K, we found no clear evidence of a classic magnetic ordering occurring (see Sec.~\ref{sec:neutrons}).\\

\begin{figure}[h]
    \centering
    \includegraphics{Figures_Ch4_Synth_Charac/Specific Heat/YbPdBi, JF011, Raw Specific Heat.pdf}
    \caption[Specific Heat $C_p(T)$ for $4~\text{K}\leq T\leq300~\text{K}$]{Molar specific heat $C_p$ of YbPdBi from $T=4$~K to $T=300$~K. Samples were grown in a Pb-flux. (Inset) $\frac{C_p}{T}$ as a function of $T^2$ shown for $T<40$~K. The sharp increase in $\frac{C_p}{T}$ as $T\xrightarrow[]{}1$~K is suggestive of a phase transition, though we saw no clear evidence of magnetic ordering in neutron diffraction data.}
    \label{fig:sample_specific_heat}
\end{figure}



\noindent To establish the heavy fermion nature of a material (HFM), the low-temperature specific heat is fitted with~\cite{coleman_heavy_2015, kittel_introduction_2005}~:
\begin{equation}\label{eq:low_T_linear_Cv}
    C_p = \gamma T + \beta T^2
\end{equation}
\noindent where $\gamma$ is the Sommerfeld coefficient characterizing the electronic contribution, and $\beta$ the phonon contribution. For HFM's, $\gamma$ is used to estimate the effective mass of the conduction electrons. It is proportional to the effective mass $m^*$ via the Fermi energy~\cite{kittel_introduction_2005}
\begin{equation}\label{eq:gamma_to_eff_mass}
    \gamma \propto g(E_\mathrm{F}) \propto \frac{1}{E_\mathrm{F}} \propto m^*
\end{equation}
\noindent where $g\left(E_\mathrm{F}\right)$ is the density of states at the Fermi energy. The value for copper is $\gamma_{\mathrm{Cu}} =0.67$~mJ/mol-K$^2$~\cite{ashcroft_solid_1976} and is often used as the reference for a normal metal, whereas for a HFM $\gamma\geq 100 $~mJ/mol-K$^2$~\cite{kondo_resistance_2006}. \\



\noindent  As can be seen from the inset of Fig.~\ref{fig:sample_specific_heat}, the low-temperature specific heat data does not present itself as a straight line. Thus, it is difficult to adequately characterize the $\gamma$ and $\beta$ values using our data. A linear fit of Eqn.~\ref{eq:low_T_linear_Cv} to our low-temperature $C_p$ data is highly dependent on the range of temperatures chosen to be fitted. Previously reported $\gamma$ values were $470$~mJ/mol-K$^2$~\cite{dhar_heavy-fermion_1988}, %(which also did not fully capture the $T=1$~K peak)
$1200$~mJ/mol-K$^2$~\cite{pietri_magnetoresistance_2000}, % (which did capture the peak)'
and $240$~mJ/mol-K$^2$\cite{robinson_neutron_2000}. Though the $T=1$~K peak was fully captured by LeBras \emph{et al}.~\cite{lebras_local_1995}, both a Schottky-like anomaly and the $T=1$~K phase transition interfered with extracting a value for $\gamma$, so they did not report a fitted $\gamma$ value for YbPdBi. \\

\noindent To augment our investigation, specific heat data below $T=3$~K was digitized from LeBras \emph{et al}.~\cite{lebras_local_1995}. These data points will be distinguished from our data in graphs by using a different marker style. See App.~\ref{app:phonon_Cv_calc} for a comparison of the digitized data against our own. Further, to approximate the phonon contribution ($C_{\mathrm{ph}}$) to the specific heat, we adapted the specific heat of YPdBi samples that were grown using the Bi-flux method (see Sec.~\ref{sec:growth}). Details of this approximation process are presented in App.~\ref{app:phonon_Cv_calc}.\\




\noindent Shown in green in Fig.~\ref{fig:magnetic_Cv} is the magnetic contribution to the specific heat $C_{\mathrm{mag}}$. The phonon contribution estimated from scaled YPdBi data is shown in black. The vertical black line represents the heavy fermion coherence temperature $T^*$ for YbPdBi. In Fig.~\ref{fig:magnetic_Cv}a the data is plotted as $C_p(T)$, and is re-plotted as $\frac{C_p(T^2)}{T}$ in Fig.~\ref{fig:magnetic_Cv}b. There are two stand-out features in the $C_{\mathrm{mag}}$ data: the first is the peak near $T=1$~K, and the second is the broad bump that forms below $T^*$. The plateau near $C_p=2$~(J/mol-K) aligns with the magnetic specific heat given by LeBras \emph{et al}., which makes sense since we are adapting their data, but our maxima are about $1$~K/mol-K less than what they observed. Though the maxima occur at approximately the same temperature, our $C_{\mathrm{mag}}$ falls off faster with increasing $T$, implying the calculations of the phonon contributions differ. Since the mass of LuPdBi is closer to that of YbPdBi, repeating the analysis shown in App.~\ref{app:phonon_Cv_calc} using the specific heat of LuPdBi rather than YPdBi would minimize the amount of shifting required, providing a closer approximation to the true phonon contribution of YbPdBi.\\

\noindent Below the value of $T^*$, as determined from the resistivity measurements, the heavy fermion phase is expected to develop. Due to the magnetic origin of the heavy fermion state, the increase in $C_{\mathrm{mag}}$ below $T^*$ could be at least partially attributable to the heavy fermion phase forming. The increase was previously attributed to a Schottky anomaly by LeBras \emph{et al}.~\cite{lebras_local_1995}. We investigate this feature and its $H$-dependence using the language of crystalline electric fields in Sec.~\ref{sec:Schottky_CEF_fitting}. After reaching a maximum near $T=10$~K, the magnetic specific heat begins to drop with temperature, plateaus just above $T=2$~K, until there is a $\lambda$-type peak near $T=1$~K. As we have stated, we found no evidence of magnetic ordering in neutron scattering diffraction data taken down to $T=0.1$~K. \\

\noindent In Fig.~\ref{fig:magnetic_Cv}b, the peak at $T=1$~K is the dominant feature, below which $C_{\mathrm{mag}}$ decreases linearly with respect to $\ln T^2$. This linear decrease is also present when $\frac{C_{\mathrm{mag}}}{T}$ is plotted against $T$ with linear scaling. Fitting $\frac{C_{\mathrm{mag}}}{T}$ with Eqn.~\ref{eq:low_T_linear_Cv} for $T<=1$ yields a zero-temperature value of $\frac{C_{\mathrm{mag}}}{T}=(0.8\pm0.1)$~J/mol-K with a slope of $(2.3\pm0.2)$~J/mol-K$^2$. While we can tentatively claim this intercept to be the Sommerfeld coefficient $\gamma$, the use of data points corresponding to the transition peak undermines this claim. Nonetheless, the value of this intercept is important for calculating the magnetic entropy.\\


\begin{figure}[h!]
    \centering
    \begin{subfigure}[t]{0.01\textwidth}
        \textbf{a)}
    \end{subfigure}    
    \centering
    \begin{subfigure}[t]{0.65\textwidth}
    \includegraphics[width=\linewidth, valign=t]{Figures_Ch4_Synth_Charac/Specific Heat/YbPdBi, magnetic specific heat ONLY.pdf} 
    \end{subfigure}\hfill\\
    \centering
    \begin{subfigure}[t]{0.01\textwidth}
        \textbf{b)}
    \end{subfigure}    
    \centering
    \begin{subfigure}[t]{0.7\textwidth}
    \includegraphics[width=\linewidth, valign=t]{Figures_Ch4_Synth_Charac/Specific Heat/YbPdBi, JF011, C over T specific heat magnetic ONLY.pdf} 
    \end{subfigure}
    \caption[Magnetic Specific Heat $C_{\mathrm{mag}}(T)$ for $0.5\text{~K}\leq T\leq 300\text{~K}$]{(Green) The magnetic $C_{\mathrm{mag}}$ and (black) phonon $C_{\mathrm{ph}}$ molar specific heat contributions of YbPdBi on semilogarithmic axes.  Plotted as a function of (a) $T$ and (b) $T^2$. The vertical dashed black line represents the heavy fermion coherence temperature $T^*$ for YbPdBi. Cross markers indicate digitized data from LeBras \emph{et al}.~\cite{lebras_local_1995}.}\label{fig:magnetic_Cv}
\end{figure}


% \begin{figure}[h]
%     \centering
%     \includegraphics[width=\linewidth]{Figures_Ch4_Synth_Charac/Specific Heat/YbPdBi, JF011, C over T specific heat.pdf}
%     \caption[ $\frac{C_p}{T}\left(T^2\right)$,  $\frac{C_{\mathrm{ph}}}{T}\left(T^2\right)$, and  $\frac{C_{\mathrm{mag}}}{T}\left(T^2\right)$ from $T=0.5$~k to $T=300$~K]{$\frac{C\left(T^2\right)}{T}$ for YbPdBi shown with logarithmic x-axis. (Crosses) Data or calculations based on data taken from~\cite{lebras_local_1995}; (Circles) our data. (Blue) Total specific heat data; (Yellow) Phonon specific heat determined from YPdBi measurements; (Green) Non-phonon contribution to the specific heat. (Vertical Black Line) $T^*=35$~K as determined from thermoresistivity measurements.}
%     \label{fig:C_over_T_Data}
% \end{figure}



%%%%%%%%%%%%%%%%%%%%%%%%%%%%%%%
%%%%%%%%%%%%%%%%%%%%%%%%%%%%%%%
%%%%%%%%%%%%%%%%%%%%%%%%%%%%%%%
\subsection{Magnetic Entropy $S_{\mathrm{mag}}(T)$}\label{sec:magnetic_entropy}

Shown in Fig.~\ref{fig:Specific_heat_entropy} is the magnetic entropy $S_{\mathrm{mag}}$ calculated by numerically integrating~\cite{schroeder_introduction_2000}~:
\begin{equation}\label{eq:Cvmag_entropy}
    S_{\mathrm{mag}}(T) = \int_{0}^T\frac{C_{\mathrm{mag}}}{T} dT
\end{equation}

\noindent The values obtained by this integration are heavily reliant upon the specific heat data we are using from LeBras \emph{et al}.~\cite{lebras_local_1995}. We find that within our range of temperatures, $S_{\mathrm{mag}}$ saturates or plateaus near $R\ln 6$, shown in Fig.~\ref{fig:Specific_heat_entropy} as a dashed red line. Such a saturation entropy would imply the crystalline electric field (CEF) states have degeneracies that sum to six, corresponding to a ground state multiplet of $J=|L-S|=5/2$. However, for Yb$^{3+}$, $J=L+S=7/2$, which suggests that a CEF state lays above the measured range of temperatures.\\

\noindent The inset of Fig.~\ref{fig:Specific_heat_entropy} shows that at $T=1.3$~K there is a magnetic entropy gain of approximately $0.45R\ln 2$. This is less than the $0.40\ln 2$ that was reported at this temperature by LeBras \emph{et al}.~\cite{lebras_local_1995}. This entropy is also less than the $R\ln 2$ of a doublet state, and so the peak in specific heat at $T=1$~K is not easily attributable to a CEF state. \\


\noindent Shown in Fig.~\ref{fig:entropy_log} is $S_{\mathrm{mag}}$ re-plotted semilogarithmically. $S_{\mathrm{mag}}$ increases sharply as $T\xrightarrow[]{}1$~K,  after which it appears to be linearized by the semilogarithmic plot. Three changes in slope are visible and occur near entropies that correspond to successive Kramers doublets (given by cyan, black, and red dashed lines).% Such a CEF degeneracy scheme are comparable to the results found in LeBras \emph{et al}.~\cite{lebras_local_1995}. 
We will investigate this further by studying the Schottky-like anomaly.\\

\begin{figure}[h]
    \centering
    \includegraphics[width=\textwidth]{Figures_Ch4_Synth_Charac/Specific Heat/YbPdBi, JF011, Magnetic Entropy.pdf}
    \caption[Magnetic Entropy $S_{\mathrm{mag}}(T)$]{Magnetic entropy $S_{\mathrm{mag}}$ calculated using Eqn.~\ref{eq:Cvmag_entropy}, with $R\ln6$ is shown as a dashed red line. (Inset) Low temperature $S_{\mathrm{mag}}$ near the $T=1$~K peak in specific heat.
    % If we take that the anomaly ends at approximately $T=1.3$~K (vertical black line), then the magnetic entropy associated with the peak is approximately $0.45R\ln2$ (dashed black line). 
    Cross markers indicate data digitized from LeBras \emph{et al}.~\cite{lebras_local_1995}.}
    \label{fig:Specific_heat_entropy}
\end{figure}

% \noindent As was stated, $S_{\mathrm{mag}}$ appears to saturate at $R\ln6$ rather than the $R\ln8$ value expected for the $J=7/2$ multiplet of Yb$^{3+}$. The diff63erence is exacerbated when we note that the entropy of the $T=1.3$~K transition contributes approximately $0.45R\ln2$ to $S_{\mathrm{mag}}$. Were it not for the contribution to the entropy by the $T=1$~K transition, we could restrict the CEF degeneracies such that they summed to six. This is because, for an $n$-level system, the saturation entropy is a function solely of the level degeneracies (see Eqn.~\ref{eq:n_lvl_thermodynamics}). The fact that $S_{\mathrm{mag}}$ saturates at $R\ln6$ suggests that a Kramers doublet lays above $T=300$~K. % or the CEF states develop from a ground state multiplet of $J=5/2$. Another possibility is that, for some reason, one of the Kramers doublet states is not participating in the thermal excitations involved. We could also be overestimating the phonon contribution to $C_p$, so we recommend repeating the analysis using LaPdBi or LuPdBi.\\


\begin{figure}[h]
        \centering
      \includegraphics[width=\textwidth]{Figures_Ch4_Synth_Charac/Specific Heat/YbPdBi, magnetic entropy log x.pdf}
    \caption[$S_{\mathrm{mag}}(T)$ Plotted Semilogarithmically]{Magnetic entropy $S_{\mathrm{mag}}$ of YbPdBi plotted semilogarithmically, which has partially linearized the data above $T=1$~K, with three changes of slope. These changes occur at approximately the entropy values corresponding to additional Kramers doublets: $R\ln2$, $R\ln4$, and $R\ln6$. Cross markers indicate digitized data from LeBras \emph{et al}.~\cite{lebras_local_1995}. }
    \label{fig:entropy_log}
\end{figure}



% \textit{Check it, maybe explain the $R\ln6$. The J=7/2 states could be frozen out, which may give another indication of the local symmetry (if the local symmetry affects how SOC lifts the J=7/2 degeneracy):}
% \begin{figure}
%     \centering
%     \includegraphics[width=0.8\linewidth]{CEF splitting with SOC for 4f electrons in ce3plus.png}
%     \caption{temp graph from "Hybridization and crystal-field effects in Kondo insulators studied by means of core-level spectroscopy" - F. Strigar. 0.1eV=1160K}
%     \label{fig:tempCEFCe3}
% \end{figure}
%%%%%%%%%%%%%%%%%%%%%%%%%%%%%%%
%%%%%%%%%%%%%%%%%%%%%%%%%%%%%%%
%%%%%%%%%%%%%%%%%%%%%%%%%%%%%%%
\subsection{Schottky-Like Anomaly and the CEF States}\label{sec:Schottky_CEF_fitting}

A Schottky anomaly consists of a broad, smooth maximum in specific heat data that is not associated with a phase transition~\cite{gopal_specific_1966, blundell_magnetism_2001}. It occurs as a system is heated due to electrons in their ground state being excited into a distribution of non-ground states. The distribution of those non-ground states is determined by crystalline electric field (CEF) effects. For some lattice ions, the ground state's degeneracy (determined by its total angular momentum $J$) is lifted by the electrostatic potentials of its neighbouring lattice ions. An example of this is shown in Fig.~\ref{fig:CEF_splitting} where the doubly degenerate \textit{d}-orbital states of a free ion are placed in an octahedral environment. Certain orbitals will be energetically more favourable (labelled here as $t_g$), and some will be less favourable (labelled here as $e_g$). This ordering is determined by the local symmetries of the surrounding ions (i.e. for a tetrahedral environment, the order of $t_g$ and $e_g$) will be reversed.\\

\noindent For the \textit{f}-electron Yb$^{3+}$ with $J=7/2$ surrounded by a tetrahedral arrangement of Pd nearest-neighbours, we expect the eight-fold degeneracy of the $J=7/2$ state to split into a $\Gamma_7$ doublet ground state, a $\Gamma_8$ quartet first excited state, and then another doublet $\Gamma_6$ above that~\cite{lea_raising_1962, canfield_groundstate_1994}. Such states are referred to as CEF states.\\ %J=5/2 in a tetragonal environment is split into three doubly degenerate CEF states, gamma_5, gamma^1_7 and gamma^2_7 : https://arxiv.org/pdf/2007.10641


\begin{figure}[h]
    \centering
    \includegraphics[width=\linewidth]{Figures_Ch4_Synth_Charac/Specific Heat/CEF Splitting Diagram.pdf}
    \caption[CEF Splitting of Degenerate Orbitals]{Crystalline electric field (CEF) splitting of a \textit{d}-orbital. (Left) Degenerate states for a free ion. (Right) Lifted degeneracy due to an octahedral CEF environment. A tetrahedral CEF environment has this scheme inverted, with two states at the lower energy and three states at the higher energy. }
    \label{fig:CEF_splitting}
\end{figure}



\noindent We can model these CEF states by constructing the partition function $Z$ for an $n$-level system and deriving several thermodynamic properties~:
\begin{align}\label{eq:n_lvl_thermodynamics}
    Z &= d_0 + \sum_{i=1}^n d_i \exp{-\Delta_i/T}\\
    F &= -R T \ln Z\\
    S &= - \frac{dF}{dT}\\
    C_p &= T\left(\frac{dS}{dT}\right)_p
\end{align}
\noindent where $F$ is the Helmholtz Free energy, $d_i$ are the degeneracies of the $i$~th level, $\Delta_i$ is the energy of the $i$-th level, and $R$ is the universal gas constant which determines our units. The set of energies $\Delta_i$ are in units of Kelvin with $\Delta_0\equiv0$~K. With some work, one can see that the saturation magnetic entropy is indeed a function of the level degeneracies~:
\begin{align}\label{eq:n_lvl_sat_S}
    S_{\mathrm{sat}} &=-\lim_{T\xrightarrow[]{}\infty}\frac{dF}{dT} \\
        &= R \ln\left(\sum_{i=0}^n d_i\right)
\end{align}
\noindent If we consider the $n$-level system as a model for the magnetic contribution to the specific heat, we may also calculate~\cite{pavlosiuk_antiferromagnetism_2016}~:
\begin{equation}\label{eq:schottky_Cv}
    C_{\mathrm{mag}} = R \frac{\sum_i d_i e^{-\Delta_i/T}\sum_i d_i \Delta_i^2 e^{-\Delta_i/T}-\left(\sum_i d_i \Delta_i e^{-\Delta_i/T}\right)^2}{T^2 \left( \sum_i d_i  e^{-\Delta_i/T}\right)^2}
\end{equation}
\noindent We will use these quantities to compare our observed Schottky-like anomaly against the results found by LeBras \emph{et al}.~\cite{lebras_local_1995} for data taken at zero-field. \\

\noindent We first compare our data against two potential CEF splitting schemes found in the literature. We then attempt to model the observed anomaly using a three-level model consisting of the expected degeneracies for Yb$^{3+}$ in a tetrahedral environment: a doublet ground state, a quadruplet first excited state, and a doublet second excited state. We then fit a quasi-degenerate four-level model as was suggested by LeBras \emph{et al}. Neither accurately describes our $C_{\mathrm{mag}}$ data. We then removed the quasidegeneracy restriction for the four-level fit and found it had excellent fidelity to the data. Finally, we investigated a three-level model of doubly degenerate states, inspired by the $S_{\mathrm{mag}}$ plateauing at $R\ln 6$ and the magnetization results for fitted $J$ in Sec.~\ref{sec:mag_pauli_brill_fits}. This somewhat underestimated $C_{\mathrm{mag}}$ when $\mu_0H = 0$~T, but we found it had the greatest success for $C_{\mathrm{mag}}$ in $\mu_0H\neq 0$~T. \\

% \noindent We continue this analysis by observing how the anomaly is affected by $\mu_0H\neq 0$~T. We again attempt to model the anomaly using an four-level system that also accounts for Zeeman splitting. We find that a four-level system is again insufficient to adequately model the data, suggesting that the observed behaviours are driven by something more complicated than simple CEF splitting. Fitting the Schottky-like anomaly with a three-level system had better results for the data taken in zero and non-zero fields.\\

% 

\subsubsection{Comparison with Literature}

As was shown in Fig.~\ref{fig:magnetic_Cv}, we observe a Schottky-like anomaly in $C_{\mathrm{mag}}$  between $T=4$~K and $T^*=34\pm4$~K. This hump was studied previously by LeBras \emph{et al}.~\cite{lebras_local_1995} using $^{170}$Yb M\"ossbauer at $0.5~\text{K} < T < 45~\text{K}$ and using specific heat measurements below $T=50$~K. Their analysis was augmented by inelastic neutron scattering (iNS) data, which provided an estimate of the CEF energies. We have been unable to locate a copy of the iNS data. According to Robinson \emph{et al}.~\cite{robinson_neutron_2000}, this unpublished neutron data is qualitatively comparable to that of the isostructural YbPtBi. M\"ossbauer measurements of the quadrupolar hyperfine interaction in the paramagnetic phase did not possess cubic character at low temperature, implying that the point symmetry of the Yb ions is not cubic.\\

\noindent  Based on their observations, LeBras \emph{et al}. proposed that the CEF states consist of four Kramers doublets, with the energies of the excited states being $\Delta_1=23$~K, $\Delta_2=80$~K,  $\Delta_3=90$~K. This ground state was determined using a mean-field fit of the thermal variation of a spontaneous Yb$^{3+}$ magnetic moment below $T=1$~K in M\"ossbauer measurements of the hyperfine field. LeBras \emph{et al}. observed evidence in their M\"ossbauer data for a lowered (non-cubic) local symmetry for the Yb$^{3+}$, which they attributed to a static Jahn-Teller effect. To account for this in the CEF splitting, they proposed the existence of the quasi-degenerate $\Delta_2$ and $\Delta_3$ excited states. \\

\noindent Shown in Fig.~\ref{fig:schottky_comparison} is a comparison of their proposed four-level scheme of degeneracies against our data for $C_{\mathrm{mag}}$. We have used Eqn.~\ref{eq:schottky_Cv} for this calculation and have plotted it from $T=2$~K to $T=50$~K on a semilogarithmic axis to facilitate comparison with LeBras \emph{et al}.~\cite{lebras_local_1995}. There is some agreement between our data and their CEF scheme, but it seriously over-estimates $C_\mathrm{mag}$ above $T=10$~K.\\



\noindent  One potential source of the discrepancy comes from how we approximated the phonon contribution to the specific heat. In their work, LeBras \emph{et al}. approximate the phonon contribution by fitting low-temperature data of LuPdBi with Eqn.~\ref{eq:low_T_Cv}. In our case, we instead scaled up specific heat data of YPdBi using a temperature-dependent Debye temperature and interpolated the results (see App.~\ref{app:phonon_Cv_calc}). \\ %However, we cannot necessarily attribute this discrepancy to differing methods of approximating the phonon specific heat. \\

\begin{figure}[h]
    \centering
    \includegraphics[width=0.8\linewidth]{Figures_Ch4_Synth_Charac/Specific Heat/YbPdBi, Cmag Schottky Comparison.pdf}
    \caption[Schottky-Like Anomaly CEF Compared to LeBras \emph{et al}.~\cite{lebras_local_1995}]{Comparison of our magnetic specific heat $C_{\mathrm{mag}}$ data against the quasidegenerate four-level CEF scheme proposed in LeBras \emph{et al}.~\cite{lebras_local_1995}, where each level is doubly degenerate. It is clear that there is little overlap above $T=10$~K between their model and our data. We have formatted this plot on a semilogarithmic axis for a range of $T$ that matches the range of the figure shown in~\cite{lebras_local_1995}.}
    \label{fig:schottky_comparison}
\end{figure}



\noindent It should be noted that LeBras \emph{et al}. also concluded that there was only minor agreement between their CEF model and their $C_{\mathrm{mag}}$ data. We are simply confirming that the discrepancy exists. It is difficult to assess why there is such a discrepancy between the CEF scheme predicted by their M\"ossbauer measurements, their inelastic neutron scattering results, and both sets of $C_{\mathrm{mag}}$ data. It is not impossible that the thesis in which these iNS measurements were published was retracted. We also do not see evidence of their proposed Jahn-Teller effect in our elastic neutron scattering data, presented in Sec.~\ref{sec:neutrons}.\\

\noindent Interestingly, there was some trouble in determining the scheme of CEF states for isostructural YbPtBi. After multiple schemes were proposed, Robinson \emph{et al.}~\cite{robinson_low-energy_1995} suggested four doublet states with energies $\Delta_1=20$~K, $\Delta_1=80$~K, and $\Delta_4=400$~K. They motivate this by the presence of a non-cubic environment around the Yb$^{3+}$ ion, seen in neutron scatter data. This is similar to the lowered point symmetry seen in previous M\"ossbauer measurements in~\cite{lebras_local_1995}. If the previously reported similarity in the iNS data for YbPdBi with that of YbPtBi is accurate, a similar scheme could apply to YbPdBi. This would also explain away the observed saturation entropy of $R\ln 6$.\\

\noindent Shown in Fig.~\ref{fig:robinson_CEF} is a plot of the CEF scheme for YbPtBi that Robinson \emph{et al.} proposed over our $C_{\mathrm{mag}}$ data for YbPdBi. It has a similar fidelity to the data as the scheme by LeBras \emph{et al.} up to $T=10$~K, above which it less over-estimates the data. We attempted to best fit the data with a similar four-level scheme by restricting the highest energy state to be above $300$~K. However, this energy parameter would always grow to thousands of degrees and converge on whatever upper bound we attempted to impose on it.\\

\begin{figure}[h]
    \centering
    \includegraphics[width=0.8\linewidth]{Figures_Ch4_Synth_Charac/Specific Heat/YbPdBi, Cmag Schottky Comparison Robinson.pdf}
    \caption[Schottky-Like Anomaly CEF Compared to Robinson \emph{et al}.~\cite{robinson_low-energy_1995} for YbPtBi]{Comparison of our YbPdBi magnetic specific heat $C_{\mathrm{mag}}$ data against the four-level CEF scheme for YbPtBi proposed by Robinson \emph{et al.}~\cite{robinson_low-energy_1995}. Each level is doubly degenerate with energies $\Delta_1=20$~K, $\Delta_1=80$~K, and $\Delta_4=400$~K.}
    \label{fig:robinson_CEF}
\end{figure}


%%%%%%%%%%%%%%%%%%%%%%%%%%%%%%%
%%%%%%%%%%%%%%%%%%%%%%%%%%%%%%%
\subsubsection{Schottky Fitting for $\mu_0H = 0$~T}\label{sec:H0_schottky_CEF_fit}

To further examine the CEF degeneracies, we decided to perform best fits of our $C_{\mathrm{mag}}$ data using Eqn.~\ref{eq:schottky_Cv}. We initially fitted a four-level system where each level is doubly degenerate, following the CEF scheme proposed by LeBras \emph{et al.} We enforced the quasi-degeneracy of the two most energetic states we restricted one state to be within $30$~K of the other. However, the resulting fits failed to describe the Schottky-like anomaly. We found that removing the quasidegenerate restriction improved the fit results. All fits were performed from $T=4$~K to $T=50$~K.\\

%%%%% FREE FIT OF 4-LEVEL SYSTEM
\noindent Shown in Fig.~\ref{fig:Schottky_fit_4lvl} are the results of the fit of the four-level system of doubly degenerate states without the quasidegeneracy restriction. The fitted energies are $\Delta_1 =13.5\pm0.2$~K, $\Delta_2 =42\pm1$~K, and $\Delta_3= 150\pm6$~K. Though the model does not perfectly capture $C_{\mathrm{mag}}$, these energies describe our data with more fidelity than the energies given by LeBras \emph{et al}. However, the energies of $\Delta_2$ and $\Delta_3$ disagree with the iNS data reported by LeBras \emph{et al}. As well, the value of $\Delta_3$ is low compared to what we'd expect due to the saturization entropy results.\\


\begin{figure}[h]
    \centering
    \includegraphics[width=0.75\linewidth]{Figures_Ch4_Synth_Charac/Specific Heat/YbPdBi, Schottky 4 level lebras unrestricted best fit.pdf}
    \caption[Four-Level Fit of Kramers Doublets]{Four-level system fitted to Schottky-like anomaly using Eqn.~\ref{eq:schottky_Cv}. All four levels are doubly degenerate, with corresponding energies of $\Delta_0\equiv0$, $\Delta_1 =(13.5\pm0.2)$~K, $\Delta_2 =(42\pm1)$~K, and $\Delta_3 =(150\pm6)$~K.}
    \label{fig:Schottky_fit_4lvl}
\end{figure}


%%%%%  FIT OF 3-LEVEL TETRAHEDRAL SYSTEM
\noindent We next fit a three-level system according to the expected splitting for a $J=7/2$ ground state in a tetrahedral environment: a doublet ground state, a quadruplet first excited state, and a doublet second excited state. The results of this fit are shown in Fig.~\ref{fig:Schottky_fit_3lvl_tetrahedral}. The position of the peak in the fit is accurate but significantly overestimates the height of the peak.\\

\begin{figure}[h]
    \centering
    \includegraphics[width=0.75\linewidth]{Figures_Ch4_Synth_Charac/Specific Heat/YbPdBi, Schottky 3 level tetrahedral.pdf}
    \caption[Three-Level Fit of Tetrahedral CEF Scheme]{Three-level system fitted to Schottky-like anomaly using Eqn.~\ref{eq:schottky_Cv} using the CEF splitting scheme for a $J=7/2$ ground state in a tetrahedral CEF environment. The system consists of a doublet ground state, a quadruplet first excited state, and a doublet second excited state. The corresponding fitted excited energies are $\Delta_1 =(20.2\pm0.9)$~K and $\Delta_2 =(100\pm16)$~K.}
    \label{fig:Schottky_fit_3lvl_tetrahedral}
\end{figure}

%%%%% FREE FIT OF 3-LEVEL SYSTEM of doublets
\noindent Finally, we fitted a three-level system of doublets to test the $R\ln6$ saturation entropy found in Sec.~\ref{sec:magnetic_entropy}. As was stated previously, such a value for $S_{\mathrm{sat}}$ implies that the CEF degeneracies sum only to six, rather than the eight expected for $J=7/2$. It is possible that the fourth Kramers doublet, which would exist above $T=300$~K, does not play a role in the Schottky anomaly, or that there is some mixed valency effect occurring for the Yb ions. Best fits of the magnetization in Sec.~\ref{sec:mag_pauli_brill_fits} suggest the possibility that $J=5/2$. The results of this fit are plotted in Fig.~\ref{fig:Schottky_3lvl_doublets}, with fitted energies of $\Delta_1=(21.2\pm0.9)$~K and $\Delta_2=(113\pm5)$~K. Data points were restricted to $T<50~$K to match the boundaries of the four-level fit. \\

\noindent Though the model is qualitatively similar to the data, it does not fully account for the magnetic-specific heat. However, these energies line up better with the iNS data reported by LeBras \emph{et al}. than the four-level fit. Further, LeBras \emph{et al}. reported that having a Kramers doublet as the second excited state was allowed within their M\"ossbauer analysis. They replaced it with two quasi-degenerate doublet states to account for a purported JTE associated with a lower symmetry for the Yb$^{3+}$ site.\\





\begin{figure}[h]
    \centering
    \includegraphics[width=0.75\linewidth]{Figures_Ch4_Synth_Charac/Specific Heat/YbPdBi, Schottky Fit 3 level.pdf}
    \caption[Three-Level Fit of Kramers Doublets]{Three-level system of doubly degenerate states fitted to a Schottky-like anomaly using Eqn.~\ref{eq:schottky_Cv}. The corresponding energies are  $\Delta_0\equiv0$; $\Delta_1 =(17.4\pm0.7)$~K; $\Delta_2 =(73\pm3)$~K.}
    \label{fig:Schottky_3lvl_doublets}
\end{figure}


\noindent An interesting possibility is that the language of CEF may simply be unsuited for describing the low-energy states of YbPdBi's magnetic moments. This was raised previously for YbPtBi~\cite{canfield_groundstate_1994} due to the Kondo temperature being comparable to the first CEF splitting. Furthermore, the development of a heavy fermion phase as $T$ is lowered means the magnetic moments of the lattice are hybridizing with the conduction electrons. In fact, we see evidence in the magnetization measurements in Sec.~\ref{sec:mag_sus} for a temperature dependence in the $J$ parameter of the Brillouin function. This could be significant for the Schottky-like anomaly, since it is the degenerate states of $J$ that are considered to be participating in the CEF splitting. \\


% %%%%%%%%%%%%%%%%%%%%%%%%%%%%%%%
% %%%%%%%%%%%%%%%%%%%%%%%%%%%%%%%
% \subsubsection{Schottky Fit when $H\geq0$~T}\label{sec:schottky_H_geq_0}
\subsubsection{Schottky Fitting for $\mu_0H\geq0$~T}

To further investigate the potential CEF effects on the specific heat, we obtained specific heat data between $T=4$~K and $T=50$~K under applied magnetic fields of $\mu_0H\in \{4.5~\text{T}, 6.75~\text{T}, 9~\text{T}\}$, which is shown in Fig.~\ref{fig:specific_heat_in_field}. Clearly, the Schottky-like anomaly exhibits a strong dependence on $H$. The maximum value increases by more than $50$~\%, while the temperature of this maximum value also increases with $H$. Surprisingly, similar behaviour was observed in $C_{\mathrm{mag}}$ for $\mu_0H<5$~T in isostructural YbPtBi~\cite{mun_yb-based_nodate}.\\

\noindent There is a kink in the curve visible for $\mu_0H=4.5$~T near $T=13$~K, but the low resolution makes it difficult to discern details. There appears to be a smaller kink for the $\mu_0H=6.75$~T data near this temperature, suggesting this kink is not simply an illusion caused by low resolution. This kink is not visible in the zero-field data. It is possible that the $T=1$~K transition is the source of this kink, having been pushed upwards in temperature by the magnetic field. However, measurements of the magnetization as a function of $H$ in Sec.~\ref{sec:mag_sus} show no indication of magnetic ordering occurring as we sweep $H$ for $T=2$~K and $T=5$~K.\\

\noindent Similar kinks were seen in measurements of isostructural YbPtBi~\cite{mun_yb-based_nodate}. There, it appears the low-$T$ $\lambda$-type peak associated with magnetic ordering is suppressed and shifted to higher temperatures. Our data appears to show similar behaviour, but sub-Kelvin measurements taken with a finer $H$ resolution will be necessary to verify this.\\

\begin{figure}[h]
    \centering
    \includegraphics[width=\textwidth]{Figures_Ch4_Synth_Charac/Specific Heat/YbPdBi, JF011, in-field Low-T specific heat, no phonons.pdf}
    \caption[Enhancement of the Schottky-Like Anomaly for $\mu_0H\neq 0$~T]{Magnetic specific heat data $C_{\mathrm{mag}}$ of the Schottky-like anomaly. Not only does the anomaly's maximum grow in height, but the peak shape seems to sharpen. As well, there is an apparent kink present near $T=15$~K for data taken in a non-zero field. Data digitized from LeBras \emph{et al}.~\cite{lebras_local_1995} is presented as crosses. The dashed lines are guides for the eyes and do not signify data points.}
    \label{fig:specific_heat_in_field}
\end{figure}


\noindent To adjust the CEF model, we assume Zeeman splitting will lift each of the doubly degenerate states into two non-degenerate states, such that the energy of each doubly degenerate level is split~:
\begin{equation}\label{eq:schotky_zeeman_splitting}
      \Delta_i \xrightarrow[]{} 
      \begin{cases}
      \Delta_i^A = \Delta_i-\beta_i H\\
      \Delta_i^B = \Delta_i+\beta_i H
      \end{cases}
\end{equation}
\noindent where $\beta_i$ is a scalar value. For an electron orbiting an atom, $\beta_i$ depends on a linear combination of $L$, $S$, and their respective projections onto $H$~\cite{townsend_modern_2000}. The energy scale is adjusted such that the lowered energy of the ground state is defined as zero.  Unfortunately, we were only able to obtain a representative fit using the three-level system of doublets shown previously in Fig.~\ref{fig:Schottky_fit_4lvl}. Additionally, the fits would not converge to a representative result if we did not allow the fit parameters $\beta_i$ to vary with $H$. This is not the expected behaviour for Zeeman splitting. \\


\noindent Shown in Fig.~\ref{fig:3lvl_4lvl_schottky}a are the fits of the three-level system of doublets plotted with the data. The fit parameters for which are shown in Fig.~\ref{fig:3lvl_4lvl_schottky}b. We re-plot the $\mu_0H=0$~T data with its three-level fit, for comparison. In all cases, the fits underestimate the height of each peak, though less so for $\mu_0 H= 6.75$~T. For each increasing $H$, the drop-off as $T$ is increased above the peak maxima and is most closely matched by the fits, but there is still noticeable disagreement. As was said, the fit parameters $\beta_i$ show a dependence on $H$, suggesting a non-linear relationship with the splitting of the energy levels.  We therefore find it difficult to suggest that Zeeman splitting of the CEF states is sufficient in explaining the observed changes in the Schottky-like anomaly in a magnetic field.\\
% The $R\ln6$ saturation entropy suggests that the three-level scheme is the more accurate model of the CEF behaviour. However, we include a plot of the fits and parameters for the four-level scheme in Fig.~\ref{fig:3lvl_4lvl_schottky}c and Fig.~\ref{fig:3lvl_4lvl_schottky}d, respectively, for the sake of completeness. \\



\begin{figure}[H]
    \centering
    \begin{subfigure}[t]{0.04\textwidth}
        \textbf{a)}
    \end{subfigure}    
    \centering
    \begin{subfigure}[t]{0.9\textwidth}
    \includegraphics[width=\linewidth, valign=t]{Figures_Ch4_Synth_Charac/Specific Heat/Schottky in-Field Zeeman Splitting Fit COMBINED 3lvl.pdf} 
    \end{subfigure}
    \centering
    \begin{subfigure}[t]{0.04\textwidth}
        \textbf{b)}
    \end{subfigure}    
    \centering
    \begin{subfigure}[t]{0.9\textwidth}
    \includegraphics[width=\linewidth, valign=t]{Figures_Ch4_Synth_Charac/Specific Heat/YbPdBi, Schottky in-Field Zeeman Splitting Fit Params 3lvl.pdf} 
    \end{subfigure}
    % \centering
    % \begin{subfigure}[t]{0.01\textwidth}
    %     \textbf{c)}
    % \end{subfigure}    
    % \centering
    % \begin{subfigure}[t]{0.475\textwidth}
    % \includegraphics[width=\linewidth, valign=t]{Figures_Ch4_Synth_Charac/Specific Heat/Schottky in-Field Zeeman Splitting Fit COMBINED 4lvl.pdf} 
    % \end{subfigure}
    % \centering
    % \begin{subfigure}[t]{0.01\textwidth}
    %     \textbf{d)}
    % \end{subfigure}    
    % \centering
    % \begin{subfigure}[t]{0.475\textwidth}
    % \includegraphics[width=\linewidth, valign=t]{Figures_Ch4_Synth_Charac/Specific Heat/YbPdBi, Schottky in-Field Zeeman Splitting Fit Params 4lvl.pdf} 
    % \end{subfigure}
    \caption[Fitted Schottky-Like Anomaly for $\mu_0H\geq 0$~T]{Specific heat data for $\mu_0H\geq0$~T fitted using Eqn.~\ref{eq:schottky_Cv}. (a) The fits represent a three-level system of doubly degenerate levels, which are modified according to Eqn.~\ref{eq:schotky_zeeman_splitting} due to Zeeman splitting. (b) Respective splitting parameters $\beta_i$ for the CEF states with corresponding energies  $\Delta_0\equiv0$; $\Delta_1 =(17.4\pm0.7)$~K; $\Delta_2 =(73\pm3)$~K.}
    \label{fig:3lvl_4lvl_schottky}
\end{figure}




%%%%%%%%%%%%%%%%%%%%%%%%%%%%%%
%%%%%%%%%%%%%%%%%%%%%%%%%%%%%%
%%%%%%%%%%%%%%%%%%%%%%%%%%%%%%
\newpage
\section{Magnetization $M(T,H)$ and Magnetic Susceptibility $\chi(T,H)$ }\label{sec:mag_sus}

The magnetic properties of YbPdBi are dominated by the unpaired  \textit{4f}-electron of the Yb$^{3+}$ ions. These unpaired electrons each possess a magnetic moment due to both their intrinsic spin $S$ and their orbital angular momentum $L$. In a ferromagnetic (FM) material, the interaction of these magnetic moments is such that their parallel alignment is energetically favourable. In antiferromagnetic (AFM) materials, their antiparallel alignment is instead energetically favourable. If the directions of the magnetic moments are randomly fluctuating due to the thermal energy of the system, it is in a paramagnetic state. Cooling a material that is in a paramagnetic state will typically induce a phase transition to a ferromagnetic or antiferromagnetic state. More exotic ground states are also possible, such as the randomly oriented moments of spin glasses and spin liquids. Exotic behaviour can also be found in the paramagnetic state, such as the Kondo effect and heavy fermion (HF) behaviour.\\


\noindent Several methods have been developed to probe these magnetic interactions. One of the most sensitive of these is through the use of a superconducting quantum interference device, also known as a SQUID~\cite{blundell_magnetism_2001, kittel_introduction_2005}. It makes use of the Josephson effect, wherein two superconducting segments are separated by a thin insulating material, creating a Josephson Junction. The insulator is thin enough to permit the electrons of the current to tunnel through it without breaking the pairings that make them superconducting~\cite{ashcroft_solid_1976}. \\

\noindent To create the SQUID itself, one joins two such Josephson junctions in parallel, forming a ring of superconducting material as shown in Fig.~\ref{fig:SQUID_diagram} is a diagram of a SQUID. We pass a magnetic flux $\phi$ through the ring and a current around either side of the ring. Due to the non-zero $\phi$, the phase of the electrons in the current is shifted, depending on having taken either the path along $I_A$ or $I_B$. An interference effect may be observed as a function of $\phi$ when the currents are recombined. Thus, the SQUID acts as a quantum interferometer that is very sensitive to changes in $\phi$, such that they can measure a sample's magnetization $M(T,H)$.\\

\begin{figure}[h]
    \centering
    \includegraphics[width=0.7\linewidth]{Figures_Ch4_Synth_Charac/Mag_and_Sus/SquidDiagram.pdf}
    \caption[SQUID Diagram]{Diagram of a SQUID device with two Josephson Junctions labelled A and B. The Junctions are connected by a superconducting material (shown in blue sections), forming a ring. The ring carries a current $I$, which is split into $I_A$ and $I_B$ that each cross a Josephson Junction. A magnetic field $H$ passes through the centre of the ring, directed into the page, creating a flux $\phi$ proportional to the area enclosed by the ring and the magnitude of $H$.}
    \label{fig:SQUID_diagram}
\end{figure}

% \noindent Magnetization measurements were taken at Universidade de Campinas by graduate student Raphel François under the supervision of Professor Cris Adriano. These measurements were performed on YbPdBi crystals grown using the Pb-flux method, and our analyses of these measurements are presented in this section. All data was taken in the DC mode of operation.\\

%%%%%%%%%%%%%%%%%%%%
%%%%%%%%%%%%%%%%%%%%
\subsection{Inverse Magnetic Susceptibility $\chi^{-1}(T)$}%$ and  $M(T)$}

Shown in Fig.~\ref{fig:inv_sus_Brazil} is the inverse magnetic susceptibility $\chi^{-1}(T)=\frac{M(T)}{H}$ derived from measurements of the molar magnetization $M(T)$ taken at $\mu_0H=0.1$~T from $T=2$~K to $T=260$~K. The dashed vertical black line corresponds to the coherence temperature $T^*$ where the heavy fermion phase begins to form. The solid black line represents a fit of the Curie-Weiss law~\cite{ashcroft_solid_1976}~:
\begin{equation}\label{eq:curie_weiss_law_inv_sus}
    \chi^{-1}(T) = \frac{C_\mathrm{CW}}{T-\Theta_\mathrm{CW}}
\end{equation}
\noindent where $C_\mathrm{CW}$ is the Curie constant, and $\Theta_\mathrm{CW}$ is the Curie-Weiss temperature. The dashed red vertical line corresponds to the lower bound of the fitted data, $T=44$~K. Below this bound, the data begins to deviate away from linear. \\

\begin{figure}[h]
    \centering
    \includegraphics[width=0.9\linewidth]{Figures_Ch4_Synth_Charac/Mag_and_Sus/JF011, YbPdBi, Fitted InvSus7p88mg_1kOe.pdf}
    \caption[Inverse Magnetic Susceptibility $1/\chi$ for YbPdBi]{Inverse magnetic susceptibility $\chi^{-1}$ for YbPdBi in a magnetic field of $\mu_0H=0.1$~T fitted with Eqn.~\ref{eq:curie_weiss_law_inv_sus}. The dashed black vertical line represents the coherence temperature $T^*$. The dashed red vertical line represents the minimum temperature that was fitted,  $T=42$~K. The solid black line represents the fit, with Curie constant $C_\mathrm{CW}=(2.4389\pm0.0003)$~emu-K/mol and a Curie-Weiss temperature $\Theta_\mathrm{CW} = (-8.14\pm0.02)$~K which corresponds to antiferromagnetic interactions between the magnetic moments of the lattice. (Inset) The data curves downward at lower temperatures.}\label{fig:inv_sus_Brazil}
\end{figure}

\noindent The data above $T=42$~K is well described Eqn.~\ref{eq:curie_weiss_law_inv_sus} with $C_\mathrm{CW}=(2.4389\pm0.0003)$~emu-K/mol and $\Theta_\mathrm{CW} = (-8.14\pm0.02)$~K, indicating that the sample's paramagnetic phase is dominated by AFM interactions, which is expected for a heavy fermion material (HFM)~\cite{coleman_heavy_2015}. From the fitted value of $C_\mathrm{CW}$, we obtain an effective moment of $\mu_{eff} = (4.4171\pm0.0003)$~$\mu_\mathrm{B}$ for the Yb$^{3+}$ ion. This is less than that expected for the free ion moment, $4.54$~$\mu_{B}$.  Previous measurements found $\mu_{eff}=4.39$~$\mu_\mathrm{B}$ and $\Theta_\mathrm{CW}=-9$~K~\cite{dhar_heavy-fermion_1988}, $\mu_{eff}=4.4$~$\mu_\mathrm{B}$ and $\Theta_\mathrm{CW}=-6$~K~\cite{lebras_local_1995}, $\mu_{eff}=4.11(5)$~$\mu_\mathrm{B}$ and $\Theta_\mathrm{CW}=-3.8(9)$~K~\cite{kaczorowski_magnetic_1999}. \\

% \noindent This shift away from Curie-Weiss behaviour for local moments is expected for heavy fermion compounds near $T^*$~\cite{yang_scaling_2008}, but it is likely attributable to $CEF$ effects.  As the temperature is lowered, the magnetic moment will change as the excited crystal field states are depopulated as the temperature is lowered~\cite{mugiraneza_tutorial_2022}. Since the deviation begins near $T=44$~K, it is unlikely that the phase transition at $T=1$~K is the cause. \\

\noindent Before continuing, we'd like to discuss the behaviour of $\chi(T)$ near the purported magnetic phase transition at $T=1$~K. Measurements of $\chi(T)$ between $0.5~\text{K}<T<2$~K at $H=66$~G were performed previously in LeBras \emph{et al}.~\cite{lebras_local_1995}. The curvature of $\chi(T)$ changes sign below $T=1$~K, reminiscent of a ferrimagnetic or ferromagnetic state rather than antiferromagnetic. Note that we do not see evidence for magnetic ordering in our powder neutron diffraction data (see Sec.~\ref{sec:neutrons}).\\ 
%Due to equipment considerations, we were unable to attempt reproducing this data. \\

% \noindent Above $T=1$~K, $\chi(T)$ behaves paramagnetically in accordance with the Curie-Weiss Law. Below $T=1$~K, however, the curvature of $\chi(T)$  changes sign and the curve longer behaves paramagnetically. This, along with the peak in specific heat data and the spontaneous magnetic moment below $T=1$~K, suggests that it is very likely that YbPdBi undergoes a magnetic phase transition near $T=1$~K. The authors of~\cite{lebras_local_1995} commented that this magnetic phase transition exhibits behaviour that is typical of neither a ferromagnetic (where $\chi(T)$ is expected to saturate below $T_C$) nor for an antiferromagnetic (where $\chi(T)$ reaches a maximum at $T_N$). Kaczorowski \emph{et al}. 1999~\cite{kaczorowski_magnetic_1999} summarize the contemporaneous literature discussing YbPdBi and propose that the $T=1$~K peak shows the coexistence of a heavy fermion state with long-range magnetic ordering. \\

% \begin{figure}[h]
%     \centering
%     \includegraphics[width=0.6\linewidth]{Figures_Ch4_Synth_Charac/Mag_and_Sus/LeBras graph susceptibility below 2K.png}
%     \caption[Re-Print of $\chi(T<2~\text{K})$ from LeBras \emph{et al}.~\cite{lebras_local_1995}]{Re-print of Figure 7 from LeBras \emph{et al}.~\cite{lebras_local_1995}. At $T=1$~K, the curvature of $\chi(T)=\frac{M(T)}{H}$ abruptly changes sign, coincidentally with a peak in the specific heat.}
%     \label{fig:LeBrasMagReprint}
% \end{figure}

% \noindent One possible, but not necessarily likely, source for this unexpected behaviour is that there is an unidentified impurity phase present in their YbPdBi samples. LeBras \emph{et al}. state that this impurity phase has a relative weight of 10\%. Should it possess Yb$^{3+}$ ions, we would expect it to influence the measured values of $\chi(T)$, which would then be an averaging of the two curves. when both YbPdBi and the impurity are both in their paramagnetic phase, the two $\chi(T)$ curves ought to be comparable so long as the number density of magnetic moments doesn't change significantly. It would, therefore, only be when one of the compounds undergoes a phase transition that any effect would be observed in $\chi(T)$. \\

% \noindent If YbPdBi transitions at $T=1$~K to an antiferromagnetic state, its contribution to $\chi(T)$ will begin to decrease while the contribution from the impurity phase will continue to increase paramagnetically. The impurity phase would necessarily have to order either coincidentally at $T=1$~K (necessarily to a ferromagnetic state) or below $T=0.5$~K, based on the lack of additional peaks in their specific heat data and the agreement in literature regarding the $T=1$~K peak belonging to YbPdBi. This hypothesis earns demerits, however, when one recalls that their impurity phase has a relative weight of 10\%. Since the susceptibility is still increasing below $T=1$~K, the paramagnetic contribution from that 10\% would need to not just equal but exceed the contribution from YbPdBi. Therefore, for this hypothesis to hold any water, it would require that it is instead YbPdBi that contibues to be in a paramagnetic state, while the impurity phase orders antiferromagnetically.\\

% \begin{figure}[h]
%     \centering
%     \includegraphics[width=\linewidth]{Figures_Ch4_Synth_Charac/Mag_and_Sus/JF011, YbPdBi, mu_eff vs T.pdf}
%     \caption[Temperature Dependence of $\mu_{eff}$]{Temperature Dependence of the effective moment of Yb$^{3+}$, calculated via $\mu_{eff} = 2.827\sqrt{\chi(T) T}$.}
%     \label{fig:T_dependence_mu_eff}
% \end{figure}



%%%%%%%%%%%%%%%%%%%%
%%%%%%%%%%%%%%%%%%%%
\subsection{Enhanced Pauli Magnetism in $M(H)$}\label{sec:mag_pauli_brill_fits}
% TODO:
% \begin{enumerate}
%     \item fix $B$, make it $H$. Maybe deal with $\mu_0 H$
%     % \item Choose whether these are functions of $H$ or $B=\mu_0H$. Check with Andrea whether to use $H$ or . Should I actually calculate $\mu_0 H$ or is simply dividing $Oe$ by 10000 to get $T$ sufficient?
%     % \item refer to CEF fitting, ask what a T-dependent $J$ would do or be caused by. Relative strengthening of SOC as system cools, affecting how $J$ is calculated? Transitioning from LS coupling to jj coupling?
%         % \item Add std errors to the fit parameters.
%     % \end{enumerate}
% \end{enumerate}
\begin{figure}[h]
    \centering
    \includegraphics[width=\linewidth]{Figures_Ch4_Synth_Charac/Mag_and_Sus/JF011, YbPdBi, muB M(H) Data.pdf}
    \caption[Magnetization $M(H)$ for $2~\text{K}\leq T\ \leq 70~\text{K}$]{Magnetization $M(H)$ for YbPdBi at a range of temperatures $T(K)$ in units of $\mu_\mathrm{B}/\text{Yb}$.}
    \label{fig:M_of_H_Brazil}
\end{figure}

Shown in Fig.~\ref{fig:M_of_H_Brazil} is the magnetization $M(H)$ in units of $\mu_\mathrm{B}/\text{Yb}$ for YbPdBi for $\mu_0H\leq7$~T for $T\in\{70~\text{K}, 15~\text{K}, 10~\text{K}, 5~\text{K}, 2~\text{K}\}$. At high temperatures, $M(H)$ increases linearly, with non-linearity developing for $T\leq15$~K. The data does not saturate below $\mu_0H\leq7$~T at these temperatures. This observation is used in Sec.~\ref{sec:Hall_measurementst} to discount the anomalous Hall effect as the source for the change in slope of $\rho_{xy}(H)$. It is safe to assume the magnetization curves above $T=70$~K will continue to behave linearly in $H$, with a slope that decreases as $T$ increases. \\

% Finally, the lack of discontinuities in $M(H)$ does not see any magnetic transition near $T=1$~K shifting to higher temperatures with increasing $H$. Such a shift was observed in YbPtBi~\cite{mun_yb-based_nodate} and is one possible explanation for the increased maximum of the Schottky-like anomaly in Sec.~\ref{sec:Schottky_CEF_fitting}.\\


\noindent We can model $M(H)$ for an ideal paramagnetic system with the Brillouin function~\cite{blundell_magnetism_2001}~:
\begin{align}\label{eq:simp_brillouin}
    M(H) &= M_s B_J\left(xH\right)\\
    x &= \frac{g_J(J)\mu_\mathrm{B}J}{k_\mathrm{B}T}\\
    g_J(J) &= \frac{3}{2} + \frac{S(S+1)-L(L+1)}{2J(J+1)}
\end{align}
\noindent where $M_s$ is the saturization magnetization, $B_J(y)$ is the Brillouin function, and  $g_J(J)$ is the Land\'e g-factor where $S=1/2$ and $L=J-S=3$ is the orbital angular momentum. Eqn.~\ref{eq:simp_brillouin} describes the magnetization of a sample as an ensemble average of the magnetic moments of an ideal paramagnet. Where it ceases to describe the data, one can say that the system is not behaving as an ideal paramagnet. \\

\noindent In the following, both $M_s$ and $J$ are treated as fit parameters whenever curve fits are performed using Eqn.~\ref{eq:simp_brillouin}. Neither $M_s$ nor $J$ are expected to have a temperature dependence, and $J$ is theoretically restricted to integer and half-integer values. Nonetheless, allowing these to be fit parameters was the only way we could fit the data shown in Fig.~\ref{fig:M_of_H_Brazil}. For Yb$^{3+}$, we expect $J=7/2$~\cite{blundell_magnetism_2001}, though our observed magnetic entropy in Sec.~\ref{sec:magnetic_entropy} saturates at $R\ln6$ (suggesting $J=5/2$) rather than $R\ln8$ below $T=300$~K. If the fitted values of $J$ show a significant temperature dependence, it could help explain why it was difficult to describe the Schottky-like anomaly in Sec.~\ref{sec:Schottky_CEF_fitting} using the language of CEF states. The $n$-level systems we used to model the CEF splitting assume that $J$ is temperature-independent.\\

%That being said, it is important to note that we are not directly measuring the value of $J$ for any specific site. Instead, we are inferring the values of $J$ from the magnetization, which is a bulk measurement across the whole sample. Should our fit parameter $J$ exhibit a temperature dependence, the true temperature dependence could lie in the physics involved in how the ensemble of moments translates into the sample's magnetization.\\

\noindent Shown in Fig.~\ref{fig:simple_brill_fits_no_pauli} are fits of $M(H)$ to Eqn.~\ref{eq:simp_brillouin}. Fit residuals are plotted above their respective fits. Only data taken at $T\geq10$~K are well-fitted. At $T=10$~K, an oscillation in the fit residuals appears, growing in amplitude as $T$ is lowered and as $H$ increases. A small offset at low-$H$ is present in the residuals of $T=15$~K and $T=10$~K. It is clear that Eqn.~\ref{eq:simp_brillouin} insufficiently represents the data, meaning we must look beyond simple paramagnetism to describe the low-temperature data.\\
% We therefore conclude that Eqn.~\ref{eq:simp_brillouin} alone is insufficient to adequately describe the $M(H)$ data. \\

\begin{figure}%[h]
    \centering
     \begin{subfigure}{0.49\linewidth}
     \includegraphics[width=\linewidth]{Figures_Ch4_Synth_Charac/Mag_and_Sus/Pauli Mag Fit/YbPdBi, Simple Brill Fit, T70K J2p35 Mp0 Ms4p1.pdf} 
    \end{subfigure}    
    \centering
     \begin{subfigure}{0.49\linewidth}
     \includegraphics[width=\linewidth]{Figures_Ch4_Synth_Charac/Mag_and_Sus/Pauli Mag Fit/YbPdBi, Simple Brill Fit, T15K J2p41 Mp0 Ms3p36.pdf} 
    \end{subfigure}
    \centering
    \begin{subfigure}{0.49\linewidth} 
    \includegraphics[width=\linewidth]{Figures_Ch4_Synth_Charac/Mag_and_Sus/Pauli Mag Fit/YbPdBi, Simple Brill Fit, T10K J2p19 Mp0 Ms3p17.pdf}
    \end{subfigure}
    \centering
     \begin{subfigure}{0.49\linewidth}
     \includegraphics[width=\linewidth]{Figures_Ch4_Synth_Charac/Mag_and_Sus/Pauli Mag Fit/YbPdBi, Simple Brill Fit, T5K J1p69 Mp0 Ms2p77.pdf} 
    \end{subfigure}
    \centering
    \begin{subfigure}{0.49\linewidth} 
    \includegraphics[width=\linewidth]{Figures_Ch4_Synth_Charac/Mag_and_Sus/Pauli Mag Fit/YbPdBi, Simple Brill Fit, T2K J0p72 Mp0 Ms2p19.pdf}
    \end{subfigure}
    \caption[Brillouin Fits of $M(H)$ Without Pauli Term]{Best fits of magnetization $M(H)$ to Eqn.~\ref{eq:simp_brillouin}.  Residuals of each fit are plotted above the data (blue dots) and the best fit (orange lines).
    % Fit parameters are plotted in Fig.~\ref{fig:simple_brill_fits_params_plots}.
    }\label{fig:simple_brill_fits_no_pauli}
\end{figure}


\noindent We found that the fits were significantly improved by including a term linearly proportional to $H$, which we attribute to strong Pauli paramagnetism ($M_\mathrm{P}$). For a metallic sample, the Pauli paramagnetic term describes the conduction electron contribution to the magnetization via their spins. Due to Zeeman splitting, conduction electron states whose spins are aligned parallel to the magnetic field are energetically favourable compared to those states whose spins are antiparallel~\cite{blundell_magnetism_2001}. Since $E_\mathrm{F}$ remains the same, an imbalance in the quantity of available states below $E_\mathrm{F}$ occurs, where one spin has more occupied states than the opposite spin. Electrons will fill those newly available states from the other now unavailable states, resulting in a net magnetization $M_\mathrm{P}$. Only states that were previously near $E_\mathrm{F}$ will switch their spin state, which necessarily means that it is the conduction electrons that are involved. At zero temperature, for $H$ applied “downwards”~\cite{blundell_magnetism_2001}~:
\begin{equation}\label{eq:Pauli_magnetization}
    M_\mathrm{P}= \mu_{\mathrm{B}}(n_{\uparrow}-n_{\downarrow}) = g\left(E_\mathrm{F}\right)\mu_{\mathrm{B}}^2 H
\end{equation}
\noindent where $n_{\uparrow/\downarrow}$ is the number of states whose spins are antiparallel/parallel to $H$ and $g(E_\mathrm{F})$ is the density of states at the Fermi energy. The derivation of Eqn.~\ref{eq:Pauli_magnetization} neglects the fact that the Fermi surface will smear out at finite temperatures. When $0~\text{K}< T<<T_\mathrm{F}$, we expect $M_\mathrm{P}\propto -\frac{T^2}{T^2_\mathrm{F}}$ plus a positive constant, where $T_\mathrm{F}=E_\mathrm{F}/k_\mathrm{B}$ is the Fermi temperature~\cite{blundell_magnetism_2001}. Though $M_\mathrm{P}$ should therefore decrease with increasing $T$, it is defacto temperature independent since $T_\mathrm{F}$ is a large number compared to experimental temperatures.\\

% \textit{Temp graph. Going to need to consider the placement of Ef wrt the hybridization gap:}
% \begin{figure}[h]
%     \centering
%     \includegraphics[width=0.7\linewidth]{Heavy Fermion Band Hybridization.png}
%     \caption{temp graph of band hybridization in heavy fermion material.}
%     \label{fig:HFbandHybrid}
% \end{figure}

%$T=2$~K: $J=(0.7\pm0.5)$; $M_s=(286\pm5)$~(emu/cm$^3$). $T=5$~K: $J=(1.69\pm0.01)$; $M_s=(360.5\pm0.6)$~(emu/cm$^3$). $T=10$~K: $J=(2.195\pm0.005)$; $M_s=(412.7\pm0.4)$~(emu/cm$^3$). $T=15$~K: $J=(2.415\pm0.008)$; $M_s=(438.4\pm0.7)$~(emu/cm$^3$).


\noindent Shown in Fig.~\ref{fig:simple_brill_fits_with_pauli} are the results of fitting the $M(H)$ data with Eqn.~\ref{eq:Pauli_magnetization} plus Eqn.~\ref{eq:simp_brillouin}. Data at $T=70$~K was omitted from the fits due to being fully described by only Eqn.~\ref{eq:simp_brillouin}. The fit residuals are plotted above their respective fits. As was stated, these fits match the data better than the previous set of fits without $M_\mathrm{P}$. There is still an oscillation present in the residuals for $T=5$~K and $T=2$~K whose amplitude grows as $T$ is lowered. However, the oscillation amplitudes are much smaller than previously. There continues to be a vertical offset present in the residuals near $\mu_0H=0$~T for $T=15$~K and $T=10$~K. Although the $M(H)$ fit residuals are oscillatory, we did not find $M$ or $\chi=M/H$ to be periodic in $1/H$, regardless of whether the Pauli term was included. It is therefore unlikely these are quantum oscillations related to the Fermi surface. In Sec.~\ref{sec:hall_AHE_THE}, somewhat similar oscillations were observed in transverse resistivity measurements $\rho_{xy}(H)$, which we ascribed to a possible topological Hall effect.\\



\begin{figure}%[H]
    \centering
    \begin{subfigure}{0.49\linewidth} \includegraphics[width=\linewidth]{Figures_Ch4_Synth_Charac/Mag_and_Sus/Pauli Mag Fit/YbPdBi, Simple Brill Fit, T2K J1p49 Mp0p11 Ms1p57.pdf}
    \centering
    \end{subfigure}
     \begin{subfigure}{0.49\linewidth}\includegraphics[width=\linewidth]{Figures_Ch4_Synth_Charac/Mag_and_Sus/Pauli Mag Fit/YbPdBi, Simple Brill Fit, T5K J1p69 Mp0 Ms2p77.pdf} 
    \end{subfigure}
    \centering
    \begin{subfigure}{0.49\linewidth} \includegraphics[width=\linewidth]{Figures_Ch4_Synth_Charac/Mag_and_Sus/Pauli Mag Fit/YbPdBi, Simple Brill Fit, T10K J2p8 Mp0p06 Ms2p18.pdf}
    \centering
    \end{subfigure}
    \centering
     \begin{subfigure}{0.49\linewidth}\includegraphics[width=\linewidth]{Figures_Ch4_Synth_Charac/Mag_and_Sus/Pauli Mag Fit/YbPdBi, Simple Brill Fit, T15K J2p45 Mp0p0 Ms3p28.pdf} 
    \end{subfigure}
    \caption[Brillouin Fits of $M(H)$ with Pauli Term]{Best fits of magnetization $M(H)$ to Eqn.~\ref{eq:simp_brillouin}, plus a Pauli magnetization given by Eqn.~\ref{eq:Pauli_magnetization}. Residuals of each fit are plotted above the data (blue dots) and the best fit (orange lines).}
    \label{fig:simple_brill_fits_with_pauli}
\end{figure}

%$T=2$~K: $J=(0.7\pm0.5)$; $M_s=(286\pm5)$~(emu/cm$^3$); $M_p=(412.7\pm0.4)$~(emu/cm$^3$/T). $T=5$~K: $J=(1.69\pm0.01)$; $M_s=(360.5\pm0.6)$~(emu/cm$^3$); $M_p=(412.7\pm0.4)$~(emu/cm$^3$/T). $T=10$~K: $J=(2.195\pm0.005)$; $M_s=(412.7\pm0.4)$~(emu/cm$^3$); $M_p=(412.7\pm0.4)$~(emu/cm$^3$/T). $T=15$~K: $J=(2.415\pm0.008)$; $M_s=(438.4\pm0.7)$~(emu/cm$^3$); $M_p=(412.7\pm0.4)$~(emu/cm$^3$/T).
\noindent The parameters of both sets of fits are plotted in Fig.~\ref{fig:simple_brill_fits_params_plots} as a function of temperature. The parameters of fits without $M_\mathrm{P}$ are shown in orange and parameters for fits which include $M_\mathrm{P}$ are shown in blue. When it was included in the fit, $M_\mathrm{P}$ increases with decreasing temperature. This increase could be due to an increasing number of itinerant spins aligning with the magnetic field. Additional itinerant spins could arise from \textit{4f}-electron spins forming itinerant heavy fermion quasiparticles (HFQPs), in accordance with the two-fluid model of heavy fermion materials~\cite{yang_two-fluid_2016}. Similarly, both $M_s$ and $J$ of the Brillouin function decrease with decreasing temperature for all fits, which could be interpreted as the gradual onset of itinerant \textit{4f}-electron spins. Interestingly, at higher temperatures, the fitted parameter $J$ appears to possibly saturate at $J=5/2$ rather than the $J=7/2$ that is expected for Yb$^{3+}$. This was also seen in the magnetic entropy $S_{\mathrm{mag}}$ saturating at $R\ln6$.  Additional data sets for $T>15$~K are needed to confirm this result.\\

\begin{figure}[h]
    \centering
    \begin{subfigure}[t]{0.01\textwidth}
        \textbf{a)}
    \end{subfigure}    
    \centering
    \begin{subfigure}[t]{0.475\textwidth}
        \includegraphics[width=\linewidth, valign=t]{Figures_Ch4_Synth_Charac/Mag_and_Sus/Pauli Mag Fit/YbPdBi, Brill Fit Params Ms Pauli Comparison.pdf} 
    \end{subfigure}
    \centering
    \begin{subfigure}[t]{0.01\textwidth}
        \textbf{b)}
    \end{subfigure}    
    \centering
    \begin{subfigure}[t]{0.475\textwidth}
        \includegraphics[width=\linewidth, valign=t]{Figures_Ch4_Synth_Charac/Mag_and_Sus/Pauli Mag Fit/YbPdBi, Brill Fit Params J Pauli Comparison.pdf} 
    \end{subfigure}
    \centering
    \begin{subfigure}[t]{0.01\textwidth}
        \textbf{c)}
    \end{subfigure}    
    \centering
    \begin{subfigure}[t]{0.48\textwidth}
        \includegraphics[width=\linewidth, valign=t]{Figures_Ch4_Synth_Charac/Mag_and_Sus/Pauli Mag Fit/YbPdBi, Brill Fit Params Mp Pauli Comparison.pdf} 
    \end{subfigure}
    \caption[Comparison of $M(H)$ Best Fit Parameters]{Fit parameters of Eqn.~\ref{eq:simp_brillouin} for magnetization $M(H)$ both with (blue circles) and without (orange squares) the inclusion of the Pauli paramagnetism given by Eqn.~\ref{eq:Pauli_magnetization}. (a) Saturation magnetization $M_s$. (b) Total angular momentum $J$. (c) Pauli magnetization coefficient $M_\mathrm{P}$. Error bars represent the standard deviations of the fit parameters. The dashed lines are only guides for the eyes and do not signify real data.}\label{fig:simple_brill_fits_params_plots}
\end{figure}




%%%%%%%%%%%%%%%%%%%%%%%%%%%%%%
%%%%%%%%%%%%%%%%%%%%%%%%%%%%%%
%%%%%%%%%%%%%%%%%%%%%%%%%%%%%%
\newpage
\section{Elastic Neutron Powder Diffraction }\label{sec:neutrons}

Multiple measurements taken by LeBras \emph{et al.} at low-temperature have shown evidence for a magnetic phase transition at and below $T=1$~K~\cite{lebras_local_1995}. The magnetic specific heat shows a peak at $T=1$~K, $^{170}$Yb M\"ossbauer measurements evince the formation of a spontaneous Yb$^{3+}$ magnetic moment of approximately 1.25~$\mu_\mathrm{B}$ for $T<1$~K, and measurements of the magnetic susceptibility changes curvature for $T<1$~K. Their M\"ossbauer data also showed evidence for lowered local site symmetry of the Yb$^{3+}$ ion while in the paramagnetic phase. This was attributed to a static Jahn-Teller effect (JTE) consisting of a small displacement of the Yb$^{3+}$ ion off its standard position. However, they were unable to observe this displacement via elastic neutron diffraction taken at $T=1.5$~K and $T=130$~K. For isostructural YbPtBi, the order was shown to be fragile and short-range~\cite{ueland_fragile_2014, ueland_magnetic-field_2019}. We therefore investigate YbPdBi using powder neutron diffraction (PND) at temperatures down to $T=0.1$~K, well below the proposed magnetic phase transition temperature.\\

\noindent PND is comparable to powder X-ray diffraction, but in PND, we are examining the magnetic environment of the lattice in addition to its structure since neutrons possess spin-1/2. Single crystals of YbPdBi were grown using a Bi self-flux, powdered, and placed into a sample holder made of copper. Neutrons possessing a wavelength of $1.89$~\AA~were shone onto the powder, and the intensities of the reflected neutrons per angle of reflection were measured. This wavelength was chosen based on an estimated absolute scattering cross-section of $0.811$~cm$^{-1}$ and a $1/e$ length of $1.233$~cm for our sample. Neutron diffraction data was taken at $T=20$~K, $T=10$~K, and $T=2$~K (where the compound is a paramagnetic phase governed by AFM interactions), and at $T=0.1$~K (below the supposed $T=1$~K magnetic phase transition).  \\

% \noindent The neutron diffraction data was taken at the Paul Scherrer Institut (PSI) under the supervision of Dr. Denis Sheptyakov using the High-Resolution Powder Diffractometer for Thermal Neutrons (HRPT) on the Swiss Spallation Neutron Source (SINQ). Measurements were taken using a dilution refrigerator to regulate the temperature.\\

% \noindent Interestingly, we observed no doubling of peaks and only minimal change in intensity of the peaks. This suggests that the phase change associated with the $T=1$~K peak in specific heat is non-obvious.

% \textbf{YbPdBi peak positions: \[28.895, 33.458, 47.958, 56.883, 59.651, 70.068, 77.429, 79.831, 89.304, 96.385, 96.385, 108.479, 116.137, 118.799, 118.799, 130.298, 140.42, 144.304\]}

%%%%%%%%%%%%%%%%%%
%%%%%%%%%%%%%%%%%%
\subsection{Absence of Magnetic Ordering Down to $T=0.1$~K}

Shown in Fig.~\ref{fig:neutron_diff_raw}a is the neutron diffraction pattern for our sample of YbPdBi at $T=20$~K. Shown in Fig.~\ref{fig:neutron_diff_raw}b is the data taken at $T=0.1$~K, plotted similarly and showing no visible difference to the $T=20$~K data. In fact, we have omitted the $T=5$~K and $T=10~K$ data due to the similarity of all data sets. We performed a Rietveld refinement to fit the peaks of both YbPdBi (blue ticks) and of the copper (Cu) sample holder (red ticks). Based on the calculated difference (cyan) between the observed peaks (blue crosses) and calculated peaks (green line), we see that some peaks are unaccounted for near $2\theta\in\{43^\circ, 83^\circ,139^\circ, 150^\circ\}$. Some may be due to the unidentified impurity phase observed in the PXRD data in Sec.~\ref{sec:growth}. The Cu-peaks dominate the observed spectra and so will be removed from the data for the next steps of the analysis. We are unable to observe evidence of the JTE proposed by LeBras \emph{et al}.~\cite{lebras_local_1995}.\\


\noindent There is little significant difference between scattering datasets. Shown in Fig.~\ref{fig:neutron_diff_20K_2K_0p1K} is the difference between the data recorded at (a) $T=2$~K and at $T=0.1$~K and (b) $T=20$~K and at $T=0.1$~K. Blue ticks along the bottom indicate peak positions for YbPdBi, and red ticks indicate where data has been sliced out due to their proximity to the dominant Cu peaks. We have kept the vertical axis limits constant between plots to highlight the lack of obvious ordering. Due to the lack of a clear signal, we were unfortunately unable to determine the nature of the magnetic ordering, if indeed any such ordering does occur.
% If this this also the case for YbPdBi, our data herein would show broad shoulders centered on nuclear peaks.\\
% \noindent Any changes due to a magnetic ordering, the size of the changes ought to be of comparable size to the peaks caused by the ions of the lattice/ \textbf{nuetron scattering length. This is demonstrable by the nuclear form factor and magnetic form factor for Yb, which basically tells you how big the peak is expected to be. Andrea suggests "Neutron Data Booklet" by }. 
The only peaks we could possibly attribute to a magnetic ordering occur at $48^\circ$ and $89^\circ$, which are on the positions of the nuclear peaks of YbPdBi. All other peaks appear to depend on temperature or are near the Cu-related angles, which exhibited significant thermal effects.\\


\noindent Further investigations of the magnetic ground states are also recommended. Local structure measurements such as X-ray absorption fine structure (XAFS) spectroscopy and/or the neutron pair distribution function (PDF) could be used to probe if the local structure is really being distorted~\cite{booth_local_2001,booth_lattice_1998}. This would explain to explain the M\"ossbauer spectra seen previously in~\cite{lebras_local_1995}.\\




\begin{figure}[h]
  \centering
    \begin{subfigure}[t]{0.05\textwidth}
        \textbf{a)}
    \end{subfigure}    
    \centering
    \begin{subfigure}[t]{0.8\linewidth} 
    \includegraphics[width=\linewidth, valign=t]{Figures_Ch4_Synth_Charac/Neutron/YbPdBi Neutron 20K Fitted.pdf}
    \end{subfigure}\hfill\\
    \centering
    \begin{subfigure}[t]{0.05\textwidth}
        \textbf{b)}
    \end{subfigure}    
    \centering
    \begin{subfigure}[t]{0.8\linewidth} 
    \includegraphics[width=\linewidth, valign=t]{Figures_Ch4_Synth_Charac/Neutron/YbPdBi Neutron 0p1K Fitted.pdf}
    \end{subfigure}
    \caption[Fitted Neutron Diffraction Data at $T\in\{20\text{~K}, 0.1\text{~K}\}$]{Powder neutron diffraction data taken at (a) $T=20$~K and (b) $T=0.1$~K. A Rietveld refinement was performed using the structures of YbPdBi (blue ticks) and copper (red ticks). Intensities are plotted as the square root of their observed values.}
    \label{fig:neutron_diff_raw}
\end{figure}

\begin{figure}[h]
    \centering
    \includegraphics[width=\linewidth]{Figures_Ch4_Synth_Charac/Neutron/YbPdBi, Diff data 20K - 0p1K and 2K - 0p1K.pdf}
    \caption[Unexpected Absence of Magnetic Ordering]{Differences between neutron diffraction data. (Upper Figure) $T=0.1$~K minus $T=2$~K; (Lower Figure) $T=0.1$~K minus $T=20$~K. The similarity of the data sets is evident. 
    %The counts of the peaks in the difference data may be attributable to a thermal broadening of the peaks.
    }
    \label{fig:neutron_diff_20K_2K_0p1K}
\end{figure}

%%%%%%%%%%%%%%%%%
%%%%%%%%%%%%%%%%%
%%%%%%%%%%%%%%%%%
% \newpage
% \section{Characterization Discussion}
% TODO:
% \begin{enumerate}
%     \item To make discussion easier, make a phase diagram with $T^*$, $H_{cross}$, $T=1$~K/$\mu_0H=0$~T peak possibly being shifted up to $T=6K$ at $\mu_0H=9$~T in Schottky data, maybe temperature of peak values of C/T Schottky data? 
% \end{enumerate}
% Summarize potential crossover fields and temperatures, and extract the Wilson ratio for YbPdBi.

% What  kind of scattering dominates where, why is it odd that Kondo scattering i occurring where chirality is found (massive vs massless)

% Discussion of susceptibility and maybe a valence phase change that's not visible in the neutron data

% Discuss neutron stuff too.
% \subsection{Heavy Fermion Phase T-dependence}
% TODO:
% \begin{enumerate}
%     \item Discuss what is going on with the magnetic moments as T is lowered
%     \item Reiterate $H_{cross}$, THE, BrillFit osc, and mention what is seen in AMRO
%     \item compare to the THE oscillations with the residuals of the brillouin fits.
%     \begin{enumerate}
%         \item the two-linear analysis of $\rho_{xy}$ analysis assumes $R_\mathrm{H}$ is very dependent on $T$
%         \item the THE analysis of $\rho_{xy}$ assumes $R_\mathrm{H}$ is approximately constant below $T\leq 120$~K
%         \item The THE analysis here seems validated by the Nature paper and related ones
%         \item but the two-lienar results above are hard to argue with since they work well with the data
%         \item Could this stem from using different growth recipes? Attibutable to different impurities? Scattering off Pb- vs Bi-impurities? Bi impurity phase?
%     \end{enumerate}
% \end{enumerate}



% \subsection{CEF Scheme}
% \begin{enumerate}
%     \item Compare to YbBiPt, CEF terminology may not be appropriate
%     \item Canfield \emph{et al}. 1994
%     \begin{enumerate}
%         \item similar thing seen for uranium compounds [28, 19]
%     \end{enumerate}
%     \item Amato \emph{et al}. 
%     \item Robinson \emph{et al}. 1995  Low energy
%     \item Thompson 1993
%     \item J=5/2
%     \begin{enumerate}
%         \item Magnetic entropy
%         \item Brillouin fits
%         \item Schottky?
%     \end{enumerate}
% \end{enumerate}



% \subsection{Ground state discussion}
% \textbf{Could it be a spin glass ground state?}
% \begin{enumerate}
%     \item Need to be careful we're not looking to make the data fit the conclusion.
%     \item There was a muSR experiment on YbBiPt that postulated such a state for that compound, but they seem to attribute it to having multiple magnetic domains, somehow. Don't give a possible discussion on structure of it wrt unit cell. "Low-temperature magnetism in YbBiPt"
%     \item The DC susceptibility change in curvature at T=1K ~\cite{lebras_local_1995} be representative of such a g.s.
%     \begin{enumerate}
%         \item See "Spin-Glass Behavior of Mechanically Milled Crystalline GdAlz", not sure how large an difference powdering makes, though.
%         \item Needs to show a maximum $\chi$ corresponding to a "spin glass freezing temperature".
%         \item Would need FC and ZFC AC susceptibility measurements to corroborate this.
%     \end{enumerate}
%     \item But what about the M(H)?
%     \item and specific heat curves?
%     \begin{enumerate}
%       \item See: "Specific Heat and Magnetocaloric Effect Studies in Multiferroic YMnO3" For similar looking peak (ish) 
%       \item Also Fig2 of "Frustration(s) and the Ice Rule: From Natural Materials to the Deliberate Design of Exotic Behaviors"
%     \end{enumerate}
%     \item and neutron data
%     \begin{enumerate}
%         \item Does the powdering process randomize the random orientations such that they average out? This would be a bulk-ish measurement, but not necessarily contradict the bulk vs local argument re: Kondo screening of the moments
%         \item "Correlations and dynamics of spins in an XY-like spin-glass (Ni0.4Mn0.6)TiO3 single crystal system" has comparable neutron data, fig. 2c, only one single peak, that is broad, and at T=1.6.  Their dc susceptibility in fig.2a matches LeBras, kinda, at T=9.1K
%         \item May need "strong magnetic diffuse scattering at finite Q which are not coincident with nuclear Bragg peaks. Find citation. I think we have those, check the Python code.
%     \end{enumerate}
% \end{enumerate} 