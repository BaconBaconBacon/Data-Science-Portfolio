\abstract

 %(150 a 250 mots) (1 page)

\noindent We explore the Fermi surface of single crystals of YbPdBi, a cubic noncentrosymmetric heavy fermion material, through angle-resolved magnetoresistance oscillations (AMRO). We characterize YbPdBi's magnetotransport, specific heat, neutron diffraction spectra, and magnetization. We observe an enhanced Pauli paramagnetism at low temperatures, a Schottky-like anomaly with a strong dependence on the magnetic field, and a topological Hall effect in the magnetotransport. We probe the topology of YbPdBi and find evidence of Weyl semimetal behaviour in the magnetotransport. We found signs of a chiral anomaly in the magnetotransport, independent of the angle between the current and magnetic field. The observed chiral coefficients exhibit the expected temperature dependence characterized by $\mu = (2.58\pm0.07)$~meV and a $v_\mathrm{F}^3\tau_v = (48\pm1)$~m$^3$~s$^{-2}$. The transverse magnetoresistivity $\rho_{xy}(H)$ presents two linear regimes of different slopes separated by a gradual crossover, without saturation of the magnetization. This crossover's temperature dependence suggests it coincides with a reported $T=1$~K peak in specific heat. Using neutron diffraction, we find no evidence of magnetic ordering down to $T=0.1$~K. An unexpected change in the symmetry of the AMRO measurements is observed, changing from two-fold to four-fold as the magnetic field was increased. This suggests that the Fermi surface of YbPdBi is changing with magnetic field strength. \\



%(max. of 10, not words in the title)
{\bfseries Keywords~: Heavy fermions, Weyl semimetal, half-Heusler, single crystals, topology, Pauli paramagnetism, noncentrosymmetric, neutron diffraction, chiral anomaly, angle-resolved magnetoresistance oscillations}
