\resume
\selectlanguage{french}

 %(150 a 250 mots) (1 page)

\noindent Nous explorons la surface de Fermi des monocristaux de YbPdBi, un matériau cubique à fermions lourds non centrosymétriques, à travers des oscillations de magnétorésistance résolues en angle (AMRO). Nous caractérisons le magnétotransport, la chaleur spécifique, les spectres de diffraction des neutrons et l'aimantation de YbPdBi. Nous observons un paramagnétisme de Pauli renforcé à basse température, une anomalie de type Schottky avec une forte dépendance au champ magnétique et un effet Hall topologique dans le magnétotransport. Nous sondons la topologie de YbPdBi et trouvons des preuves du comportement de Weyl semimetal dans le magnétotransport. Nous avons trouvé des signes d'une anomalie chirale dans le magnétotransport, indépendamment de l'angle entre le courant et le champ magnétique. Les coefficients chiraux observés présentent la dépendance attendue à la température caractérisée par $\mu = (2,58\pm0,07)$~meV et  $v_\mathrm{F}^3\tau_v = (48\pm1)$~m$^3$~s$^{-2}$. La magnétorésistivité transversale $\rho_{xy}(H)$ présente deux régimes linéaires de pentes différentes séparés par un croisement progressif, sans saturation de l'aimantation. La dépendance à la température de ce croisement suggère qu'il coïncide avec un pic $T=1$~K signalé en chaleur spécifique. En utilisant la diffraction neutronique, nous ne trouvons aucune preuve d'ordre magnétique jusqu'à $T=0,1$~K. Un changement inattendu de la symétrie des mesures AMRO est observé, passant de deux à quatre fois à mesure que le champ magnétique augmentait. Cela suggère que la surface de Fermi de YbPdBi change avec l'intensité du champ magnétique.\\

{\bfseries Mots clés~: Fermions lourds, semi-métal de Weyl, demi-Heusler, monocristaux, topologie, paramagnétisme de Pauli, diffraction des neutrons, anomalie chirale,
oscillations de magnétorésistance résolues en angle} 

