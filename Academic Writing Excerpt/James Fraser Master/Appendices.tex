\debutannexes
% \addtocontents{lot}{\protect\setcounter{tocdepth}{0}}  % Use this to remove tables from List of Tables (lot)
% \addtocontents{lof}{\protect\setcounter{tocdepth}{0}}  % Use this to remove figures from List of Figures (lof)

\annexe{Comparison of Pb- and Bi-Flux Recipes}\label{app:Pb_vs_Bi_flux}

\begin{figure}[H]
    \centering
      \includegraphics[width=\textwidth]{Figures_Appendices/Pb vs Bi/HC3 Cv_o_T of T anomaly.jpg}
    \caption[Specific Heat Anomaly for YbPdBi Grown in Bi-flux]{Specific heat anomaly for samples grown using the Bi-flux recipe. Note the presence of the anomaly at the lower temperatures, which is absent from the specific heat data for the Bi-flux samples. Low-temperature specific heat anomaly near $T=12$~K for samples grown using the Bi-flux recipe.}
    \label{fig:Bi_recipe_transition_loT_C_o_T}
\end{figure}

Shown in Fig.~\ref{fig:Bi_recipe_transition_loT_C_o_T} is a peak near $T=12$~K in the $\frac{C_p}{T}$ data for a sample grown using the Bi-flux recipe. This peak was found in other sample batches grown using the Bi-flux method, but it was not found in the specific heat data for the Pb-flux samples, as shown in Fig.~\ref{fig:sample_specific_heat}). The cause of this peak may be the impurity phase seen in the YbPdBi PXRD refinement in Fig.~\ref{fig:sample_PXRD_bi_flux}. \\


\noindent Unfortunately, not only does this peak happen within the range of $T$ for which we took AMRO data, but this anomaly was discovered only after the majority of our measurements had already been performed using Bi-flux grown samples. Therefore, it was necessary to compare experiments performed with samples from either method for significant deviations between results. We can eliminate the impurity transition as a source of the AMRO symmetry change. \\

\begin{figure}[H]
    \centering
    \includegraphics[width=\linewidth]{Figures_Appendices/Pb vs Bi/YbPdBi, ACTRot6, JF011, Symmetry change in Pb based recipes.png}
    \caption[Pb-Flux AMRO Data at $\mu_0H=7$~T for $T\in\{2\text{~K}, 10\text{~K}\}$]{AMRO data taken at $\mu_0H=7$~T using a sample grown in Pb-flux. The sample was mounted in the $\perp$ geometry. The expected two-fold symmetry is seen in the $T=10$~K data on the right, while the four-fold symmetry contribution still occurs at $T=2$~K. 
    % Spikes seen in the data are likely attributable to the conductive silver epoxy used to attach the contact wires before we switched to spot welding the wires on.
    }
    \label{fig:Pb_symmetry_change_app}
\end{figure}

\noindent Shown in Fig.~\ref{fig:Pb_symmetry_change_app} is  AMRO data in the $\perp$ geometry, taken at $T=2$~K and $T=10$~K at $\mu_0H=7$~T using a sample grown using the Pb-flux method. The conductive wires were glued on using conductive silver epoxy, whereas the AMRO data in Ch.~\ref{CH:AMRO_main} had wires that were spot welded to the sample.\\

\noindent Spikes at certain rotational angles are visible. These are reminiscent of singular angle-resolve magnetoresistance (SAMR), which is associated with the creation and destruction of magnetic domains across which magnetotransport is occurring. These spikes are generally absent from the AMRO data sets of Sample Puck 11 $\perp$ and Sample Puck 13 $\parallel$. However, possible signs of these spikes can be seen in the $\mu_0H=7$~T AMRO data for Sample Puck 11 $\perp$ at $T=10$~K.\\

\noindent Most importantly, there is still a visible competition between two-fold and four-fold symmetry. Though it exhibits a different dependence on $T$ and $H$ than what was seen in Ch.~\ref{CH:AMRO_main}, the change from a dominant two-fold symmetry to a four-fold symmetry still occurs. We can therefore conclude that the impurity phase is not the source of the change in symmetry for the Bi-flux samples shown in Ch.~\ref{CH:AMRO_main}. \\

\noindent The AMRO data sets in Fig.~\ref{fig:Pb_symmetry_change_app} have mean resistivities of $\bar\rho(\mu_0H =7~\text{T}, T=2~\text{K})\approx 1.86$~mohm-cm and $\bar\rho(\mu_0H =7~\text{T}, T=10~\text{K})\approx3.84$~mohm-cm. These Pb-flux values are orders of magnitude larger than their corresponding values seen in Bi-flux grown Sample Puck 11 $\perp$~: $\bar\rho(\mu_0H =7~\text{T}, T=2~\text{K})\approx0.06$~mohm-cm and $\bar\rho(\mu_0H =7~\text{T}, T=2~\text{K})\approx0.26$~mohm-cm. An additional impurity phase should result in an increase in resistivity rather than a decrease. \\ %We had difficulties with the conductive silver epoxy, which was introducing noise to the signals. It's possible a layer of oxidation formed between the silver epoxy and the sample, which caused this increase in resistivity. \\

\noindent There is no significant change to the behaviour of the magnetotransport visible in the AMRO data. We can conclude that there is minimal impact on magnetotransport experiments.  \\

% \noindent Note that we attribute the sharp peaks in resistivity seen in Fig.~\ref{fig;Pb_symmetry_change_app} (particularly near $\theta=60^\circ$) to the silver epoxy used to affix the gold wires to the sample. We circumvented these peaks by switching from silver epoxy to spot-welding the wires to the sample.\\

\begin{figure}[H]
    \centering
    \includegraphics{Figures_Appendices/Pb vs Bi/YbPdBi, JF027 Bi Flux Thermoresistivity.pdf}
    \caption[Bi-Flux Recipe Resistivity Below $T=300$~K]{Resistivity curve below $T=300$~K for a sample of YbPdBi grown using the Bi-flux method.}
    \label{fig:thermoresistivity_Bi_flux_app}
\end{figure}

\noindent Shown in Fig.~\ref{fig:thermoresistivity_Bi_flux_app} is a resistivity curve for YbPdBi samples grown in a Bi-flux. Within the limits of the $T$ resolution of the data, it appears to match the $T$-dependence shown in Fig.~\ref{fig:thermo_res_0_field} of the Pb-flux grown sample. The impurity phase does not appear to dramatically affect transport measurements in $\mu_0H=0$~T. \\

\noindent Unfortunately, we have no magnetization data for samples grown with the Bi-flux method. We cannot assess the influence of the impurity phase on the magnetization data. This would depend on the nature of the impurity phase and the relative abundance of the magnetic Yb atoms within the impurity phase.

% Finally, Its influence on the neutron scattering data is a moot point owing to the fact we saw no change in the neutron scattering intensity between $T=0.1$~K and $T=20$~K.





%%%%%%%%%%%%%%%%%%%%%%%%%%%%%
% \annexe{Characterization Misc.}\label{app:sec_4_graphs}

% %\section{Resistivity Measurements}
% \annexe{Magnetoresistance Data}\label{app:MR_plots}

% % \textbf{why is this here???}\\


% \begin{figure}[H]
%     \centering
%     \begin{subfigure}{0.7\linewidth} \includegraphics[width=\linewidth]{Figures_Ch4_Synth_Charac/Resistivity/YbPdBi, pct diff MR actrot12.pdf}
%     \end{subfigure}
%      \begin{subfigure}{0.78\linewidth}\includegraphics[width=\linewidth]{Figures_Ch4_Synth_Charac/Resistivity/YbPdBi, pct diff MR ZOOM actrot12.pdf} 
%     \end{subfigure}
%     \caption[ $\rho_{xx}(H)$ of YbPdBi with $H\perp I$]{Magnetoresistance for YbPdBi with $H$ orthogonal to $I$, $\Delta\rho/\rho_0=\frac{\rho(H)-\rho(0)}{\rho(0)}$. (Top) Plotted between $\mu_0H=\pm9$~T. (Bottom) Replotted for $\mu_0H=\pm4$~T. Note the upwards-curving behaviour present near and below $|\mu_0H|=4$~T, which is attributable to weak anti-localization (WAL) effects. The dashed lines are guides for the eyes and do not represent data points. }\label{fig:MR_actrot12_app}
% \end{figure}


%%%%%%%%%%%%%%%%%%%%%%%%%%%%%%%%%%%%%
%%%%%%%%%%%%%%%%%%%%%%%%%%%%%%%%%%%%%
%%%%%%%%%%%%%%%%%%%%%%%%%%%%%%%%%%%%%
% %\section{Specific Heat Measurements}
%%%%%%%%%%%%%%%%%%%%%%%%%%%%%%%%%%%%%
%%%%%%%%%%%%%%%%%%%%%%%%%%%%%%%%%%%%%
\annexe{Specific Heat Dataset Concatenation}\label{app:phonon_Cv_calc}
% \section{Low-T Data Concatenation and Phonon Extrapolation}\label{app:lebras_concat}


\begin{figure}[H]
    \centering
    \includegraphics[width=\linewidth]{Figures_Appendices/specific heat app/YbPdBi, Specific Heat, Lebras Concat.pdf}
    \caption[Low-Temperature Specific Heat, $C_p(T)$]{Low-temperature specific heat for YbPdBi taken by (blue)  us and  (green) digitized from LeBras \emph{et al}.~\cite{lebras_local_1995}. Though there is a disjoint transition near $T=3.5K$ between the data sets, it is also present in~\cite{lebras_local_1995}.}\label{fig:lebras_concat}
\end{figure}


Shown in Fig.~\ref{fig:lebras_concat} is the concatenation of our YbPdBi specific heat data (blue) with data digitized from LeBras \emph{et al}.~\cite{lebras_local_1995}. Though it appears there is a sharp drop between their data and ours, such a drop is also observed in this temperature range in their specific heat data. We can conclude that such a concatenation of data accurately represents the specific heat of YbPdBi.\\\\

%%%%%%%%%%%%%%%%%%%%%%%%%%%%%%%%%%%%%
%%%%%%%%%%%%%%%%%%%%%%%%%%%%%%%%%%%%%
\annexe{Approximating Phonon Specific Heat for YbPdBi using YPdBi}\label{app:YPdBi_Phonons}



To approximate the phonon contribution $C_{\mathrm{ph}}(T)$ to YbPdBi's specific heat, we measured the specific heat of the isostructural and non-magnetic YPdBi. Shown in Fig.~\ref{fig:YPdBi_spec_heat}a is the specific heat of YPdBi (orange), which we have extended to sub-Kelvin temperatures (blue) by fitting a straight line as shown in Fig.~\ref{fig:YPdBi_spec_heat}b. It was necessary to fix the intercept of this fit to $\gamma=0$ since the fit was converging to a non-physical $\gamma<0$.The slope of the resulting linear fit was $\beta = (0.000691\pm 0.000009)$~(J/mol-K/K), implying a Debye temperature of $\Theta_D\approx144$~K at low temperature.\\

%\section{Phonon Scaling using $\theta_D(T)$}\label{app:phonon_scaling_debye}

\noindent Making use of the newly extrapolated specific heat data, we were able to calculate a temperature-dependent Debye temperature $\Theta_{D}(T)$ for YPdBi. This was done for each data point by solving for $\Theta_{D}(T)$ within the Debye model function~\cite{ashcroft_solid_1976}~:
\begin{equation}\label{eq:Debye_eqn_Spec_heat}
    C_p(T) = 9rN_ak_b\left(\frac{T}{\Theta_{D}(T)}\right)^3 \int_0^{\Theta_{D}(T)/T}\frac{y^4e^y}{(e^y-1)^2}dy
\end{equation}
where $r=3$ is the number of atoms per molecule, $N_a$ is Avogadro's number, $k_b$ is the Boltzmann constant. It was then necessary to scale $\Theta_{D}(T)$ up based on an approximate relationship between the mass of YPdBi and YbPdBi~\cite{ashcroft_solid_1976}~:
\begin{equation}\label{eq:Debye_mass_scaling}
    \Theta_{D}^{\mathrm{YbPdBi}}(T)=\Theta_{D}^{\mathrm{YPdBi}}(T)*\sqrt{\frac{m_\mathrm{YPdBi}}{m_{\mathrm{YbPdBi}}}}
\end{equation}
\noindent where $m_x$ is the mass of a single molecule of $x$. Shown in Fig.~\ref{fig:YPdBi_Debye_Shifting}a is the original (orange) and the shifted (blue) Debye temperature curves. The disjointed jump for $T<5$~K seen in Fig.~\ref{fig:YPdBi_Debye_Shifting}a corresponds to the extrapolated YPdBi data. We found this disjointedness was improved by allowing $\gamma $ to be negative, but this was rejected due to non-physicality. \\


\begin{figure}[H]
    \centering
    \begin{subfigure}[t]{0.19\textwidth}
        \textbf{a)}
    \end{subfigure}    
    \centering
    \begin{subfigure}[t]{0.8\textwidth}  
        \includegraphics[width=\linewidth, valign=t]{Figures_Appendices/specific heat app/YPdBi, JF045, Phonon Specific Heat.pdf}
    \end{subfigure}
    \centering
    \begin{subfigure}[t]{0.19\textwidth}
        \textbf{b)}
    \end{subfigure}    
    \centering
    \begin{subfigure}[t]{0.8\textwidth}  
        \includegraphics[width=\linewidth, valign=t]{Figures_Appendices/specific heat app/YPdBi, JF045, Low T Linear Fit, lmfit Result.pdf}
    \end{subfigure}
    \caption[YPdBi Specific Heat Data $C_p(T)$]{(a) YPdBi Specific heat, extended to very low temperature via best fit. (b) Low-temperature linear fit of specific heat for YPdBi between $T=5$~K and $T=12$~K to Eqn.~\ref{eq:low_T_linear_Cv}.}
    \label{fig:YPdBi_spec_heat}
\end{figure}


\begin{figure}[H]
    \centering
    \begin{subfigure}[t]{0.1\textwidth}
        \textbf{a)}
    \end{subfigure}    
    \centering
    \begin{subfigure}[t]{0.85\textwidth}  
        \includegraphics[width=\linewidth, valign=t]{Figures_Appendices/specific heat app/YPdBi, JF045, Shifted and OG Debye Temp.pdf}
    \end{subfigure}
    \centering
    \begin{subfigure}[t]{0.1\textwidth}
        \textbf{b)}
    \end{subfigure}    
    \centering
    \begin{subfigure}[t]{0.85\textwidth}  
        \includegraphics[width=\linewidth, valign=t]{Figures_Appendices/specific heat app/YPdBi, JF045, OG and Shifted Curves.pdf}
    \end{subfigure}
    \caption[Shifting $C_{\mathrm{ph}(T)}$ using $\Theta_D(T)$]{(a) Temperature-dependent Debye temperature for YPdBi showing the original curve and the one shifted using Eqn.~\ref{eq:Debye_mass_scaling}. (b) YPdBi phonon specific heat data (orange) scaled up such that we have approximated the phonon specific heat of YbPdBi (blue).}
    \label{fig:YPdBi_Debye_Shifting}
\end{figure}



\noindent Shown in Fig.~\ref{fig:YPdBi_Debye_Shifting}b is the original (orange) and shifted (blue) specific heat of YPdBi, as calculated using the temperature-dependent Debye temperature. We treat this shifted data as the phonon contribution to the specific heat of YbPdBi, and subtract it from the data measured in Sec.~\ref{sec:YbPdBi_spec_heat}, which gives the magnetic specific heat shown in Fig.~\ref{fig:Phonon_removed}.\\

\begin{figure}[H]
    \centering
    \includegraphics[width=0.9\linewidth]{Figures_Ch4_Synth_Charac/Specific Heat/YbPdBi, magnetic specific heat.pdf}
    \caption[$C_p(T)$, $C_{\mathrm{ph}}(T)$, and $C_{\mathrm{mag}}(T)$ for $0.5\text{~K}\leq T \leq300\text{~K}$]{Molar specific heat ($C_p$) data for YbPdBi separated into phonon and non-phonon parts. (Blue) Measured $C_p$ for YbPdBi; (Orange) Phonon $C_{\mathrm{ph}}$ determined from measurements on YPdBi; (Green) The electronic and magnetic contribution to $C_p$. (Inset) The same data on a logarithmic x-axis shows that the non-phonon contribution begins to increase as $T$ goes below the heavy fermion coherence temperature $T^*=33$~K (vertical black line). Phonon data for $T<4$~K was extrapolated using a linear fit of YPdBi $\frac{C_p}{T}$ measurements below $T=14$~K. (Crosses) Data taken digitized from LeBras \emph{et al}.~\cite{lebras_local_1995}, (Dots) our data.}
    \label{fig:Phonon_removed}
\end{figure}


%%%%%%%%%%%%%%%%%%%%%%%%%%%
%%%%%%%%%%%%%%%%%%%%%%%%%%%
\newpage
% \annexe{Hall Measurements}\label{app:hall_measurements}
% %\section{Hall Measurements}
\annexe{Raw Hall Data}\label{app:hall_measurements}
\begin{figure}[H]
    \centering
    \includegraphics[width=0.8\linewidth]{Figures_Ch4_Synth_Charac/Hall Data/YbPdBi, Hall Coeff raw ch2.pdf}
    \caption[Raw Hall Coefficient $R_H(H)$ for $2~\text{K}\leq T\leq  120~\text{K}$]{Raw, un-antisymmetrized Hall Coefficient measurements for YbPdBi taken using a 4-point contact setup, shown with a logarithmic x-axis to highlight the $T$ dependence of the curves. As $T$ is lowered, the data maximum and minimum shift closer to zero. Connecting lines are shown simply to help guide the eye and do not represent data points.}
    \label{fig:raw_Hall}
\end{figure}



\begin{figure}[H]
    \centering
    \includegraphics{Figures_Appendices/Hall app/YbPdBi, Hall resistivity ch2, RAW APPENDIX.pdf}
    \caption[Raw $\rho_{xy}(H)$ for $2~\text{K}\leq T \leq 120~\text{K}$]{Raw, un-antisymmetrized transverse resistivity measurements performed in a 4-point Hall setup. The resistivity is dominated by a contribution from the longitudinal resistance due to the nature of the 4-point setup.}
    \label{fig:raw_Rxy_data_app}
\end{figure}


%%%%%%%%%%%%%%%%%%%%%%%%%%%
%%%%%%%%%%%%%%%%%%%%%%%%%%%
\annexe{Linear Fits to Transverse Resistivity $\rho_{xy}(H)$}\label{app:linear_fits_rhoxy}
% \subsubsection{Full Range of $H$}
\begin{figure}[H]
    \centering
    \begin{subfigure}{0.45\linewidth} \includegraphics[width=\linewidth]{Figures_Appendices/Hall app/YbPdBi, JF042, rhoxy lin fits 120K APPENDIX.pdf}
    \end{subfigure}
    \centering
     \begin{subfigure}{0.45\linewidth}\includegraphics[width=\linewidth]{Figures_Appendices/Hall app/YbPdBi, JF042, rhoxy lin fits 60K APPENDIX.pdf}
    \end{subfigure}
    \centering
     \begin{subfigure}{0.45\linewidth}\includegraphics[width=\linewidth]{Figures_Appendices/Hall app/YbPdBi, JF042, rhoxy lin fits 40K APPENDIX.pdf}
    \end{subfigure}
    \caption[Detailed Plots of Linear Fits of $\rho_{xy}(H)$ for $40\text{~K}\leq T \leq 120~\text{K}$]{ Full range linear fits of transverse resistivity $\rho_{xy}$ from $\mu_0H=0$~T to $\mu_0H=9$~T for $T\in\{ 40~\text{K}, 60~\text{K}, 120~\text{K}\}$.}\label{fig:rho_xy_hi_T_lin_fits_App}
\end{figure}
% \subsubsection{Low-$H$ Range and High-$H$ Range}\label{fig:rho_xy_lo_T_lin_fits_App}


\begin{figure}[H]
    \centering
     \begin{subfigure}{0.49\linewidth}
     \includegraphics[width=\linewidth]{Figures_Appendices/Hall app/YbPdBi, JF042, rhoxy lin fits 30K APPENDIX.pdf}
    \end{subfigure}
    \centering
    \begin{subfigure}{0.49\linewidth}
    \includegraphics[width=\linewidth]{Figures_Appendices/Hall app/YbPdBi, JF042, rhoxy lin fits 30K loT hiH APPENDIX.pdf}
    \end{subfigure}
    \caption[Detailed Linear Fits of $\rho_{xy}(H)$ at $T=30$~K]{Linear fits of $\rho_{xy}$ at $T=30$~K performed at both high and low ranges of field strengths.}\label{fig:rho_xy_lo_T_lin_fits_App20K}
\end{figure}
\begin{figure}[H]
    \centering
    \begin{subfigure}{0.49\linewidth}
    \includegraphics[width=\linewidth]{Figures_Appendices/Hall app/YbPdBi, JF042, rhoxy lin fits 15K APPENDIX.pdf}
    \end{subfigure}
    \centering
    \begin{subfigure}{0.49\linewidth}
    \includegraphics[width=\linewidth]{Figures_Appendices/Hall app/YbPdBi, JF042, rhoxy lin fits 15K loT hiH APPENDIX.pdf}
    \end{subfigure}
    \caption[Detailed Linear Fits of $\rho_{xy}(H)$ at $T=15$~K]{Linear fits of $\rho_{xy}$ at $T=15$~K performed at both high and low ranges of field strengths.}\label{fig:rho_xy_lo_T_lin_fits_App15K}
\end{figure}


\begin{figure}[H]
    \centering
    \begin{subfigure}{0.49\linewidth}
    \includegraphics[width=\linewidth]{Figures_Appendices/Hall app/YbPdBi, JF042, rhoxy lin fits 10K APPENDIX.pdf}
    \end{subfigure}
    \centering
    \begin{subfigure}{0.49\linewidth}
    \includegraphics[width=\linewidth]{Figures_Appendices/Hall app/YbPdBi, JF042, rhoxy lin fits 10K loT hiH APPENDIX.pdf}
    \end{subfigure}
    \caption[Detailed Linear Fits of $\rho_{xy}(H)$ at $T=10$~K]{Linear fits of $\rho_{xy}$ at $T=10$~K performed at both high and low ranges of field strengths.}\label{fig:rho_xy_lo_T_lin_fits_App10K}
\end{figure}



\begin{figure}[H]
    \centering
    \begin{subfigure}{0.49\linewidth}
    \includegraphics[width=\linewidth]{Figures_Appendices/Hall app/YbPdBi, JF042, rhoxy lin fits 5K APPENDIX.pdf}
    \end{subfigure}
    \centering
    \begin{subfigure}{0.49\linewidth}
    \includegraphics[width=\linewidth]{Figures_Appendices/Hall app/YbPdBi, JF042, rhoxy lin fits 5K loT hiH APPENDIX.pdf}
    \end{subfigure}
    \caption[Detailed Linear Fits of $\rho_{xy}(H)$ at $T=5$~K]{Linear fits of $\rho_{xy}$ at $T=5$~K performed at both high and low ranges of field strengths.}\label{fig:rho_xy_lo_T_lin_fits_App5K}
\end{figure}



\begin{figure}[H]
    \centering
    \begin{subfigure}{0.49\linewidth}
    \includegraphics[width=\linewidth]{Figures_Appendices/Hall app/YbPdBi, JF042, rhoxy lin fits 2K APPENDIX.pdf}
    \end{subfigure}
    \centering
    \begin{subfigure}{0.49\linewidth}
    \includegraphics[width=\linewidth]{Figures_Appendices/Hall app/YbPdBi, JF042, rhoxy lin fits 2K loT hiH APPENDIX.pdf}
    \end{subfigure}
    \caption[Detailed Linear Fits of $\rho_{xy}(H)$ at $T=2$~K]{Linear fits of $\rho_{xy}$ at $T=2$~K performed at both high and low ranges of field strengths.}\label{fig:rho_xy_lo_T_lin_fits_App2K}
\end{figure}



\begin{table}[H]
\centering
\begin{tabular}{c|c|c|c|c|c}
    T (K) & $R_\mathrm{H}^{F/L}$ $\left(\frac{\mu\Omega-cm}{T}\right)$ & $b^{F/L}$ $\left(\mu\Omega-cm\right)$ & $R_\mathrm{H}^{H}$ $\left(\frac{\mu\Omega-cm}{T}\right)$ &$b^{H}$ ($\mu\Omega$-cm) & $H_{cross}$ (T)  \\\hline
    120 & 2.94 $\pm$ 0.01  & -0.21$\pm$ 0.07   & N/A & N/A & N/A \\
    60  & 3.81 $\pm$ 0.01  & -0.18$\pm$ 0.07   & N/A & N/A & N/A \\
    40  & 4.54 $\pm$ 0.02  & -0.01$\pm$ 0.09   & N/A & N/A & N/A \\
    30  & 5.20 $\pm$ 0.01  & 0.24 $\pm$ 0.06   & 4.7  $\pm$ 0.1   &  3.4 $\pm$  0.9 & 7   $\pm$2   \\
    15  & 7.33 $\pm$ 0.06  & 0.27 $\pm$ 0.09   & 3.65 $\pm$ 0.09  & 16.7 $\pm$  0.7 & 4.5 $\pm$0.2  \\
    10  & 7.8  $\pm$  0.1  & 0.4 $\pm$ 0.1     & 2.46 $\pm$ 0.05  & 18.1 $\pm$  0.4 & 3.3 $\pm$0.1  \\
    5   & 6.8  $\pm$  0.3  & 0.9 $\pm$ 0.3     & 2.67 $\pm$ 0.03  &  9.0 $\pm$  0.2 & 2.0 $\pm$0.2  \\
    2   & 6.5  $\pm$  0.4  & 0.4 $\pm$ 0.2     & 3.09 $\pm$ 0.03  &  3.2 $\pm$  0.2 & 0.8 $\pm$0.1  \\
\end{tabular}
\caption[Parameters of Linear Fits of $\rho_{xy}(H)$ at Low-$T$]{Linear fit parameters for low temperature $\rho_{xy}$ data, shown in Fig.~\ref{fig:low_T_rhoxy_lin_fits} for high and low field strengths. $m^i$ and $b^i$ are the slopes and intercepts, respectively, for full range ($F$), low-$H$ ($L$), and high-$H$ ($H$) linear fits. At high-$T$, there was only one linear regime of $H$ that could be fitted. The uncertainty for each fit parameter is the standard deviation obtained by the fitting function; the uncertainty for $H_{cross}$ was derived by the propagation of these standard deviations.}\label{tab:rho_xy_lo_T_lin_fits}
\end{table}




%%%%%%%%%%%%%%%%%%%%%%%%%%
%%%%%%%%%%%%%%%%%%%%%%%%%%
%%%%%%%%%%%%%%%%%%%%%%%%%%
\annexe{Hall Effect Fit Parameters}\label{app:AHE_fit_params}
\begin{figure}[H]
    \centering
    \begin{subfigure}[t]{0.15\textwidth}
        \textbf{a)}
    \end{subfigure}    
    \centering
    \begin{subfigure}[t]{0.8\textwidth}  
        \includegraphics[width=\linewidth, valign=t]{Figures_Appendices/Hall app/YbPdBi, AHE gamma Fit Parameter Plot.pdf}
    \end{subfigure}
    \begin{subfigure}[t]{0.15\textwidth}
        \textbf{b)}
    \end{subfigure}    
    \centering
    \begin{subfigure}[t]{0.8\textwidth}  
        \includegraphics[width=\linewidth, valign=t]{Figures_Appendices/Hall app/YbPdBi, AHE b Fit Parameter Plot.pdf}
    \end{subfigure}
    \caption[Fit Parameters of $\rho_{xy}(T) - \rho_{xy}(T=120\text{~K})$ ]{Fit parameters for Eqn.~\ref{eq:AHE_scaling_relation} to transverse resistivity $\rho_{xy} - \rho_{xy}(T=120K) = \rho^A_{xy}+\rho^T_{xy}$ data.}\label{fig:AHE_curve_fit_params_app}
\end{figure}





%%%%%%%%%%%%%%%%%%%%%%%%%%%%%%%%%
\newpage
% \annexe{Demagnetization Effects on AMRO Amplitude}

\annexe{Demagnetization Effects}\label{app:demag}
% TODO:
% \begin{enumerate}
%     \item Just explain the mechanical steps of what you did. Theory/justification is not needed.
%     \item Only the total magnetization matters for the MR(H, M(H), theta) consideration, but we need Pauli magnetism to extrapolate to 9 T...
%     \item demag B calculation needs to be updated s.t. $B=\mu_0(H+M)$ (SI) or $B=H+4\pi M$ (CGS), cause right now it's just divided by 10000
%     \item Ensure that $a_{Dem}$ is in terms of the mean resistivity $\bar\rho(H,T)$, to match the amplitudes of the AMRO data...
%     \item We neglected $B=H+4\pi M_{vol}$ in Sec.~\ref{sec:mag_sus} since $\mu_0H>>4\pi M_{vol}$ (\textbf{todo: verify})
%     \item Finish updating the demagnetization appendix. 
%     \begin{enumerate}
%         \item Need to account for Pauli demag during 9T extrapolation. 
%         \item M(H) ->Extrapolate to 9T -> Get M(B) -> Get $\rho_{MR}(B, D)$ -> Get $\rho(B, D(\theta=\{0^\circ,90^\circ\}))$
%         \item Just the extrapolation needs to be updated with Pauli term. The rest is all $M_{vol}$.
%         \item remove mention of ACTRot12 and ACTRot14
%     \end{enumerate}
%     \item using a different M(H) sample than what is in Sec.~\ref{sec:mag_sus}, since we needed to know the sample dimensions for the D-factor
% \end{enumerate}

% \section{Demagnetization Introduction}

Here, we are interested in the effect on magnetoresistance of the magnetization per unit volume  $\vec{M}_{vol}$, which has been measured up to $\mu_0H=7$~T. We must extrapolate $\vec{M}_{vol}$ to $\mu_0H=9$~T using the results of Sec.~\ref{sec:mag_sus}, including the Pauli magnetism contribution. A demagnetization field ($\vec{M}_{D}$) will occur in any magnetized material, and the geometry of the sample will ultimately determine how it deviates from $\vec{M}_{vol}$. We account for the influence of sample geometry by making use of a demagnetization factor $D$, such that~\cite{blundell_magnetism_2001}~:
\begin{equation}
    \vec{M}_d = -D\vec{M}_{vol}
\end{equation}

\noindent Herein, we assume the samples to be perfect rectangular prisms for which~\cite{aharoni_demagnetizing_1998}~:
\begin{equation}\label{eq:demag_factor}
\begin{split}
    D = \frac{1}{\pi} \Biggl( &\frac{(b^2 - c^2) \ln\left(\frac{\sqrt{a^2 + b^2 + c^2} - a}{\sqrt{a^2 + b^2 + c^2} + a}\right)}{2bc} + \frac{(a^2 - c^2) \ln\left(\frac{\sqrt{a^2 + b^2 + c^2} - b}{\sqrt{a^2 + b^2 + c^2} + b}\right)}{2ac} + \frac{b \ln\left(\frac{\sqrt{a^2 + b^2} + a}{\sqrt{a^2 + b^2} - a}\right)}{2c} \\
    &+ \frac{a \ln\left(\frac{\sqrt{a^2 + b^2} + b}{\sqrt{a^2 + b^2} - b}\right)}{2c} + \frac{c \ln\left(\frac{\sqrt{b^2 + c^2} - b}{\sqrt{b^2 + c^2} + b}\right)}{2a} + \frac{c \ln\left(\frac{\sqrt{a^2 + c^2} - a}{\sqrt{a^2 + c^2} + a}\right)}{2b} \\
    &+ 2 \arctan\left(\frac{ab}{c\sqrt{a^2 + b^2 + c^2}}\right) + \frac{a^3 + b^3 - 2c^3}{3abc} + \frac{(a^2 + b^2 - 2c^2) \sqrt{a^2 + b^2 + c^2}}{3abc} \\
    &+ \frac{c(\sqrt{a^2 + c^2} + \sqrt{b^2 + c^2})}{ab} - \frac{(a^2 + b^2)^{3/2} + (b^2 + c^2)^{3/2} + (c^2 + a^2)^{3/2}}{3abc}\Biggr)
\end{split}
\end{equation}
\noindent where $c$ is half the length of the dimension of the axis parallel to the magnetic field, and the orthogonal dimensions are half the lengths of $a$ and $b$. The measured samples were prepared through polishing to approximate rectangular prisms.  \\

\noindent Since $D$ is dependent on the geometry of the sample relative to the magnetic field, $D$ possesses a $\theta$-dependence. It is necessary to eliminate this as a potential source of oscillations in the AMRO data. Such a demagnetization oscillation will have an amplitude of the following form~:
\begin{equation}\label{eq:demag_amp_calc}
    a_{dem}(\vec{B},T) = \frac{\rho(\vec{B}, T, \theta=90^\circ)- \rho(\vec{B}, T, \theta=0^\circ)}{\overline{\rho}(\vec{B}, T)}
\end{equation}
\noindent where $\bar\rho(\vec{B}, T)$ is the mean resistivity of the oscillation and $\vec{B}$ is the magnetic flux density~\cite{beaudin_possible_2022, aharoni_demagnetizing_1998, ashcroft_solid_1976}~:
\begin{align}\label{eq:B_of_H_and_M}
    \vec{B} &= \vec{H}+4\pi(1-D)\vec{M}_{vol}  \quad \text{(CGS)}\\
    \vec{B} &= \mu_0(\vec{H}+(1-D)\vec{M}_{vol}) \quad \text{(SI)}
\end{align}
\noindent which allows us to account for the influence of $\vec{H}$, $\vec{M}(\vec{H})$ and $D(\theta)$ on AMRO. For simplicity, vector notation will no longer be used moving forward. All symbols that were formerly vectors will be the components of their respective field that are parallel to the magnetic field $H$.\\\\

%%%%%%%%%%%%%%%%%%%%%%%%%%%%%%
% \section{Extrapolating $M_{vol}$ to $9$~T by Brillouin Best Fit}\label{app:Mag_extrap_brill}


\noindent SQUID measurements were performed on a crystal of YbPdBi grown in a flux of Pb up to an applied magnetic field of $\mu_0H=7$~T. It was necessary to extrapolate the $M(H)$ to $\mu_0H=9$~T, and so a fit was performed of Eqn.~\ref{eq:simp_brillouin} with the Pauli magnetism term Eqn.~\ref{eq:Pauli_magnetization}. For this extrapolation, we are only interested in the relationship between $M_{vol}$ and $H$, so for now, we omit the influence of $D$. Further, we perform this extrapolation on $M(H)$ data that was taken using different samples than what was presented in Sec.~\ref{sec:mag_sus}. The samples here more closely approximate rectangular prisms.\\

\noindent The results of this extrapolation are shown in Fig.~\ref{fig:brillouin_best_fits} for $T=15$~K, $T=10$~K, $T=5$~K, and $T=2$~K, with their respective fit residuals. The red crosses show data, the blue lines show the respective best fits, and the yellow crosses show the extrapolated data points. Fit parameters for each best fit are presented in Table~\ref{tab:brill_fit_params}.\\\\
% ~\cite{blundell_magnetism_2001, gabrielle_quantum_nodate, aharoni_demagnetizing_1998},
% \begin{align}\label{eq:Brill_fit_eqns}
%     M_{vol} &= M_s * B_J(x)+cH\\
%     M_s &= Jng_j\mu_B \\
%     x &= \frac{g_j \mu_B J }{k_b}\frac{B}{T} \\
%     B &= \left(H+4\pi(1-D)M_{vol}\right)\\
%     g_j &= \frac{3}{2} + \frac{S(S+1)-L(L+1)}{2J(J+1)}
% \end{align}
% \noindent where the parameters of the fit function are  the total angular momentum $J$,  the number of magnetic moments per unit volume in the sample $n$, and a fudge factor $c$ introduced by us for this analysis. $B_J(x)$ is the Brillouin function derived from mean-field theory,  $S=\frac{1}{2}$, $g_J$ is the g-Lande factor,$L = J-S$ is the orbital angular momentum, $D$ is the demagnetization factor~\cite{aharoni_demagnetizing_1998}. The phenomenological $cH$ term improved the fits for data taken $T=5$~K and $T=2$~K. The inclusion of this term should be interpreted as a measure of the break down of the mean-field assumptions upon which the Brillouin model is predicated. Fit parameters are shown in Table~\ref{tab:brill_fit_params}; a value given without uncertainties indicates the parameter was fixed for curve fitting at that temperature. The saturation magnetization for each temperature $M_s$ was calculated as per Eqn.~\ref{eq:Brill_fit_eqns}.\\



\begin{figure}[H]
    \centering
    \begin{subfigure}{.45\textwidth} \includegraphics[width=\linewidth]{Figures_Ch4_Synth_Charac/Mag_and_Sus/Brillouin Fits and Extrapolation, Linear Term with L 15K.pdf}
    \end{subfigure}
    \centering
     \begin{subfigure}{.42\textwidth}\includegraphics[width=\linewidth]{Figures_Ch4_Synth_Charac/Mag_and_Sus/Brillouin Fits and Extrapolation, Linear Term with L 10K.pdf}   
    \end{subfigure}
    \centering
    \begin{subfigure}{.45\textwidth} \includegraphics[width=\linewidth]{Figures_Ch4_Synth_Charac/Mag_and_Sus/Brillouin Fits and Extrapolation, Linear Term with L 5K.pdf}
    \end{subfigure}
    \centering
     \begin{subfigure}{.45\textwidth}\includegraphics[width=\linewidth]{Figures_Ch4_Synth_Charac/Mag_and_Sus/Brillouin Fits and Extrapolation, Linear Term with L 2K.pdf}   
    \end{subfigure}
   \caption[Extrapolation of $M(H)$ to $\mu_0H=9$~T]{(Blue Line) Brillouin fits. (Red Crosses) Magnetization data. (Yellow Crosses) extrapolations. Magnetization of YbPdBi in units per unit volume. Minimum and maximum are plotted as dashed, yellow lines. } \label{fig:brillouin_best_fits}
\end{figure}


\begin{table}[H]
\centering
{\renewcommand{\arraystretch}{1.7}}
\begin{tabular}{c|c|c|c|c}
    $T$ (K)& $J$ & $M_p$ $\left(\frac{\text{emu/cm$^3$}}{T}\right)$& $n$ (mols/cm$^3$)& $M_s$ (emu/cm$^3$)  \\ \hline 
    15 & 2.6861 $\pm$ 0.0006 & 0  & 0.003189 & 56.75  $\pm$  0.01\\\hline
    10 & 2.494$\pm$ 0.001 & 0 & 0.003189  & 53.33 $\pm$ 0.02\\\hline
    5 & 1.917 $\pm$ 0.003 & 0.60 $\pm$ 0.01   & 0.003189  & 43.04  $\pm$ 0.06\\\hline
    2 & 1.45 $\pm$ 0.02 & 2.01 $\pm$ 0.02& 0.00257$\pm$ 0.00003 & 28.0 $\pm$ 0.4 
\end{tabular}
\caption[Demagnetization Best Fit Parameters for $M(H)$]{Parameters for best fits of Eqn.~\ref{eq:simp_brillouin} and Eqn.~\ref{eq:Pauli_magnetization} to $M(H)$ data of YbPdBi. 
% Errors shown were obtained from the diagonal of the covariance matrix of each fit. For the 4f-orbital of Yb$^{+3}$ we expect $J=7/2= 3.5$.
} \label{tab:brill_fit_params}
\end{table}


% \noindent The residuals of the fit and the experimental data demonstrate that the fitted function does not fully capture the physics of $M_{vol}$. This is seen most strongly in the 2K and 5K data at higher fields. This is sensible since there is strongly correlated physics present at lower temperatures. The differences from the fit are sufficiently small such that we may treat the fitted parameters as a reasonable approximation with which to extrapolate. A fit performed on a data set at 2K take on another sample resulted in a similar fit.\textbf{todo: show figure?}\\

% \noindent \textbf{TODO:} M0 and A should theoretically be temperature independent (insert equations from blundell). Each is directly proportional to the Land\'e g-value $g_j$, and so are functions the quantum numbers of the f-orbital. The fact they are both decreasing with temperature could indicate weird stuff is happening to the orbital. Maybe if the ultimate groundstate is indeed AFM~\cite{pietri_magnetoresistance_2000}, then this tendency grows as the T drop and it is harder for the material to magnetize in response to the applied magnetic field.\\

% \noindent These parameters will allow us to approximate the magnetization of both the MR and the AMR samples during their respective experiments. Thus, we may find the MR in terms of $B(D,H,M)$, compare it against what would be expected when we vary D and M to match those values for the AMRO experiments, and extract a theoretical oscillation amplitude, $a_{dem}$.\\


%%%%%%%%%%%%%%%%%%%%%%%%%%%%%%
% \section{Extracting $a_{dem}$ from $\rho_{MR}$}
% \begin{figure}[H]
%     \centering
%     \begin{subfigure}{.45\textwidth} \includegraphics[width=\linewidth]{Figures_Ch4_Synth_Charac/Resistivity/MR Cheb Fits 15K.pdf}
%     \end{subfigure}
%     \centering
%      \begin{subfigure}{.45\textwidth}\includegraphics[width=\linewidth]{Figures_Ch4_Synth_Charac/Resistivity/MR Cheb Fits 10K.pdf}   
%     \end{subfigure}
%     \centering
%     \begin{subfigure}{.45\textwidth} \includegraphics[width=\linewidth]{Figures_Ch4_Synth_Charac/Resistivity/MR Cheb Fits 5K.pdf}
%     \end{subfigure}
%     \centering
%      \begin{subfigure}{.45\textwidth}\includegraphics[width=\linewidth]{Figures_Ch4_Synth_Charac/Resistivity/MR Cheb Fits 2K.pdf}   
%     \end{subfigure}
%    \caption*[Chebyshev Interpolations of Magnetoresistance]{chebyshev interpolations of magnetoresistance......} 
%    \label{fig:MR_cheb_fit}
% \end{figure}
% \noindent Now that we have extrapolated $M_{vol}$ up to 9T and found these fit parameters, we may approximate the magnetization of the MR sample and so find $\rho_{M}$ as a function of $B$. Doing so allows us to determine the expected resistivity for the AMRO samples when they are oriented at $\theta = 0$ $(D_{0})$ and at $\theta = 90$ $(D_{90})$. \\

% \noindent Using the relevant fitted parameters found in the previous section, we find an approximate value for the magnetization of $\rho_{MR}$ by finding the roots of~:
% \begin{align}
%     0 &= M - M_0B_j(x)\\
%     x &= \frac{A}{T}\left(H+(1-D)M\right)
% \end{align}
% \noindent using the optimize.fsolve function from the scipy Python library. With these magnetization values, we then find $\rho_{MR}\left(B\right)$ for YbPdBi, and interpolate these curves using a 15-degree Chebyshev polynomial. The results of these interpolations are shown in Fig.~\ref{fig:MR_cheb_fit}, and the residuals demonstrate the interpolations are in good agreement with the MR data.\\


% \noindent We repeat this root-finding process for the AMRO data for $D_{0}$ and $D_{90}$ to find $M_{vol}$, which is used to calculate $B$ in the sample at these orientations. Any oscillation in the magnetoresistance due to the rotation of the sample will possess a two-fold symmetry. Thus we may extract the oscillation amplitude $a_{dem}$ by taking half the difference between the expected $\rho_{MR}(B)$ and dividing by the mean of these two resistivities. For clarity, the absolute values of these amplitudes are presented in Fig.~\ref{fig:demag_amplitudes} as a percentage of the AMRO mean resistivity. Negative amplitudes were found for ACTRot11 due to the sample's geometry.\\

\begin{figure}[H]
    \centering
    \includegraphics[width=\linewidth]{Figures_Ch5_AMRO_Results/Demag Amplitudes vs B.pdf}
    \caption[Demagnetization Oscillation Amplitudes]{Absolute values of the demagnetization amplitudes as a function of magnetic flux density for each indicated temperature. Each amplitude is shown as a percentage of the mean resistivity for each relevant AMRO experiment. (Orange) Sample Puck 13 $\parallel$ geometry;  (Blue) Sample Puck 11 $\perp$ geometry.}
    \label{fig:demag_amplitudes}
\end{figure}

Using the extrapolated $M_{vol}(H)$ with Eqn.~\ref{eq:B_of_H_and_M} and Eqn.~\ref{eq:demag_factor}, we were able to map the $\rho_{MR}$ onto $B$ by recursively solving for $M_{vol}(B)$ using $D$ for each sample. We also calculated $B$ for each AMRO sample when oriented $\theta=0^\circ$ and $\theta=90^\circ$ orientations. This allows us to obtain an expected change in $\rho_{MR}$ caused by the changing $B$. These changes in $\rho_{MR}$ were then used to calculate $a_{dem}$ according to Eqn.~\ref{eq:demag_amp_calc}. The results are shown in Fig.~\ref{fig:demag_amplitudes} as a function of $B$.\\

\noindent Based on this work, we find that any demagnetization oscillation for Sample Puck 11 $\perp$ or Sample Puck 13 $\parallel$ is effectively non-existent when comparing the expected magnetoresistance for the $B$ at $0^\circ$ to that expected for the $B$ at $90^\circ$. \\

% \noindent Each of these patterns demonstrates that the amplitude is a small fraction of a percent of the mean AMRO resistivity. For each of the four samples, $a_{dem}$ follows a similar pattern where the amplitude decreases with the applied field. This is sensible because the source of the demagnetization field (the magnetization) will also decrease. There is a clear deviation from this pattern for the $2$~K data, which is most likely due to the earlier deviation of the Brillouin fit for the $2$~K magnetization at higher field strengths. Fortunately, even if the  $2$~K amplitude at $9$~T were to be quadruple that of the $3$~T amplitude (as is approximately so for the other three temperatures), it would still be around less than half a percent. Thus, the effects of the demagnetizing field are insignificant within the AMRO data and can be neglected. \\



%%%%%%%%%%%%%%%%%%%
% % %\section{Topology Analyses}\label{app:weyl_fits}
% \annexe{Magnetoresistance Curves Used to Fit for Chiral Effects in Electron Transport}\label{app:weyl_fits}

% \begin{figure}[H]
%     \centering
%     \includegraphics[width=0.9\linewidth]{Figures_Appendices/YbPdBi, JF039, actrot12, magnetoRES, Dresden Data APPENDIX.pdf}
%     \caption*{Magnetoresistance measurements of YbPdBi with angle dependence. $0^\circ$ orientation has $H\perp I$, while $90^\circ$ has $H||I$.}
%     \label{fig:chiral_search_raw_MR}
% \end{figure}


%%%%%%%%%%%%%%%%%%%%%%%%%%%%%%%%%%%%%
%%%%%%%%%%%%%%%%%%%%%%%%%%%%%%%%%%%%%
%%%%%%%%%%%%%%%%%%%%%%%%%%%%%%%%%%%%%
% \annexe{Kondo Scaling Results}\label{app:kondo_scaling}

% Shown in Fig.~\ref{fig:single_ion_Kondo_scaling_comparison} on the left is our $\rho(H, T)$ plotted against the scaling function for which $T^+=0.96$~K. Units are unnecessary for this discussion and so are omitted for convenience. Plotted to the right of this is the same data against the scaling function for which $T^+=2.5$~K. Isotherms omitted from the minimization have been included in the plots to demonstrate their deviation from overlap. It is important to note that this is a very qualitative method of analysis which requires a fairly dramatic change in isotherm behaviour for it to be 'ruled out'. Not only does such a decision depend on one's subjective opinion of 'overlapping', but even worse, on the formatting of the plot features such as x-limits and marker size. Nonetheless, we may say that the magnetoresistivity data taken at \textbf{FIX THIS: $T=3$~K, $T=3$~K, and $T=3$~K} is no longer dominated by Kondo scattering as $H$.\\


% \begin{figure}[H]
%     \centering
%     \begin{subfigure}[b]{0.9\textwidth}
%         \centering
%       \includegraphics[width=\textwidth]{Figures_Ch4_Synth_Charac/Resistivity/Kondo scaling for T_star 0p96K minT 3.pdf}
%     \end{subfigure}
%     \centering
%     \begin{subfigure}[b]{0.9\textwidth}
%         \centering
%       \includegraphics[width=\textwidth]{Figures_Ch4_Synth_Charac/Resistivity/Kondo scaling for T_star 2p5K minT 3.pdf}
%     \end{subfigure}
%     \caption*[$\rho(H,T)$ Kondo Scaling Using Single Ion Kondo Impurity Model]{Single ion Kondo impurity scaling behaviour of $\rho(H, T)$ described by Eqn.~\ref{eq:Kondo_scaling}. For overlapping isotherms, the mechanism of the magnetoresistance is expected to be comparable. (Left) Our data scaled using $T^+=0.96$~K; (Right) Our data scaled using  $T^+=2.5$~K~\cite{pietri_magnetoresistance_2000}, which shows a clear reduction in isotherm overlap.}
%     \label{fig:single_ion_Kondo_scaling_comparison}
% \end{figure}



%%%%%%%%%%%%%%%%%%%%%%%%%%%%%%%%%%%%%
%%%%%%%%%%%%%%%%%%%%%%%%%%%%%%%%%%%%%
%%%%%%%%%%%%%%%%%%%%%%%%%%%%%%%%%%%%%





    
\annexe{Chiral Fits to Magnetoconductance}\label{app:chiral_fits_to_MC}
% \begin{figure}[H]  
%     \centering
%     \begin{subfigure}{\linewidth}  
%        \includegraphics[width=0.95\linewidth]{Figures_Appendices/chiral_fitting_app/YbPdBi_actrot12_sym_chiral_fitting_for_0deg.pdf}
%     \end{subfigure}
%     \centering
%     \begin{subfigure}{\textwidth} 
%     \includegraphics[width=0.95\linewidth]{Figures_Appendices/chiral_fitting_app/YbPdBi_actrot12_sym_chiral_fitting_for_15deg.pdf}
%     \label{fig:MC_fit_15deg}
%     \end{subfigure}
%     \centering
%      \begin{subfigure}{\textwidth}
%      \includegraphics[width=0.95\linewidth]{Figures_Appendices/chiral_fitting_app/YbPdBi_actrot12_sym_chiral_fitting_for_30deg.pdf} 
%      \label{fig:MC_fit_30deg}
%     \end{subfigure}
%     \centering
%      \begin{subfigure}{\linewidth} 
%      \includegraphics[width=0.95\linewidth]{Figures_Appendices/chiral_fitting_app/YbPdBi_actrot12_sym_chiral_fitting_for_45deg.pdf} 
%     \end{subfigure}
%     \centering
%      \begin{subfigure}{\textwidth}
%      \includegraphics[width=0.95\linewidth]{Figures_Appendices/chiral_fitting_app/YbPdBi_actrot12_sym_chiral_fitting_for_60deg.pdf}
%      \label{fig:MC_fit_60deg}
%     \end{subfigure}
%     \centering
%      \begin{subfigure}{\textwidth}
%      \includegraphics[width=0.95\linewidth]{Figures_Appendices/chiral_fitting_app/YbPdBi_actrot12_sym_chiral_fitting_for_75deg.pdf}
%      \label{fig:MC_fit_75deg}
%     \end{subfigure}
%     \centering
%      \begin{subfigure}{\linewidth} 
%      \includegraphics[width=0.95\linewidth]{Figures_Appendices/chiral_fitting_app/YbPdBi_actrot12_sym_chiral_fitting_for_90deg.pdf}
%     \end{subfigure}
%     \caption[Grid of Chiral Fits of $\sigma(H)$ from Sample Puck 12 $\parallel$ for $0^\circ \leq \theta \leq 90^\circ$ and $10\text{~K}\leq T \leq 150\text{~K} $]{Fits of Eqn.~\ref{eq:MC_chiral} to magnetoconductance data from Sample Puck 12 $\parallel$. Note there is a $3.9^\circ$ offset of the current vector relative to the plane of rotation due to a misalignment of sample mounting.}\label{fig:extra_chiral_MR_fits_app}
% \end{figure}

\begin{figure}[H]  
    \centering 
    \includegraphics[width=0.95\linewidth]{Figures_Appendices/chiral_fitting_app/YbPdBi_actrot12_sym_chiral_fitting_GRID.pdf}
    \caption[Grid of Chiral Fits of $\sigma(H)$ from Sample Puck 12 $\parallel$ for $0^\circ \leq \theta \leq 90^\circ$ and $10\text{~K}\leq T \leq 150\text{~K} $]{Fits of Eqn.~\ref{eq:MC_chiral} to magnetoconductance data from Sample Puck 12 $\parallel$. Note there is a $3.9^\circ$ offset of the current vector relative to the plane of rotation due to a misalignment of sample mounting.}\label{fig:extra_chiral_MR_fits_app}
\end{figure}




\begin{figure}[H]
    \centering
    \begin{subfigure}[b]{0.43\textwidth}
        \centering
      \includegraphics[width=\textwidth]{Figures_Appendices/chiral_fitting_app/YbPdBi_actrot12_sym_chiral_fit_T10.pdf}
    \end{subfigure}    \centering
    \begin{subfigure}[b]{0.43\textwidth}
        \centering
      \includegraphics[width=\textwidth]{Figures_Appendices/chiral_fitting_app/YbPdBi_actrot12_sym_chiral_fit_T15.pdf}
    \end{subfigure}
    \centering
    \begin{subfigure}[b]{0.43\textwidth}
        \centering
      \includegraphics[width=\textwidth]{Figures_Appendices/chiral_fitting_app/YbPdBi_actrot12_sym_chiral_fit_T40.pdf}
    \end{subfigure}
        \centering
    \begin{subfigure}[b]{0.43\textwidth}
        \centering
      \includegraphics[width=\textwidth]{Figures_Appendices/chiral_fitting_app/YbPdBi_actrot12_sym_chiral_fit_T60.pdf}
    \end{subfigure}
        \centering
    \begin{subfigure}[b]{0.43\textwidth}
        \centering
      \includegraphics[width=\textwidth]{Figures_Appendices/chiral_fitting_app/YbPdBi_actrot12_sym_chiral_fit_T80.pdf}
    \end{subfigure}
    \centering
    \begin{subfigure}[b]{0.43\textwidth}
        \centering
      \includegraphics[width=\textwidth]{Figures_Appendices/chiral_fitting_app/YbPdBi_actrot12_sym_chiral_fit_T100.pdf}
    \end{subfigure}
    \centering
    \begin{subfigure}[b]{0.43\textwidth}
        \centering
      \includegraphics[width=\textwidth]{Figures_Appendices/chiral_fitting_app/YbPdBi_actrot12_sym_chiral_fit_T150.pdf}
    \end{subfigure}
    \caption[Overlaid Chiral Fits at Each $T$]{Overlaid fits of Eqn.~\ref{eq:MC_chiral} to magnetoconductance data from Sample Puck 12 $\parallel$, showing that the angular dependence has a slight temperature dependence. The \textit{y}-axis values are given by $ \Delta\sigma = \frac{\sigma(H)-\sigma(0)}{\sigma(0)}$.}
    \label{fig:MC_chiral_fits_as_fcn_of_theta}
\end{figure}

\annexe{Fitting of the Chiral Constant $c_a(T)$}\label{app:T_dependence_chiral_constant}

% Due to some difficulty finding the appropriate constants of Eqn.~\ref{eq:ca_T_dep}, we briefly re-derive the temperature dependence of the chiral anomaly $c_a$. We repeat the derivation by Li \emph{et al}.~\cite{li_chiral_2016}, who adapted work on quark-gluon plasma by Fukushima \emph{et al}.~\cite{fukushima_chiral_2008}. We do this in a more sparing fashion without background or justification. \\

% \noindent Beginning with Eq.43 of Fukushima, we obtain the chiral charge density, $n_5=n_R-n_L$~:
% \begin{equation}
%     n_5 = \frac{\mu_5^3}{3\pi^2}+\frac{\mu_5}{3}\left(T^2+\frac{\mu^2}{\pi^2}\right)
% \end{equation}



\begin{figure}[H]
    \centering
    \begin{subfigure}[b]{0.3\textwidth}
        \centering
      \includegraphics[width=\textwidth]{Figures_Appendices/chiral_fitting_app/YbPdBi, Ca of T from MC Fit, 0deg, APPENDIX.pdf}
    \end{subfigure}    \centering
    \begin{subfigure}[b]{0.3\textwidth}
        \centering
      \includegraphics[width=\textwidth]{Figures_Appendices/chiral_fitting_app/YbPdBi, Ca of T from MC Fit, 15deg, APPENDIX.pdf}
    \end{subfigure}
    \centering
    \begin{subfigure}[b]{0.3\textwidth}
        \centering
      \includegraphics[width=\textwidth]{Figures_Appendices/chiral_fitting_app/YbPdBi, Ca of T from MC Fit, 30deg, APPENDIX.pdf}
    \end{subfigure}
        \centering
    \begin{subfigure}[b]{0.3\textwidth}
        \centering
      \includegraphics[width=\textwidth]{Figures_Appendices/chiral_fitting_app/YbPdBi, Ca of T from MC Fit, 45deg, APPENDIX.pdf}
    \end{subfigure}
        \centering
    \begin{subfigure}[b]{0.3\textwidth}
        \centering
      \includegraphics[width=\textwidth]{Figures_Appendices/chiral_fitting_app/YbPdBi, Ca of T from MC Fit, 60deg, APPENDIX.pdf}
    \end{subfigure}
    \centering
    \begin{subfigure}[b]{0.3\textwidth}
        \centering
      \includegraphics[width=\textwidth]{Figures_Appendices/chiral_fitting_app/YbPdBi, Ca of T from MC Fit, 75deg, APPENDIX.pdf}
    \end{subfigure}
    \centering
    \begin{subfigure}[b]{0.3\textwidth}
        \centering
      \includegraphics[width=\textwidth]{Figures_Appendices/chiral_fitting_app/YbPdBi, Ca of T from MC Fit, 90deg, APPENDIX.pdf}
    \end{subfigure}
    \caption[Individual Fits $c_a(T)$ for $0^\circ \leq \theta \leq 90^\circ$]{Individual fits of the chiral coefficient $c_a(T)$ to Eqn.~\ref{eq:ca_T_dep}, where $0^\circ$ corresponds to $B\perp I$ and $90^\circ$ corresponds to $B||I$. Due to their similarity, we fitted the data together in the main text.}
    \label{fig:MC_ca_fit_separate}
\end{figure}




\begin{figure}[H]
    \centering
    \begin{subfigure}[t]{0.1\textwidth}
        \textbf{a)}
    \end{subfigure}    
    \centering
    \begin{subfigure}[t]{0.85\textwidth}  
        \includegraphics[width=\linewidth, valign=t]{Figures_Appendices/chiral_fitting_app/YbPdBi, mu parameter from Ca Fit for all angles.pdf}
    \end{subfigure}
    \centering
    \begin{subfigure}[t]{0.1\textwidth}
        \textbf{b)}
    \end{subfigure}    
    \centering
    \begin{subfigure}[t]{0.85\textwidth}  
        \includegraphics[width=\linewidth, valign=t]{Figures_Appendices/chiral_fitting_app/YbPdBi, vf3tauv parameter from Ca Fit for all angles.pdf}
    \end{subfigure}
    \caption[Parameters of Individual Fits of $c_a(T)$]{Parameters of best fits of $T$-dependence of the chiral coefficient $c_a$ described by Eqn.~\ref{eq:ca_T_dep} where $0^\circ$ corresponds to $B\perp I$ and $90^\circ$ corresponds to $B||I$.}
    \label{fig:MC_ca_fit_separate_params}
\end{figure}



%%%%%%%%%%%%%%%%%%%%%%%%%%%
\newpage
\annexe{Additional Neutron Diffraction Data}\label{app:neutron_diff}
% %\section{Neutron Diffraction Graphs}
% \begin{figure}[H]
%     \centering
%     \includegraphics[width=\linewidth]{Figures_Ch4_Synth_Charac/Neutron/YbPdBi, 20K Fitted Graph.pdf}
%     \caption*{Powder neutron scattering data at $T=20$~K for YbPdBi taken on the HRPT at SINQ, PSI. Data fitted in GSAS-II.}\label{fig:fitted_neutron_20K}
% \end{figure}

\begin{figure}[H]
    \centering
    \includegraphics[width=\linewidth]{Figures_Appendices/neutron_app/YbPdBi, 2K Fitted Graph.pdf}
    \caption[Powder Neutron Scattering Data at $T=2$~K]{Powder neutron scattering data at $T=2$~K for YbPdBi taken on the HRPT at SINQ, PSI. Data fitted in GSAS-II.}\label{fig:fitted_neutron_2K}
\end{figure}









%%%%%%%%%%%%%%%%%%%%%%%%%%%
\newpage
% \annexe{AMRO Measurements}\label{app:full_AMRO_grids}
%%%%%%%%%%%%%%%%%%%%%%%%
%%%%%%%%%%%%%%%%%%%%%%%%
%%%%%%%%%%%%%%%%%%%%%%%%
% %\section{AMRO Plots}
%%%%%%%%%%%%%%%%%%%%%%%%
%%%%%%%%%%%%%%%%%%%%%%%%
\annexe{Raw AMRO Data}\label{app:raw_amro}
\begin{figure}[H]
    \centering
    \includegraphics[width=1\linewidth]{Figures_Appendices/AMRO raw petals and fits app/ACTRot11 perp raw AMRO.pdf}
    \caption[Raw AMRO Data for Sample Puck 11 $\perp$]{Raw AMRO data for Sample Puck 11 $\perp$. Field strength increases from left to right; temperature increases from top to bottom. (Yellow Markers) Data taken at $-H$ values; (Blue Markers) Data taken at $+H$. }
    \label{fig:ACTRot11_raw_amro}
\end{figure}

% \begin{figure}[H]
%     \centering
%     \includegraphics[height=0.9\linewidth]{Figures_Appendices/AMRO raw petals and fits app/ACTRot12 para raw AMRO.pdf}
%     \caption*{Raw AMRO data for ACTRot12, taken in the parallel geometry.  Field strength increases left to right, temperature increases from top to bottom. (Yellow markers) Data taken at $-H$ values; (Blue markers) Data taken at $+H$. }
%     \label{fig:ACTRot12_raw_amro}
% \end{figure}

\begin{figure}[H]
    \centering
    \includegraphics[width=0.85\linewidth]{Figures_Appendices/AMRO raw petals and fits app/ACTRot13 para raw AMRO.pdf}
    \caption[Raw AMRO Data for Sample Puck 13 $\parallel$]{Raw AMRO data for Sample Puck 13 $\parallel$.  Field strength increases from left to right, and temperature increases from top to bottom. (Yellow Markers) Data taken at $-H$ values; (Blue Markers) Data taken at $+H$.}
    \label{fig:ACTRot13_raw_amro}
\end{figure}

% \begin{figure}[H]
%     \centering
%     \includegraphics[width=0.8\textwidth]{Figures_Appendices/AMRO raw petals and fits app/ACTRot14 perp raw AMRO.pdf}
%     \caption*{Raw AMRO data for ACTRot14, taken in the perpendicular geometry.  Field strength increases left to right, temperature increases from top to bottom. (Yellow markers) Data taken at $-H$ values; (Blue markers) Data taken at $+H$. }
%     \label{fig:ACTRot14_raw_amro}
% \end{figure}



% \begin{figure}[H]
%     \centering
%     \includegraphics[width=0.75\linewidth]{Figures_Appendices/AMRO raw petals and fits app/ACTRot11 ACTRot14 overlaid, pct change from mean.pdf}
%     \caption*{Perpendicular geometries overlaid one another. Shown as a percentage change from their respective mean values for each T and H.}
%     \label{fig:11_14_overlaid}
% \end{figure}


% \begin{figure}[H]
%     \centering
%     \includegraphics[width=0.75\textwidth]{Figures_Appendices/AMRO raw petals and fits app/ACTRot12 ACTRot13 overlaid, pct change from mean.pdf}
%     \caption*{Parallel geometries overlaid one another. Shown as a percentage change from their respective mean values for each T and H.}
%     \label{fig:12_13_overlaid}
% \end{figure}



%%%%%%%%%%%%%%%%%%%%%%%%
%%%%%%%%%%%%%%%%%%%%%%%%
%%%%%%%%%%%%%%%%%%%%%%%%
\annexe{AMRO Wave Plots}\label{app:AMRO_wave_plots}
\begin{figure}[H]
    \centering
    \includegraphics[width=0.7\linewidth]{Figures_Appendices/AMRO raw petals and fits app/YbPdBi, ACTRot11 perp Full Wave Grid real res.pdf}
    \caption[Wave Plots of AMRO Data for Sample Puck 11 $\perp$]{AMRO data plotted as waves for Sample Puck 11 $\perp$ colour-coded by temperature.}
    \label{fig:actrot11_AMRO_full_wave_grid_app}
\end{figure}




\begin{figure}[H]
    \centering
    \includegraphics[width=0.75\linewidth]{Figures_Appendices/AMRO raw petals and fits app/YbPdBi, ACTRot13 para Full Wave Grid real res.pdf}
    \caption[Wave Plots of AMRO Data for Sample Puck 13 $\parallel$]{AMRO data plotted as waves for Sample Puck 13 $\parallel$ colour-coded by temperature.}
    \label{fig:actrot13_AMRO_full_wave_grid_app}
\end{figure}





%%%%%%%%%%%%%%%%%%%%%%%%
%%%%%%%%%%%%%%%%%%%%%%%%
%%%%%%%%%%%%%%%%%%%%%%%%
\annexe{AMRO Fourier Transforms}\label{app:amro_FT_grids}

% \noindent Shown in Fig.~\ref{fig:act11_ft_grid} and Fig.~\ref{fig:act13_ft_grid} are the results of Fourier transforming the data shown previously in Fig.~\ref{fig:actrot11_AMRO_full_grid} and Fig.~\ref{fig:actrot13_AMRO_full_grid}.

\begin{figure}[H]
    \centering
    \includegraphics[width=0.75\linewidth]{Figures_Ch5_AMRO_Results/ACTRot11 AMRO FT Grid Full.pdf}
    \caption[Fourier Transformations of AMRO from Sample Puck 11 $\perp$]{Fourier Transformations of Sample Puck 11 $\perp$ AMRO measurements. Arranged to match Fig.~\ref{fig:actrot11_AMRO_full_grid}.}
    \label{fig:act11_ft_grid}
\end{figure}



\begin{figure}[H]
    \centering
    \includegraphics[width=0.8\linewidth]{Figures_Ch5_AMRO_Results/ACTRot13 AMRO FT Grid Full.pdf}
    \caption[Fourier Transformations of AMRO from Sample Puck 13 $\parallel$]{Fourier Transformations of Sample Puck 13 $\parallel$ AMRO measurements. Arranged to match Fig.~\ref{fig:actrot13_AMRO_full_grid}.}
    \label{fig:act13_ft_grid}
\end{figure}

%%%%%%%%%%%
% \annexe{AMRO Polar Plots}\label{app:AMRO_polar_app}


%%%%%%%%%%%%%%%%%%%%%%%%
%%%%%%%%%%%%%%%%%%%%%%%%
%%%%%%%%%%%%%%%%%%%%%%%%
\annexe{AMRO Best Fit Parameters}\label{app:fitted_AMRO_app}

% \begin{figure}[H]
%     \centering
%     \includegraphics[width=\linewidth]{Figures_Appendices/AMRO raw petals and fits app/AMRO ACTRot12 JF039 Best Fit Grid.pdf}
%     \caption*{Fitted oscillations, ample puck 12, para, seems shifted by 45 degrees. }
%     \label{fig:fitted_actrot12_4x4grid}
% \end{figure}


% \begin{figure}[H]
%     \centering
%     \includegraphics[width=\linewidth]{Figures_Appendices/AMRO raw petals and fits app/AMRO ACTRot14 JF039 Best Fit Grid.pdf}
%     \caption*{Fitted oscillations, ample puck 14, perp, This will likely benefit from using the Suzuki equation, I believe.}
%     \label{fig:fitted_actrot14_4x4grid}
% \end{figure}

\begin{figure}[H]
    \centering
    \includegraphics[width=\linewidth]{Figures_Appendices/AMRO raw petals and fits app/YbPdBi, best fit ABS Amp Plot, actrot11 actrot13, sym combined.pdf}
    \caption[Fitted AMRO Symmetry Amplitudes]{Amplitudes of symmetries fitted using  Eqn.~\ref{eq:AMRO_sym_eqn} to AMRO data, plotted in terms of absolute resistivity. }
    \label{fig:fitted_actrot12_abs_4x4grid}
\end{figure}

\begin{figure}[H]
    \centering
    \includegraphics[width=\linewidth]{Figures_Ch5_AMRO_Results/AMRO Fitted Plots/YbPdBi, best fit phase Plot, actrot11 actrot13, sym combined.pdf}
    \caption[Fitted AMRO Symmetry Phases]{Phases of best fits of Eqn.~\ref{eq:AMRO_sym_eqn} to AMRO data. Error bars are given by the standard deviation provided by the fit.}
    \label{fig:AMRO_best_fit_phase_dependencies_13_11}
\end{figure}


%%%%%%%%%%%%%%%%%%%%%%%%%%%%%%%%%%%%%%
% \newpage
\annexe{Sample Puck Dimensions and Photos}\label{app:puck_dims_photos}
% %\section{Sample Puck Photos}

\begin{table}[H]
    \centering
    \begin{tabular}{ c | c | c |c | c | c |c }
     ID & Geo. & W (mm) & H (mm) & L (mm) & Wire Sep. (mm) & $D_{010}/D_{001}$\\  \hline
     11 & $\perp$ & $0.93\pm0.03$ & $0.88\pm0.02$ & $2.10\pm0.02 $&$ 0.35\pm0.02$  &  0.94\\
     12 & $\parallel$ & $0.92\pm0.02$ & $0.8\pm0.1 $& $0.79\pm0.01 $& $0.19\pm0.02$  &  1.02\\
     13 & $\parallel$ & $1.80\pm0.03$ & $1.38\pm0.06$ & $1.11\pm0.01$ &$ 0.43\pm0.02$  & 1.24 %\\
     % 14 & $\perp$ & $0.55\pm0.01$ & $0.56\pm0.03$ &$ 10.7\pm0.8$ &$ 0.43\pm 0.01 $  &  1.01
    \end{tabular}
    \caption[Dimensions of Sample Puck Crystals]{Dimensions of single crystal samples used for AMRO and chiral anomaly experiments. Values are an average of multiple measurements taken with ImageJ. Measurement uncertainties were determined by taking the standard deviation of those measurements. The column $D_{010}/D_{001}$ shows the ratio of the demagnetization factor along the $\theta=0^\circ$ and the $\theta=90^\circ$ sample orientations. }
    \label{tab:AMRO_chiral_sample_dims}
\end{table}

\begin{figure}[H]
    \centering
    \begin{subfigure}[b]{0.95\textwidth}
        \centering
      \includegraphics[width=0.5\textwidth]{Figures_Appendices/Puck Photos app/ACTRot11_JF038_YbPdBi_wiring.pdf}
    \end{subfigure}
    \centering
    \begin{subfigure}[b]{0.95\textwidth}
        \centering
      \includegraphics[width=0.5\textwidth]{Figures_Appendices/Puck Photos app/ACTRot11_JF038_YbPdBi_sideview.pdf}
    \end{subfigure}
    \centering
    \begin{subfigure}[b]{0.95\textwidth}
        \centering
      \includegraphics[width=0.5\textwidth]{Figures_Appendices/Puck Photos app/ACTRot11_JF038_YbPdBi_overview.pdf}
    \end{subfigure}
    \caption[Microscope Photos of Sample Puck 11 $\perp$]{Microscope photos of Sample Puck 11 $\perp$ mounted in the perpendicular orientation. Graph paper is 1/4" square.}
    \label{fig:ACTRot11_sample_photos}
\end{figure}


\begin{figure}[H]
    \centering
    \begin{subfigure}[b]{\textwidth}
        \centering
      \includegraphics[width=0.5\textwidth]{Figures_Appendices/Puck Photos app/ACTRot12_JF039_YbPdBi_wiring.pdf}
    \end{subfigure}
    \centering
    \begin{subfigure}[b]{\textwidth}
        \centering
      \includegraphics[width=0.5\textwidth]{Figures_Appendices/Puck Photos app/ACTRot12_JF039_YbPdBi_sideview.pdf}
    \end{subfigure}
    \centering
    \begin{subfigure}[b]{\textwidth}
        \centering
      \includegraphics[width=0.5\textwidth]{Figures_Appendices/Puck Photos app/ACTRot12_JF039_YbPdBi_overview.pdf}
    \end{subfigure}
    \caption[Microscope Photos of Sample Puck 12 $\parallel$]{Microscope photos of Sample Puck 12 $\parallel$. Graph paper is 1/4" square.}
    \label{fig:ACTRot12_sample_photos}
\end{figure}

\begin{figure}[H]
    \centering
    \begin{subfigure}[b]{\textwidth}
        \centering
      \includegraphics[width=0.5\textwidth]{Figures_Appendices/Puck Photos app/ACTRot13_JF036_parallel_measured.pdf}
    \end{subfigure}
    \centering
    \begin{subfigure}[b]{\textwidth}
        \centering
      \includegraphics[width=0.5\textwidth]{Figures_Appendices/Puck Photos app/ACTRot13_JF036_parallel_side.pdf}
    \end{subfigure}
    \centering
    \begin{subfigure}[b]{\textwidth}
        \centering
      \includegraphics[width=0.5\textwidth]{Figures_Appendices/Puck Photos app/ACTRot13_JF036_parallel_overview.pdf}
    \end{subfigure}
    \caption[Microscope Photos of Sample Puck 13 $\parallel$]{Microscope photos of Sample Puck 13 $\parallel$. Graph paper grid dimension is $1$~mm$^2$.}
    \label{fig:ACTRot13_sample_photos}
\end{figure}


% \begin{figure}[H]
%     \centering
%     \begin{subfigure}[b]{\textwidth}
%         \centering
%       \includegraphics[width=0.5\textwidth]{Figures_Appendices/Puck Photos app/ACTRot14_JF036_perp_measured.pdf}
%     \end{subfigure}
%     \centering
%     \begin{subfigure}[b]{\textwidth}
%         \centering
%       \includegraphics[width=0.5\textwidth]{Figures_Appendices/Puck Photos app/ACTRot14_JF036_perp_side.pdf}
%     \end{subfigure}
%     \centering
%     \begin{subfigure}[b]{\textwidth}
%         \centering
%       \includegraphics[width=0.5\textwidth]{Figures_Appendices/Puck Photos app/ACTRot14_JF036_perp_overview.pdf}
%     \end{subfigure}
%     \caption*{Microscope photos of Sample Puck 14 (ACTRot14) mounted in the perpendicular orientation. Graph paper is 1mm square.}
%     \label{fig:ACTRot14_sample_photos}
% \end{figure}



%%%%%%%%%%%%%%%%%%%%%%%%%%%
\annexe{Single Crystal XRD for Pb-Flux Samples}\label{app:singl_xtal_XRD}

See following pages.
% %\section{Single Crystal XRD For Pb-Flux Grown Samples}
\includepdf[pages=1,pagecommand={},offset=-0cm -2cm]{bian873_cifreport.pdf}
\includepdf[pages=2,pagecommand={},offset=-0cm -2cm]{bian873_cifreport.pdf}
\includepdf[pages=3,pagecommand={},offset=-0cm -2cm]{bian873_cifreport.pdf}
\includepdf[pages=4,pagecommand={},offset=-0cm -2cm]{bian873_cifreport.pdf}
