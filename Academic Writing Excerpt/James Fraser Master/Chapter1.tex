%\introduction
\chapter{Introduction}\label{ch:intro}


 
% Understanding the intricacies of the Fermi surface in heavy fermion materials stands as a pivotal pursuit in modern condensed matter physics. 

In this study, our primary objective is to investigate the Fermi surface of the heavy fermion material YbPdBi, a member of the noncentrosymmetric half-Heusler family of structures. It is isostructural to YbPtBi, which has been found to possess heavy fermions with significantly enhanced effective mass~\cite{fisk_massive_1991} and non-trivial topology~\cite{guo_evidence_2018}. To probe the Fermi surface of YbPdBi, we perform angle-resolved magnetoresistance oscillations (AMRO) in a range of magnetic fields and at temperatures near and below the coherence temperature at which the heavy fermion phase begins to develop. During AMRO measurements, we observe a temperature-dependent transition from a two-fold oscillation symmetry to a four-fold symmetry induced by a magnetic field. This shift in symmetry suggests the geometry of the Fermi surface of YbPdBi is dependent on the magnetic field. \\

% This change in symmetry roughly correlates to a change in the slope of the transverse magnetoresistivity $\rho_{xy}(H)$.\\

\noindent To contextualize these AMRO measurements, we characterized a range of properties of YbPdBi. In magnetization measurements, we find an enhanced Pauli paramagnetism at low temperatures. Powder neutron diffraction data down to $T=0.1$~K finds no clear indication that magnetic ordering occurs, despite evidence to the contrary found in other experiments. In specific heat measurements, we explore a Schottky-like anomaly below $T=10$~K that exhibits a strong dependence on magnetic field strength, which could not be adequately described using crystal electric field (CEF) states. Probing the magnetotransport properties, we observe a potential topological Hall effect (THE) and a quadratic dependence in the longitudinal resistivity on the magnetic field, characteristic of a chiral anomaly. This suggests YbPdBi hosts non-trivial topology and is, therefore, a candidate Weyl-Kondo semimetal. Unexpectedly, evidence for a chiral anomaly was observed not only for parallel electric and magnetic fields but also for orthogonal fields. Surprisingly, the associated chiral coefficients exhibit the expected temperature dependence regardless of the angle between electric and magnetic fields. \\


\noindent This document is structured as follows: Ch.~\ref{ch:hH_HF} discusses background information about fermiology, half-Heuslers, heavy fermion materials, and Weyl semimetals. This provides theoretical context and motivates our experimental efforts. Using the methods mentioned above, we characterize YbPdBi in Ch.~\ref{ch:characterization}. In Ch.~\ref{ch:topology}, we present evidence of the chiral anomaly in magnetotransport measurements, which indicate non-trivial topology. In Ch.~\ref{CH:AMRO_main}, we present the results of our AMRO measurements, where we observe a shift from two-fold to four-fold symmetry. Finally, in Ch.~\ref{ch:conclusion}, we compare and contrast the characterization results against the observed symmetry change in the AMRO data.\\
